%! TEX root = ../../master.tex
\lecture[Decktransformationen von $\exp\colon\R -> S^1$. Normalisatoren. Gruppenisomorphismus $\Delta(p)\cong \cofaktor{H}{N_{\pi_1(X,x_0)}H}$ zwischen Decktransformationen und ~Fundamentalgruppe. Gruppenwirkungen. Semidirekte Produkte und die Fundamentalgruppe des Torus $\mathbb{Z} \rtimes \mathbb{Z}$.]{Di 29 Jun 2021}{Decktransformationen}
\begin{example}
    \begin{enumerate}[1)]
        \item Betrachte wieder die Überlagerung $\exp \colon  \R \to  S^1$.
            \begin{claim*}
                $\Delta(\exp) = \left \{T_n \mid  n\in \mathbb{Z}\right\} $ mit $T_n(x) = x+n$.
            \end{claim*}
            \begin{proof}
                Wir zeigen zunächst, dass es sich überhaupt um Decktransformationen handelt. Offensichtlich sind das Automorphismen von $\R$, wir müssen noch zeigen, dass diese über $S^1$ sind, es ist hierzu
                 \[
                     \exp (T_n(x)) = \exp (x+n) = \exp (x)
                .\] 

                Sei nun $f\in \Delta(\exp )$ eine beliebige Decktransformation. Dann ist $f(0) \in \mathbb{Z}$ weil $\exp ^{-1}(1) \in \mathbb{Z}$ und $f$ ein Morphismus über  $S^1$ ist. Betrachte nun
                    \begin{equation*}
            \varphi \colon \left|        \begin{array}{c c l} 
                    \R & \longrightarrow & \R \\
                    x & \longmapsto &  f(x) - (x + f(0))
                    \end{array}\right.
                \end{equation*}
               Dann ist
               \[
                   \exp (f(x) - (x + f(0)) = \exp (f(x)) \cdot  \exp (x)^{-1} \cdot \exp (f(0))^{-1} = \exp (x) \exp (x)^{-1} = 1
               .\] 
               Also ist $\varphi (\R) \in \mathbb{Z}$, also konstant weil $\mathbb{Z}$ die diskrete Topologie trägt. Mit $f(0) - (0 + f(0)) = 0$ sehen wir $\varphi  \equiv 0$.
                \[
                    f(x) = x + f(0) \quad \implies \quad f = T_{f(0)}
               .\] 
               also haben wir tatsächlich alle solchen Überlagerungen gefunden.
            \end{proof}
        \item Betrachte für $n\geq 2$ die Überlagerung $p\colon  S^n \to  \R\mathbb{P}^n$.
            \begin{claim*}
                Dann ist $\Delta(p) = \left \{\id_{S^n}, ι\right\} $, wobei $ι$ die Antipodalabbildung sein sollte.
            \end{claim*}
            \begin{proof}
                Es ist klar, dass die beiden vorgegebenen Abbildungen Decktransformationen sind.

                Ist $f\in \Delta(p)$, so ist bereits $f(x) = \sgn(x) \cdot x$, wobei
                \[
                \sgn \colon  S^n \to  \left \{-1,1\right\} 
                .\] 
                stetig ist. Die Abbildung $\sgn$ ist also konstant, und es folgt  $f=\id_{S^n}$ für $\sgn \equiv 1$, und $f = ι$ für  $\sgn \equiv -1$.
            \end{proof}

            Also ist $\Delta(p) \cong \mathbb{Z} / 2\mathbb{Z}$, denn es gibt nur eine Gruppe mit 2 Elementen.
    \end{enumerate}
\end{example}

\begin{notation*}
    Ist $H$ eine Untergruppe von  $G$, so schreiben wir  $H \leq G$, nicht wie ansonsten üblich $H \subset G$.
\end{notation*}

\begin{definition}[Normalisator]\label{def:normalisator}
    Sei $H\leq  G$ eine Untergruppe. Der \vocab{Normalisator} von $H$ in  $G$ ist definiert durch
    \[
    N_GH \coloneqq  \left \{g\in G \mid  gH g^{-1} = H\right\} 
    .\] 
    Diest ist eine Untergruppe von $G$.
\end{definition}

\begin{remark*}
    Der Begriff Normalisator kommt daher, dass es sich bei $N_GH$ um die größte Untergruppe von  $G$ handelt, in der  $H$ noch normal ist. Ist  $H$ ein Normalteiler von  $G$, so ergibt sich  $N_GH = G$. Außerdem erkennt man so, dass auch  $H \leq  N_GH$.
\end{remark*}

\begin{remark}
    Die Bedingung $gH g^{-1} = H$ sagt \textit{nicht} notwendigerweise aus, dass $ghg^{-1} = h$ für $h\in H$, nur dass die beiden entstehenden Mengen gleich sind.    
\end{remark}

\begin{remark}
    Per Definition ist $H$ normal in  $N_GH$. Also ist $\cofaktor{H}{N_GH}$ (die Menge der Linksnebenklassen) wieder eine Gruppe mittels
    \[
        (Hg) (Hg') = H(gg')
    .\] 
\end{remark}

\begin{oral}
    Es gibt erstmal keinen Grund, warum wir die \textit{Links}nebenklassen verwenden, für die Rechtsnebenklassen gilt Obiges auch. Wir werden aber später im Zusammenhang mit Gruppentheorie sehen, dass es aus Konsistenzgründen besser ist, jetzt schon mit Linksnebenklassen zu arbeiten, damit wir später nicht zwischen den beiden Konzepten wechseln müssen.
\end{oral}

\underline{Erinnerung} Ist $p\colon  E\to X$ eine Überlagerung mit $X$ zusammenhängend und lokal wegzusammenhängend,  $x_0\in X$ und $e_0\in p^{-1} (x_0)$ sowie $H\coloneqq p_*(\pi_1(E,e_0))$ die charakteristischen Untergruppe. Dann ist
    \begin{equation*}
    \varphi : \left| \begin{array}{c c l} 
        E & \longrightarrow & \left \{\text{Weg $w$ in  $X$}\mid  w(0) = x_0\right\} / \text{$H$-Äquivalenz} \\
e & \longmapsto &  \left[ p \circ  v \right] _H
    \end{array} \right.
\end{equation*}
eine Bijektion, wobei $v$ ein Weg von  $e_0$ nach $e$ sei. Nach \autoref{thm:äquivalenz-von-überlagerungen-über-lokal-wegzsuammenhängendem-wegzusammenhängendem-raum} ist nämlich der Isomorphismus
    \begin{equation*}
    \begin{array}{c c l} 
        E & \longrightarrow & E(H) \\
        e_0 & \longmapsto &  \left[ c_{p(e_0)} \right] _H
    \end{array}
\end{equation*}
durch Wegeliftung bestimmt. $\left[ p \circ  v \right] _H$ ist nun aber per Definition der Endpunkt der Liftung von $p \circ  v$ mit Startpunkt $\left[ c_{p(e_0)} \right] _{H}$, also genau das Bild von $e$ unter dem bereits bekannten Isomorphismus.

Insbesondere erhalten wir auch eine Bijektion
\[
    \varphi (p^{-1} (x_0)) = \faktor{\pi_1(X,x_0)}{\text{$H$-Äquivalenz}}
.\] 

\begin{restatable}{theorem}{ThmDecktransformationNormalisator}\label{thm:isomorphismus-von-decktransformationen-mit-nebenklassengruppe-von-charakteristischer-untergruppe-in-seinem-normalisator}
    Seien $E,X,p,x_0,e_0,H$ wie oben.
    \begin{enumerate}[1)]
        \item Für $f\in \Delta(p)$ ist
            \[
                \varphi (f(e_0)) \in \cofaktor{H}{N_{\pi_1(X,x_0)}H}
            .\] 
        \item Die Abbildung
                \begin{equation*}
                \psi : \left| \begin{array}{c c l} 
                    \Delta(p) & \longrightarrow & \cofaktor{H}{N_{\pi_1(X,x_0)}H} \\
                    f & \longmapsto &  \varphi (f(e_0))
                \end{array} \right.
            \end{equation*}
           ist ein Gruppenisomorphismus. 
    \end{enumerate}
\end{restatable}

\begin{dcorollary}\label{cor:bei-universeller-überlagerung-sind-decktransformationen-isomorph-zu-fundamentalgruppe}
    Ist $p$ die universelle Überlagerung von  $X$, so ist
    \[
        \Delta(p) \cong \pi_1(X,x_0)
    .\] 
\end{dcorollary}

\begin{proof}
    Ist $p$ die universelle Überlagerung, so ist  $H = p_*(\pi_1(E,e_0)) = \left \{1\right\} $ die triviale Untergruppe, d.h. $N_{\pi_1(X,x_0)}H = \pi_1(X,x_0)$.
\end{proof}

\begin{proof}[Beweis von \autoref{thm:isomorphismus-von-decktransformationen-mit-nebenklassengruppe-von-charakteristischer-untergruppe-in-seinem-normalisator}]
    \begin{enumerate}[1)]
        \item Es ist $f$ eine Hebung von  $p$, denn es kommutiert ja
            \[
             \begin{tikzcd}
                & E \ar{d}{p} \\
                 E \ar{ur}{f} \ar[swap]{r}{p} & X
            \end{tikzcd}
        \]
    Nach dem \nameref{thm:allgemeiner-liftungssatz} ergibt sich nun
    \[
        H = p_*(\pi_1(E,e)) \leq  p_*(\pi_1(E,f(e_0))
    .\] 
    Sei $v$ ein Weg von  $e_0$ zu $f(e_0)$. Dann ist
    \[
        H\leq p_*(\pi_1(E,f(e_0)) = [p \circ  v] ^{-1} \star p_*(\pi_1(E,e_0)) \star [p \circ  v]
    .\]
    Also ergibt sich zusammen
    \[
        [p \circ  v] ^{-1} \star H \star [p \circ  v] \leq H
    .\] 
    Vertauschen von $e_0,f(e_0)$ sowie Ersetzung von $f$ durch  $f^{-1}$ ergibt nun, dass auch
    \[
        [p \circ  v] \star H \star [p \circ v] ^{-1} \leq H
    .\] 
    und aus Symmetrie folgt dann
    \[
        [p \circ  v] \star H \star [p \circ  v] ^{-1} = H
    .\] 
    also folgt $[p \circ v] \in N_{\pi_1(X,x_0)} H$, und somit insbesondere
    \[
        \varphi (f(e_0)) = [p \circ  v]_H \in \cofaktor{H}{N_{\pi_1(X,x_0)} H}
    .\] 
\item
    \begin{description}
        \item[Injektivität] Seien $f,g \in \Delta(p)$ mit
            \[
                \varphi (f(e_0)) f \psi (f) = \psi (g) = \varphi (g(e_0))
            .\] 
            so folgt aus Injektivität von $\varphi $ zunächst $f(e_0) = g(e_0)$, aber $f,g$ sind Hebungen von  $p$ und stimmen in  $e_0$ überein, also gilt $f=g$ nach der Eindeutigkeit im  \nameref{thm:allgemeiner-liftungssatz}.
        \item[Surjektivität] Sei $[w] \in N_{\pi_1(X,x_0)}H$ ein Element aus dem Normalisator. Sei $e_0' \coloneqq  L(w,e_0)(1)$ der Endpunkt der Hebung von $e_0'$.

            Es ist nach \autoref{thm:charakteristische-untergruppen-innerhalb-der-faser-sind-nur-konjugiert-wenn-weg-zwischen-urbildern-existiert} nun:
            \begin{IEEEeqnarray*}{rCl}
                p_*(\pi_1(E,e_0')) & = & [\underbrace{p \circ  L(w,e_0)}_{=w^{-1}}]^{-1}\star H \star [\underbrace{p \circ  L(w,e_0)}_{= w}] \\
                                  & \stackrel{[w]\in N_{\pi_1(X,x_0)}H}{=}  & H
            \end{IEEEeqnarray*}
            Nach \autoref{thm:äquivalenz-von-überlagerungen-über-lokal-wegzsuammenhängendem-wegzusammenhängendem-raum} gibt es also einen Homöomorphismus $f\colon  E \stackrel{\cong}{\longrightarrow} E$ mit $p \circ  f = p$ und $f(e_0) = e_0'$, bei $f$ handelt es sich dann um eine Decktransformation, und wir sehen leicht
             \[
                 \psi (f) = \varphi (f(e_0)) = \varphi (e_0') = [p \circ  L(w,e_0)]_H = [w]_H
            .\] 
        \item[Gruppenhomomorphismus]
            Seien $f,g \in \Delta(p)$ und $v$ ein Weg von  $e_0$ nach $f(e_0)$ sowie $v'$ ein Weg von  $e_0$ nach $g(e_0)$. Dann ist $g \circ v$ ein Weg von $g(e_0)$ nach $g(f(e_0))$ und $v' \star (g \circ v)$ ein Weg von $e_0$ nach $g(f(e_0))$.

            Es ist nun
            \begin{IEEEeqnarray*}{rCl}
                \psi (g \circ f) & = & [p \circ  (v\star (g \circ v))]_H \\
                                 & = & H[p \circ  (v' \star (g \circ v))] \\
                                 & = & H[(p \circ  v') \star p(g \circ v)] \\
                                 & = & H[p\circ v')] \star H[p (g \circ v)] \\
                                 & \stackrel{p \circ  g = p}{=}  & H[p \circ v'] \star H[p \circ v] \\
                                 & = & \psi (g) \star \psi (f)
            \end{IEEEeqnarray*}
    \end{description}
    \end{enumerate}
\end{proof}

\begin{remark*}
    Hier kam in der Vorlesung etwas zu Gruppen und Nebenklassen etc. Das wird noch nachgeholt, ich war mit Links- / Rechtsnebenklassen und deren Bezeichnung selbst etwas verwirrt.
\end{remark*}


\begin{definition}[$G$-Menge]\label{def:g-menge}
    Sei $G$ eine Gruppe. Eine \vocab[$G$-Menge]{(Rechts-) $G$-Menge} ist eine Menge  $X$ mit einer Abbildung
        \begin{equation*}
        .: \left| \begin{array}{c c l} 
        X\times G & \longrightarrow & X \\
        (x,g) & \longmapsto &  x.g
        \end{array} \right.
    \end{equation*}
    für die gilt
    \begin{enumerate}[i)]
        \item $x.1 = x$ für  $x\in X$
        \item $(x.g).h = x.(gh)$ für  $x\in X$ und $g,h\in G$.
    \end{enumerate}
    Ein \vocab{Homomorphismus von $G$-Mengen} ist eine Abbildung $f\colon X \to  Y$, so dass
\[
    f(x.g) = f(x).g \qquad \forall x\in X, g\in G
.\] 
\end{definition}

\begin{remark}[Ent- oder auch Verwirrung bezüglich Rinks und Lechts]
    Analoges können wir für links-$G$-Mengen definieren, dann fordern wir  $h.(g.x) = (hg).x$. Eine Links- $G$-menge ist eine Rechts-$G$-Menge mittels
     \[
    x.g \coloneqq g^{-1}x.
    .\] 
    bzw. wir können auch eine Links-$G$-Menge als Rechts-$G^{\op}$-Menge auffassen.

Eine Links-$G$-Menge ist nämlich nichts anderes als Ein Funktor  $\mathcat{G} \to  \Set$, und eine Rechts-$G$-Menge ist ein Funktor  $\mathcat{G}^{\op} \to  \Set$, aber wegen $\mathcat{G} \cong \mathcat{G}^{\op}$ sind die Konzepte äquivalent.
\end{remark}

\begin{trivial}
    $G$-Mengen sind also genau Funktoren  $\mathcat{G} \to  \Set$, und Morphismen von $G$-Mengen  $X,Y \colon  \mathcat{G} \to  \Set$ sind genau natürliche Transformationen $X \to Y$.

    Die Kategorie der $G$-Mengen über einer Gruppe  $G$ ist dann genau die Funktorkategorie  $[\mathcat{G},\Set]$.
\end{trivial}

\begin{example}
    Ist $H\leq G$, so ist $\cofaktor{H}{G}$ eine  $G$-Menge mittels  $(Hg)g' = H(gg')$.
\end{example}

\begin{nexample}\label{ex:fundamentalgruppe-der-kleinschen-flasche-mittels-semidirektem-produkt-z-z}
    Das semidirekte Produkt $\mathbb{Z} \rtimes Z$ von $\mathbb{Z}$ mit $\mathbb{Z}$ ist die Gruppe mit Menge $\mathbb{Z}\times \mathbb{Z}$ und Verknüpfung
    \[
        (n,m)(n',m') = (n+(-1)^m n', m + m')
    .\] 
    Es gibt eine Rechtswirkung
        \begin{equation*}
        \begin{array}{c c l} 
            \R^2\times (\mathbb{Z}\rtimes \mathbb{Z}) & \longrightarrow & \R^2 \\
            ((x,y),(n,m)) & \longmapsto &  ((-1)^m(x+n),y+m)
        \end{array}
    \end{equation*}
    (Übung, dass das die Axiome erfüllt).
    \begin{claim*}
        $\forall (x,y) \in \R^2 \; \exists (n,m) \in \mathbb{Z}\rtimes \mathbb{Z}$, sodass
        \[
            (x,y).(n.m) \in I^2
        .\] 
    \end{claim*}
    \begin{proof}
        Sei $(x,y) \in \R^2$ beliebig. Zunächst existiert ein $m\in \mathbb{Z}$, sodass $y + m \in I$, und dann existiert ein $n\in \mathbb{Z}$ mit $(-1)^m x + n\in I$, dann ist
        \[
            (x,y) . \underbrace{(0,m).(n,0)}_{=((-1)^mn, m)} = ((-1)^mx, y+m)(n,0) = ((-1)^mx + n, y+m)
        .\] 
    \end{proof}
        Falls $x\not\in \mathbb{Z}, y\not\in \mathbb{Z}$, so sind $n,m$ eindeutig bestimmt. Mittels den Gleichungen
        \[
            (t,0).(0,1) = (-t,1) \qquad (0,t).(1,0) = (1,t)
        .\] 
        folgt nun, dass $\faktor{\R^2}{(\mathbb{Z} \rtimes \mathbb{Z})}$ die Kleinsche Flasche ist.
        \begin{claim*}
            Die Abbildung $\R^2 \to  \faktor{\R^2}{\mathbb{Z} \rtimes \mathbb{Z}}$ ist eine Überlagerung und $\mathbb{Z} \rtimes \mathbb{Z}$ sind die Decktransformationen.
        \end{claim*}
        \begin{proof}
            Übung.
        \end{proof}
        Da $\R^2$ einfach zusammenhängend ist, folgt mit \autoref{cor:bei-universeller-überlagerung-sind-decktransformationen-isomorph-zu-fundamentalgruppe}, dass
        \[
            \pi_1(K) \cong \mathbb{Z} \rtimes \mathbb{Z}
        .\] 
\end{nexample}

\begin{oral}
    Semidirekte Produkte machen zwischen beliebigen Gruppen $G, H$ Sinn, wenn wir eine Wirkung von $G$ auf  $H$ haben. In obigem Fall gibt es aber nur eine nichttriviale Wirkung von  $\mathbb{Z}$ auf $\mathbb{Z}$, weswegn wir von \textit{dem} semidirekten Produkt $\mathbb{Z} \rtimes \mathbb{Z}$ sprechen können. 
\end{oral}
