%! TEX root = ./master.tex
\lecture[Wege $x\to x'$ induzieren Isomorphismen  $\pi_1(X,x) \cong \pi_1(X,x')$. Fundamentalgruppoiden. Überlagerungen. Lokale Homöomorphismen. Überlagerungen von $S^1$, triviale Überlagerungen. Blätterzahl. Teaser der Liftungssätze.]{Do 10 Jun 2021 10:15}{Fundamentalgruppen und Überlagerungen}

\begin{theorem}[$\pi_1$ auf Wegzusammenhangskomponenten]\label{thm:pi-1-ist-gleich-auf-wegusammenhangskomponente}
    Seien $x,x' \in X$ durch einen Weg verbunden, d.h. $\exists w\colon  I \to X$ mit $w(0) = x$,  $w(1) = x'$. Dann sind die Gruppen  $\pi_1(X,x)$ und $\pi_1(X,x')$ homöomorph.
\end{theorem}

\begin{remark}
    Liegen $x,x'$ in verschiedenen Wegzusammenhangskomponenten, so gibt es (im Allgemeinen) keine Beziehung zwischen  $\pi_1(X,x)$ und $\pi_1(X,x')$. Betrachte hierzu z.B. $X = X_1 \coprod X_2$ mit $x\in X_1$ und $x' \in X_2$, so ist $p_1(X,x) = p_1(X_1,x)$ und $p_1(X,x') = p_1(X_2,x')$. Es genügt daher zu sehen, dass es Räume mit verschiedenen Fundamentalgruppen gibt.
\end{remark}

\begin{figure}[ht]
    \centering
    \incfig{disjunkte-vereinigung-von-räumen-mit-verschiedener-fundamentalgruppe}
    \caption{Disjunkte Vereinigung von Räumen mit verschiedener Fundamentalgruppe}
    \label{fig:disjunkte-vereinigung-von-räumen-mit-verschiedener-fundamentalgruppe}
\end{figure}

\begin{proof}[Beweis von \autoref{thm:pi-1-ist-gleich-auf-wegusammenhangskomponente}]
    Sei $v\colon  I \to  X$ eine Weg von $x$ nach  $x'$. Dann definiere
        \begin{equation*}
        v_*: \left| \begin{array}{c c l} 
            \pi_1(X,x') & \longrightarrow & (X,x) \\
            \left[w\right] & \longmapsto &  \left[v \star w \star \overline{v}\right]
        \end{array} \right.
    \end{equation*}
    wobei wie üblich $\overline{v}\colon [0,1] \to X$ gegeben ist durch $\overline{v}(t) = v(1-t)$.
\begin{figure}[ht]
    \centering
    \incfig{schleifen-zwischen-verschiedene-basispunkten-durch-weg}
    \caption{Aus einer Schleife $w$ um  $x'$ können wir mittels  $v,\overline{v}$ eine Schleife um $x$ erzeugen}
    \label{fig:schleifen-zwischen-verschiedene-basispunkten-durch-weg}
\end{figure}
    Hierbei definieren wir
    \[
        (v \star w \star \overline{v})(t) = \begin{cases}
            v(3t) & 0 \leq  t \leq  \frac{1}{3}\\
            w(3t-1) & \frac{1}{3}\leq t \leq  \frac{2}{3} \\
            v(3(1-t)) & \frac{2}{3} \leq  t \leq  1
        \end{cases}
    .\] 
    Damit diese Konstruktion ein Gruppenisomorphismus ist, zeigen wir:
    \begin{description}
        \item[Wohldefiniertheit] Für $w \stackrel{H}{\sim } w'$ ist auch $v \star w \star \overline{v} \sim  v \star w' \star \overline{v}$ mittels der Homotopie
            \[
                H(t,s) = \begin{cases}
                    v(3t) & 0\leq t\leq \frac{1}{3} \\
                    H(3t-1,s) & \frac{1}{3}\leq  t \leq  \frac{2}{3} \\
                    v(3(1-t)) & \frac{2}{3}\leq t\leq 1
                \end{cases}
            .\] 
        \item[Gruppenhomomorphismus] Es ist:
            \begin{IEEEeqnarray*}{rCl}
                v \star (w \star w') \star \overline{v} &\stackrel{\overline{v}\star v \sim c_{x'}}{=} &v \star (w \star (\overline{v} \star v) \star w') \star \overline{v} \\
                                                        & \stackrel{\text{assoz.}}{=}  & (v \star w \star \overline{v}) \star (v \star w' \star \overline{v}) \\
                                                        & = & v_*(w) \circ v_*(w')
            \end{IEEEeqnarray*}
        \item[Isomorphismus] $\overline{v}$ induziert analog zu $v$ eine Abbildung
             \[
                 \overline{v}_*\colon  \pi_1(X,x) \to  \pi_1(X,x')
            .\] 
            Wir behaupten, dass $\overline{v}_*$ ein Inverses ist, also $v_* \circ  \overline{v}_* = \id_{\pi_1(X,x)}$, dann folgt wegen $\overline{\overline{v}} = v$ auch sofort $\overline{v}_* \circ  v_ = \id_{\pi_1(X,x')}$. 

            Sei also $[w] \in  \pi_1(X,x)$, dann ist
            \begin{IEEEeqnarray*}{rCl}
                (                v_* \circ  \overline{v}_*)([w]) & = & [v \star \overline{v} \star w \star v \star \overline{v}] \\
                                                                 & = & [w]
            \end{IEEEeqnarray*}
            also ist $v_* \circ  \overline{v}_*$ tatsächlich die Identität.
    \end{description}
\end{proof}

\begin{warning}
    Der Isomorphismus $v_*\colon  \pi_1(X,x') \to  \pi_1(X,x)$ hängt von $v$ ab (genauer von der Homotopieklasse von  $v$ relativ Endpunkten). Die Gruppen sind aber in jedem Fall isomorph.    
\end{warning}

\begin{recap}
    Ist $G$ eine Gruppe,  $g\in G$, so ist
        \begin{equation*}
        c_g: \left| \begin{array}{c c l} 
        G & \longrightarrow & G \\
        h & \longmapsto &  ghg^{-1}
        \end{array} \right.
    \end{equation*}
    eine Automorphismus von $G$, die  \vocab[Gruppe!Konjugation]{Konjugation mit $g$}. Diese heißen \vocab[Gruppe!innerer Automorphismus]{innere Automorphismen} 
\end{recap}

\begin{dremark}\label{rm:verschiedene-wege-induzieren-konjugierte-isomorphismen-der-fundamentalgruppen}
Sind $v,v' \colon  I \to  X$ Wege von $x$ nach  $x'$, dann gilt:
\begin{IEEEeqnarray*}{rCl}
    v_*([w]) &=& v\star [w] \star \overline{v}\\
             & = & [v \star \overline{v'}]  \star v_*'([w]) \star [v' \star \overline{v}]\\& = & c_{[v \star \overline{v'}]} (v_*'[w])
\end{IEEEeqnarray*}
die von verschiedenen Wegen $v$, $v'$ von $x$ nach  $x'$ induzierent Wege unterscheiden sich also nur um Konjugation.
\end{dremark}

\begin{remark*}
    Es gibt auch Wege $v,w \colon  x' \to x$, die nicht in der gleichen Homotopieklasse (relativ Endpunkten) liegen, die aber dennoch denselben Isomorphismus induzieren. Das liegt daran, dass die beiden induzierten Isomorphismen sich nur um Konjugation unterscheiden (s. \autoref{rm:verschiedene-wege-induzieren-konjugierte-isomorphismen-der-fundamentalgruppen}), und damit z.B. insbesondere für abelsche Fundamentalgruppen denselben Isomorphismus induzieren.
\end{remark*}

\begin{oral}
    Man muss aufpassen, bei wegzusammenhängenden Räumen einfach von 'der' Fundamentalgruppe zu sprechen. Es gibt zwar nur eine, allerdings sind die Isomorphismen nicht zwingend kanonisch, weswegen man trotzdem implizit noch die Wahl des Basispunktes mit sich rumschleppt. Das motiviert die Betrachtung der nächsten Definition:
\end{oral}


\begin{definition}[Fundamentalgruppoid]\label{def:fundamentalgruppoid}
    Der \vocab{Fundamentalgruppoid} $\Pi(X)$ ist die Kategorie mit  $\ob(\Pi(X)) = X$ und
     \[
         \Mor_{\Pi(X)}(x,x') = \faktor{\left \{w\colon  I \to  X \mid  w(0) = x, w(1) = x'\right\}}{\sim \text{ rel } \left \{0,1\right\} }
    .\] 
    d.h. die Morphismen sind genau die Wege von $x$ nach $x'$ modulo Homotopie bezüglich endpunkten. Die Verknüpfung der Morphismen ist durch $\star$ gegeben.
\end{definition}

\begin{remark*}
    \begin{enumerate}[1)]
        \item Ein \vocab{Gruppoid} ist eine Kategorie, in der jeder Morphismus invertierbar ist, d.h. in der jedere Morphismus ein Isomorphismus ist. In obigem Fall ist das gegeben, weil wir für $w$ den Weg $\overline{w}$ als Inverses haben.
            \begin{warning}
                Wir fordern nicht, dass zwischen je zwei Objekten ein Isomorphismus existiert. Das ist auch in $\Pi(X)$ nicht der Fall, wenn  $X$ nicht wegzusammenhängend ist.
            \end{warning}
            Ist $\cat{G}$ ein Gruppoid, so ist $\Mor_{\cat{G}}(A,A)$ für $A\in \Ob(\cat{C})$ stets eine Gruppe. Vergleiche hierzu auch \autoref{ex:mehr-kategorien:gruppen-produkte-wedge-smash-einhängung-funktorkategorie} 1).
        \item Per Definition ist $\Mor_{\Pi(X)}(x,x) \cong \pi_1(X,x)$ (als Gruppen).
        \item Per Definition ist nun
            \[
                \pi_0(X) \cong \faktor{\Ob(\Pi(X))}{\text{Isomorphi}}
            .\] 
    \end{enumerate}
\end{remark*}

\begin{oral}
    Obige Konstruktion ist funktoriell (Zuordnung $X \to  \Pi(X)$). Dadurch 'verpacken' wir die Basispunktwahl in eine Kategorie und umgehen so willkürliche Wahlen, können die Informationen aus der Kategorie (über die Fundamentalgruppe) aber jederzeit 'zurückgewinnen' (bzw. haben sie schon).
\end{oral}


\section{Überlagerungen Teil 1}

\begin{definition}[Überlagerung]\label{def:überlagerung}
    Eine \vocab{Überlagerung} ist eine stetige, surjektive Abbildung
    \[
    p\colon  E \to  X
    .\] 
    mit den folgenden Eigenschaften:
    Für jedes $x\in X$ gibt es eine Umgebung $U$ von  $x$, einen diskreten Raum  $F$ und einen Homöomorphismus
     \[
         v\colon  p^{-1}(U) \stackrel{\cong}{\longrightarrow} U \times F
    .\] 
    über $U$, d.h.
    \[
\begin{tikzcd}
    p^{-1}(U) \ar{rr}{v} \ar[swap]{dr}{p}& &  U \times F \ar{dl}{\pr_U} \\
                          & U
\end{tikzcd}
\]
\end{definition}

\begin{figure}[ht]
    \centering
    \incfig{definition-überlagerung}
    \caption{Veranschaulichung einer trivialen Überlagerung (links), und der Überlagerung $\R\stackrel{\exp }{\longrightarrow}  S^1$}
    \label{fig:definition-überlagerung}
\end{figure}


\begin{remark}
    $F$ ist homöomorph zu  $F_x = p^{-1}(\left \{x\right\} )$, der \vocab{Faser über $x$} mittels
    \[
        v|_{p^{-1}(x)} \colon  p^{-1}(x) \to  \left \{x\right\} \times F
    .\] 
\end{remark}


\begin{lemma}[Überlagerung von Teilräumen]\label{lm:überlagerung-von-teilräumen}
    Sei $p\colon E \to  X$ eine Überlagerung, $Y\subset X$ ein Teilraum. Dann ist auch
    \[
        p|_{p^{-1} (Y)} \colon  p^{-1} (Y) \to  Y
    .\] 
    eine Überlagerung.
\end{lemma}

\begin{proof}
    Sei $x\in Y$. Dann $\exists U\subset X$ Umgebung von $x$ und  $F$ diskret und ein Homöomorphismus
     \[
         v\colon  p^{-1} (U) \to  U\times F
    .\] 
    über $U$. Dann ist $U\cap Y$ eine umgebung von $x$ in $Y$ und somit
     \[
         v|_{p^{-1} (U\cap Y)}\colon p^{-1} (U\cap Y) \to  (U\cap Y) \times F
    .\] 
    ein Homöomorphismus.
\end{proof}

\begin{example}
    Sei $X$ ein topologischer Raum,  $F$ diskret. Dann ist
     \[
    \pr_X \colon  X \times F \to  X
    .\] 
    eine Überlagerung.
\end{example}

\begin{remark}
    Ist $p\colon  E \to  X$ eine Überlagerung, so heißt diese \vocab[Überlagerung!trivial]{trivial}, falls ein Homöomorphismus $u\colon  E \to  X\times F$ über $X$ existiert, wobei  $F$ diskret.
\end{remark}

\begin{definition}
    $f\colon X \to  Y$ ist ein \vocab{lokaler Homöomorphismus}, falls $\forall x\in X$ eine offene Umgebung $x\in V \subset X$ existiert mit
    \begin{enumerate}[i)]
        \item $f(V)\subset Y$ ist offen
        \item $f|_V\colon  V \to  f(V)$ ist ein Homöomorphismus
    \end{enumerate}
\end{definition}

\begin{oral}
    Die Abbildung $f|_V \to  f(V)$ erfüllt etwas stärkere Eigenschaften als eine \nameref{def:einbettung}, weil wir hier zusätzlich fordern, dass $f(V)\subset Y$ offen ist.

    Es ist z.B. $\R \hookrightarrow \R^2$ eine Einbettung, jedoch kein lokaler Homöomorphismus.
\end{oral}

\begin{lemma}\label{lm:überlagerung-ist-lokaler-homöomorphismus}
    Eine Überlagerung ist ein lokaler Homöomorphismus.
\end{lemma}

\begin{proof}
   Sei $p\colon  E \to  X$ eine beliebige Überlagerung. Für $x\in X$ existiert $x\in U\subset X$ offen mit $u\colon p^{-1} (U) \cong U\times p^{-1} (\left \{x\right\} )$ (per Definition der Überlagerung) 

   Sei nun $e\in E$ beliebig und setze $x = p(e)$. Wähle  $V = u^{-1}(U\times \left \{e\right\} )$. Dann ist $p|_V \colon  V \to  U$ ein Homöomorphismus, denn (nach Anwendung von $u$ bzw.  $u|_V$) für $U\times \left \{e\right\} \to U$ ist dies trivial.
\[
    \begin{tikzcd}
        V \ar[out=220, in = 210,looseness=2,blue, near start,swap]{ddr}{p|_V} \ar{rr}{\cong}[swap]{u|_V}\ar[phantom]{d}[rotate=-90]{\subset }& &U \times \left \{e\right\} \ar[phantom]{d}[rotate=-90]{\subset } \ar[out=-30, in = -30,looseness=2,blue,dashed,near start]{ddl}{\cong}\\
        p^{-1} (U) \ar[swap]{dr}{p|_{p^{-1} (U)}} \ar{rr}{\cong}[swap]{u} & & U\times p^{-1} (\left \{x\right\} ) \ar{dl}{\pr_U}\\
                   & U 
    \end{tikzcd}
\]
Da $p^{-1} (U)\subset E$ offen ist (Stetigkeit von $p$) und 
\[V \stackrel{u}{\cong}U\times \left \{e\right\} \subset U\times p^{-1} (\left \{x\right\} ) \stackrel{u^{-1}}{\cong} p^{-1} (U)
\]
offen, ist $V\subset p^{-1} (U)\subset E$ offen. Also ist $V$ eine offene Umgebung von  $e$, und  $p|_V \colon  V \stackrel{\cong}{\longrightarrow} U$ ein Homöomorphismus mit offenem Bild $U\subset X$, und wir erfüllen alle Eigenschaften eines lokalen Homöomorphismus.
\end{proof}

\begin{lemma}\label{lm:offene-überdeckung-von-überlagerung-mit-homöomorphismen}
    Sei $p\colon  E \to  X$ eine Überlagerung. Dann gibt es eine offene Überdeckung $\left \{U_i\right\} _{i \in I}$ von $X$, so dass gilt:

    Für alle $i\in I$, $x\in U_i$ und $y\in p^{-1} (x)$ gibt es eine stetige Funktion $s\colon  U_i \hookrightarrow   E$ mit 
    \begin{itemize}
        \item $s(x) = y$
        \item  $p \circ s = \id_{U_i}$
    \end{itemize}
\end{lemma}

\begin{oral}
    Im Wesentlichen passiert hier nicht viel mehr als beim lokalen Homöomorphismus. Der Unterschied der Aussage des Lemmas ist, dass wir hier die  $U_i$ und das $y$ fest wählen können, und erst dann den Homöomorphismus bauen.
\end{oral}

\begin{proof}[Beweis von \autoref{lm:offene-überdeckung-von-überlagerung-mit-homöomorphismen}]
    Für jedes $x\in X$ gibt es eine offene Umgebung $U_x$, auf der $p$ trivial ist (im Wesentlichen die Definition einer Überlagerung). Dann ist $\left \{U_x\right\} _{x \in X}$ eine offene Überdeckung, die die Eigenschaften des Lemmas erfüllt:
    \[
    \begin{tikzcd}
        E &        p^{-1} (U_x) \ar{r}{\cong}[swap]{v} \ar[phantom]{l}{\supset }& U_x \times F \\
          &        U_x \ar{r}{\cong} \ar[out = 20, looseness=4, in=-15,blue,near end]{ul}{s}& U_x \times \left \{f\right\} \ar[hook]{u}
    \end{tikzcd}
\]
$s$ als Verknüpfung der anderen Abbildungen hat nämlich genau die gewünschten Eigenschaften, sofern wir  $f$ natürlich so wählen, dass  $y \stackrel{v}{\leftrightarrow} (x,f)$ in Bijektion stehen.
\end{proof}

\begin{remark}
    Nicht jeder lokale Homöomorphismus ist eine Überlagerung. Die Abbildung $(0,2) \stackrel{\exp}{\longrightarrow} S^1$ ist ein lokaler Homöomorphismus, aber keine Überlagerung.

    Das 'Problem' ist hierbei, dass $1\in S^1$ nur ein Urbild unter $\exp$ hat, jede Umgebung von  $1\in (0,2)$ im Urbild ist hat jedoch drei Urbilder von den Punkten nahe $1$, ist also nicht einfach nur ein Intervall.
\end{remark}

\begin{notation*}
    Ist $p\colon  E \to  X$ eine Überlagerung, so nennen wir
    \begin{itemize}
        \item $X$ die  \vocab[Überlagerung!Basis]{Basis} oder den \vocab[Überlagerung!Basisraum]{Basisraum}  
        \item $E$ den  \vocab[Überlagerung!Totalraum]{Totalraum}
        \item $p$ die  \vocab[Überlagerung!Überlagerungsabbildung]{Überlagerungsabbildung} oder \vocab[Überlagerung!Überlagerungsprojektion]{Überlagerungsprojektion}  
        \item $p^{-1} (x)$ die \vocab{Faser über $x$}, $F_x$ 
        \item $\abs{p^{-1} (x)} $ die \vocab[Überlagerung!Blätterzahl]{Blätterzahl} . Diese ist lokal konstant.
    \end{itemize}
\end{notation*}

\begin{example}\label{ex:überlagerungen}
    \begin{enumerate}[1)]
        \item Die triviale Überlagerung
        \item Die unendlich-blättrige Überlagerung
                \begin{equation*}
                \exp: \left| \begin{array}{c c l} 
                \R & \longrightarrow & S^1 \\
                t & \longmapsto &  e^{2\pi it}
                \end{array} \right.
            \end{equation*}
        \item Es gibt auch eine $k$-blättrige Überlagerung des Einheitskreises:
                \begin{equation*}
                    ()^k : \left| \begin{array}{c c l} 
                S^1 & \longrightarrow & S^1 \\
                z & \longmapsto &  z^k
                \end{array} \right.
            \end{equation*}
            (wir fassen hier $S^1 \subset \C$ auf, um $z^k$ zu definieren). 


            \begin{minipage}{\textwidth}
                \centering
                \incfig{k-blättrige-überlagerung-des-kreises}
                \captionof{figure}{4-blättrige Überlagerung $S^1 \to  S^1$}
                \label{fig:k-blättrige-überlagerung-des-kreises}
            \end{minipage}
        \item Sind  $E,E' \stackrel{p,p'}{\longrightarrow} X$ Überlagerungen, so auch
            \[
            E \coprod E' \stackrel{p \coprod p'}{\longrightarrow} X
            .\] 
        \item Die Projektion $S^n \stackrel{p}{\longrightarrow} \R \mathbb{P}^n \cong \faktor{S^n}{x \sim  -x}$ ist eine 2-blättrige Überlagerung. Bezeichne hierzu mit $N$,  $S$ den Nord- und den Südpol von  $S^n$ (die den gleichen Punkt in  $\R\mathbb{P}^n$ darstellen und wegen Symmetrie ein generischer Punkt aus $\R\mathbb{P}^n$ sind). Dann können wir die offene ober bzw. untere Halbkugel betrachten, d.h.
            \begin{IEEEeqnarray*}{rCl}
                S^n \cap  \left \{(x_1,\ldots,x_{n+1})\mid x_{n+1} > 0\right\} \\
                S^n \cap  \left \{(x_1,\ldots,x_{n+1})\mid x_{n+1}<0\right\} 
            \end{IEEEeqnarray*}
            deren Projektion auf $\R\mathbb{P}^n$ dann genau eine 2-blättrige triviale Überlagerung auf eine offene Umgebung von $p(N) = p(S)$ darstellt.
    \end{enumerate}    
\end{example}


\begin{oral}
    Man kann zeigen, dass die endlichen Überlagerungen die einzigen \textit{zusammenhängenden} Überlagerungen von $S^1$ sind. Für 'schöne' Räume werden wir diese auch noch klassifizieren.
\end{oral}

In der nächsten Woche behandeln wir \textit{Liftungssätze}, d.h. wir fragen uns
\[
\begin{tikzcd}
    & E \ar{d}{p} \\
    T \ar[dashed]{ur}{\exists \tilde{f}?} \ar[swap]{r}{f} & X
\end{tikzcd}
\]
das ganze gilt z.B. für Wege, also wenn $T = I$, wir können uns dann ein Urbild  $e \mapsto f(0) = x$ wählen und erhalten  $\tilde{f}$ mit $\tilde{f}(0) = e$, sodass obiges kommutiert.

\begin{oral}[ca.]
    Wir werden dann feststellen, dass die Hebung von $w\colon  I \to  X$ nicht zwingend eine Schleife ist, auch wenn $w$ es war. Wenn wir dann auch noch Homotopien heben können, so können wir schließen, dass  $w$ nicht nullhomotop war, weil das sonst auch für den gehobenen Weg gelten müsste, dieser aber nichtmal dieselben Endpunkte hat.
\end{oral}
