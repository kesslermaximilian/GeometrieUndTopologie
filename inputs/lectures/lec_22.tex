%! TEX root = ../../master.tex
\lecture{Do 08 Jul 2021}{Beweis Seifert van Kampen}
%\lecture[Bijektion zwischen Orbits und Wegekomponenten. Morphismen von transitiven $g$-Mengen. Alternativer Beweis des Isomorphismus  $\Delta(p) \cong H \backslash N_{\pi_1(X,x_0)}H$. Beginn des Beweises von Seifert-van-Kampen.]{Do 08 Jul 2021}{Beweis des Hauptsatzes der Überlagerungstheorie}

Wir wollen nun noch den Satz von \nameref{thm:seifert-van-kampen} beweisen.

\begin{proof}[Beweis von \autoref{thm:seifert-van-kampen}]
    Wir wollen zunächst eine Surjektion
    \[
\psi \colon         \pi_1(U_1,x_0) \star \pi_1(U_2,x_0) \twoheadrightarrow \pi_1(X,x_0)
    .\] 
    finden, wie auch schon in der letzten Vorlesung angedeutet. Da $I$ kompakt ist, existiert  für jeden Weg $[w] \in \pi_1(X,x_0)$ ein $k\in \N$, sodass
    \[
w_l \coloneqq         w|_{\left[ \frac{l}{k}, \frac{l+1}{k} \right) } \subset U_i
    .\] 
    für alle $l$. Für alle  $l\in \left \{1,k-1\right\}$ wähle einen Weg $v_l$ von  $w\left( \frac{l}{k} \right) $ nach $x_0$ in 
    \[
    \begin{cases}
        U_3 & \text{falls } w\left( \frac{l}{k} \right) \in U_3 \\
        U_1 & \text{falls } w\left( \frac{l}{k} \right) \not\in U_2 \\
        U_2 & \text{falls } w\left( \frac{l}{k} \right)  \not\in U_1
    \end{cases}
    .\] 
    Wir können nun schreiben
    \[
        [w] = [w_0\star v_1] \star [v_1^{-1} w_1 v_2] \star \ldots \star [v_{k-1}^{-1}w_{k-1}]
    .\] 
    und jeder der geklammerten Wege verläuft nun in einem der $U_i$.
\end{proof}

\begin{remark*}
    An dieser Stelle fehlt eine Skizze zum weiteren Verlauf des Beweises.
\end{remark*}

