\begin{aufgabe} Es sei $f\colon X\to Y$ eine stetige Abbildung zwischen topologischen Räumen und $U\subseteq X$ ein Unterraum.
\begin{enumerate}[i)]
	\item Es gilt $f(\overline{U})\subseteq\overline{f(U)}$.
	\item Sei $U$ dicht in $X$, $Y$ Hausdorffsch und $g\colon X\to Y$ eine weitere stetige Abbildung mit $g|_U=f|_U$, d.h. $g(u)=f(u)$ für alle $u\in U$. Dann gilt bereits $g=f$.
\end{enumerate}
\end{aufgabe}
\begin{aufgabe}[15 Punkte] Für $n\in\N$ sei \[C_n:=\bigcup_{i=0}^{\frac{3^n-1}{2}}[\tfrac{2i}{3^n},\tfrac{2i+1}{3^n}]\subseteq \R.\] Dann ist $C_n$ mit der Teilraumtopologie abgeschlossen. Sei $C:=\bigcap_{n\in\N}C_n$. Als Schnitt abgeschlossener Mengen ist $C\subseteq \R$ abgeschlossen. Da $C$ auch beschränkt ist, ist $C$ kompakt. $C$ heißt \emph{Cantor Menge}. Weiterhin lässt sich jedes $x\in[0,1]$ in triadischer Darstellung schreiben, d.h. als $x=\sum_{i=1}^\infty\tfrac{a_i}{3^i}$ mit $a_i\in\{0,1,2\}$.
\begin{enumerate}[i)]
	\item $C$ besteht gerade aus den Punkten von $[0,1]$ für die eine triadische Darstellung mit $a_i\in\{0,2\}$ existiert.
	\item Für alle Punkte in $C$ ist die triadische Darstellung mit $a_i\in\{0,2\}$ eindeutig.
	\item $C$ ist homöomorph zu $\prod_{i=1}^\infty\{0,2\}$.
	\item $C$ ist homöomorph zu $C\times C$.
\end{enumerate}
\end{aufgabe}
\begin{aufgabe}[Ein-Punkt-Kompaktifizierung, (15 Punkte)]
Wir zeigen in dieser Aufgabe, dass man jeden lokal-kompakten Hausdorff-Raum durch Hinzufügen eines Punktes in einen kompakten Raum einbetten kann.\\
Es sei $X$ ein topologischer Raum. $X$ heißt {\itshape lokal-kompakt}, wenn es für jeden Punkt $x\in X$ und jede Umgebung $U$ von $x$ eine kompakte Umgebung $K\subseteq U$ von $x$ gibt, die in $U$ enthalten ist.

Es sei nun $X$ ein lokal-kompakter Hausdorffraum. Dann definieren wir die {\itshape Ein-Punkt-Kompaktifizierung} von $X$ als $X^+= X\cup\{\infty\}$, und nennen eine Teilmenge $U\subset X^+$ offen genau dann, wenn entweder $U\subseteq X$ eine offene Teilmenge von $X$ ist oder wenn $\infty\in U$ ist und $X^+\setminus U\subseteq X$ kompakt ist mit der Unterraumtopologie von $X$.
\begin{enumerate}[i)]
	\item Dies definiert eine Topologie auf $X^+$, mit der die Inklusion $X\to X^+$ stetig und offen ist.
	\item $X^+$ ist kompakt und Hausdorffsch.
	\item Für die Menge $\mathbb N_{>0}$ der natürlichen Zahlen (ohne 0) mit der diskreten Topologie ist die Ein-Punkt-Kompaktifizierung $\mathbb N_{>0}^+$ homöomorph zu $\{\frac{1}{n}\mid n\in \mathbb N_{>0} \} \cup \{0\} \subset \mathbb R$ ist.
	\item Sei $s\colon Y\to X$ ein Schnitt einer beliebigen stetigen Abbildung $f\colon X\to Y$, dann ist $s(Y)$ abgeschlossen. 
	\item $X$ ist offen in $\beta(X)$.
\end{enumerate}
\end{aufgabe}
