\begin{aufgabe}
	Sei $p\colon X\to Y$ eine stetige Abbildung. Ein \emph{Schnitt} von $p$ ist eine stetige Abbildung $s\colon Y\to X$ mit $p\circ s=\id_Y$.
	\begin{enumerate}[i)]
		\item Besitzt $p$ einen Schnitt, so ist $p$ surjektiv ist und die Topologie auf $Y$ ist die Quotiententopologie bezüglich der Abbildung $p$.
		\item Gib zwei verschiedene Schnitte für die Abbildung $\R^2\to \R,(x,y)\mapsto x+y$ an.
	\end{enumerate}
\end{aufgabe}

\begin{aufgabe}\label{Aufgabe:Abbildungen_auf_Teilmengen}
	Es sei $X$ ein topologischer Raum und $\{X_i\}_{i\in I}$ eine Familie von Teilmengen von $X$ mit $X=\bigcup_{i\in I} X_i$.
	Sei
	\begin{enumerate}[i)]
		\item $X_i$ offen für alle $i\in I$ oder
		\item $X_i$ abgeschlossen für alle $i\in I$ und $I$ endlich.
	\end{enumerate}
	Sei $f\colon X\to Y$ eine Abbildung. Dann sind folgenden Aussagen äquivalent:
	\begin{enumerate}[a)]
		\item Die Abbildung $f$ ist stetig.
		\item Jede der eingeschränkten Abbildungen $f\vert_{X_i} \colon X_i \rightarrow Y$
		mit $i \in I$ ist stetig.
	\end{enumerate}
\end{aufgabe}

\begin{aufgabe}
	Jeder metrisierbare topologische Raum ist normal.
\end{aufgabe}

\begin{aufgabe}
	Es sei $X$ ein topologischer Raum und $A, B\subset X$ kompakte Unterräume (d.h. kompakt als topologische Räume mit der Unterraumtopologie).
	\begin{enumerate}[i)]
		\item Die Vereinigung $A\cup B\subset X$ ist ein kompakter Unterraum.
		\item Wenn $X$ Hausdorffsch ist, dann ist auch $A\cap B$ kompakt.
		\item Die Teilraumtopologie auf $\R$ als Teilraum der Gerade mit zwei Ursprüngen ist die euklidische Topologie. Also ist das Intervall $[-1,1]$ als Teilmenge der Geraden mit zwei Ursprüngen kompakt.
		\item Geben Sie ein Beispiel für $X$, $A$ und $B$ an, sodass $A$ und $B$ kompakt sind, aber $A\cap B$ nicht.
	\end{enumerate}
\end{aufgabe}
