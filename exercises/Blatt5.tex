\begin{aufgabe}[Zusammenhängend $\neq$ Wegzusammenhängend]
Es gilt laut Vorlesung, dass jeder wegzusammenhängende topologische Raum auch zusammenhängend ist. In dieser Aufgabe zeigen wir, dass die Umkehrung nicht gilt.
 
Der Raum $S$ sei definiert als folgender Unterraum
des $\mathbb{R}^2$:
\[S = \Big\{ \left(x,\sin \frac{1}{x}\right) \in \mathbb{R}^2 \mid x > 0\Big\}
\ \cup \ \{0\} \times [-1,1].\]

\hspace*{-1cm}
%\includegraphics{bilder/sinus}
(Mit anderen Worten: $S$ ist der Abschluss, in $\mathbb{R}^2$, des Graphen
der Funktion $x \mapsto \sin \frac{1}{x}$ für $x > 0$.)

Zeige: 
\begin{enumerate}[i)]
\item Der Raum $S$ ist zusammenhängend.
\item Der Raum $S$ ist nicht wegzusammenhängend.
\end{enumerate}
\begin{minipage}{\textwidth}
    \centering
            \begin{tikzpicture}[domain=0.001:1, xscale = 6]
                \draw[color=blue!30!white,smooth,samples=100,domain=0.001:0.01,line width = 0.1pt] plot[id=gnuplots/topologists-sine-curve-1] function{sin(1/x)};
                \draw[color=blue!30!white,smooth,samples=1000,domain=0.01:0.1, line width = 0.1pt] plot[id=gnuplots/topologists-sine-curve-2] function{sin(1/x)};
                \draw[color=blue!30!white,smooth,samples=100,domain=0.1:1, line width = 0.1pt] plot[id=gnuplots/topologists-sine-curve-3] function{sin(1/x)};
                \draw[color=red,thick] (0,-1) -- (0,1);
                \draw[->] (0,0) -- (1,0);
                \foreach \x in {1,2,3,4,5,6,7,8,9} {
                    \draw (0.1*\x,-0.1) node[anchor=north]{0,\x} -- (0.1*\x, 0.1);
                }
                \draw (-0.01,-1) node[anchor = east] {-1} -- (0.01,-1);
                \draw (-0.01,1) node[anchor = east] {1} -- (0.01,1);
            \end{tikzpicture}
\end{minipage}
\captionof{figure}{Sinuskurve des Topologen}
\end{aufgabe}

\begin{aufgabe}
Es seien $X$ und $Y$ zusammenhängende Räume. Zeige:
\begin{enumerate}[i)]
	\item Dann ist auch $X\times Y$ zusammenhängend.
	\item Wenn $X$ und $Y$ beide mehr als ein Element besitzen, dann ist auch $X\times Y\setminus \{(x,y)\}$ für ein festes $(x,y)\in X\times Y$ zusammenhängend.
	\item Es gibt keinen topologischen Raum $X$, sodass $\mathbb R$ homöomorph zu $X\times X$ ist.
\end{enumerate}
\end{aufgabe}

\begin{aufgabe}
Es sei $X$ ein topologischer Raum. Eine Folge $(x_i)_{i\geq 1}$, $x_i\in X$ {\itshape konvergiert gegen} $x\in X$, wenn für jede Umgebung $U$ von $x$ alle bis auf endlich viele $x_i$ in $U$ enthalten sind. Dann heißt $x$ \textit{Limes} oder \textit{Grenzwert} der Folge.
\begin{enumerate}[i)]
\item Sei $X$ Haussdorffsch. Dann ist $x$ eindeutig.
\item Gib ein Beispiel einer Folge in einem topologischen Raum mit zwei unterschiedlichen Grenzwerte. 
\item Sei $f\colon X\to Y$, and $(x_i)_{i\geq 1}$ eine konvergente Folge in $X$. Beweise oder widerlege: $(f(x_i))_{i\geq 1}$ konvergiert gegen $f(y)$.
\item Sei $(s_i)_{i\geq 1}$ eine Folge stetiger Funktionen $s_i\colon X\to Y$ und $Y$ ein metrischer Raum. $(s_i)_{i\geq 1}$ konvergiere \textit{gleichmäßig} gegen eine Funktion $s\colon X\to Y$, d.h.\ für jedes $\epsilon>0$ existiert eine $N\in \N$ so dass für alle $x\in X$, $n\geq \N$ gilt $d(s_i(x),s(x))< \epsilon$. Zeige, dass $s$ stetig ist. 
\end{enumerate}

\end{aufgabe}

\begin{aufgabe}
Es sei $X$ ein topologischer Raum. Wir sagen, dass $X$ {\itshape folgenkompakt} ist, wenn jede Folge $(x_i)_{i\in \mathbb N}$ in $X$ eine konvergente Teilfolge hat. 
\begin{enumerate}[i)]
	\item Zeige, dass ein kompakter metrischer Raum $X$ folgenkompakt ist. 
	\item Beweise das Lebesgue-Lemma:
	
	Es sei $X$ ein folgenkompakter metrischer Raum und $(U_i)_{i\in I}$ eine offene Überdeckung von $X$. Dann gibt es ein $\epsilon>0$ sodass für jede Teilmenge $A$ mit Durchmesser $D<\epsilon$ ein $i\in I$ existiert, sodass $A\subset U_i$ gilt.\\
	(Der Durchmesser eines  metrischen Raumes $A$ ist $\sup \{d(x,y) \mid x,y\in A\} $.)
	\item Zeige, dass ein folgenkompakter metrischer Raum $X$ kompakt ist. 
	%\item Sei $Y:=\prod_{[0,1]} \{0,1\}$. Zeige, dass $Y$ kompakt, aber nicht folgenkompakt ist.
	
	%\emph{Hinweis:} Betrachte die Folge $(a_i)_{i\geq 1}$ so dass $x=\sum_{i=1}^\infty \tfrac{a_i(x)}{2^i}$.
\end{enumerate}
\end{aufgabe}
