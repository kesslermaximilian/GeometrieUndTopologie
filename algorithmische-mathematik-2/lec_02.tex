\lecture{2}{Mi 14 Apr 2021 10:17}{Wahrscheinlichkeitsräume}
Wir kennen nun die Grundbegriffe $\Omega, \mathcal{F}, \mathbb{P}$ zur Beschreibung von Zufallsexperimenten, die wir uns nun genauer ansehen wollen:
\begin{question}
    Welche Struktur muss $\mathcal{F}$ besitzen.
\end{question}
Sein $A,B\in \mathcal{F}$, dann können wir das Ereignis $A \cap B$ betrachten, d.h. beide der Eigeschaften treten ein. Genauso sollte
 \[
A^{c} := \Omega \setminus A 
.\] 
, das \vocab{Komplement von $A$}, bzw. das \vocab{Gegenereignis} von $A$ ebenfalls in  $\mathcal{F}$  sein. Aus den beiden vorherigen Eigenschaften folgt bereits, dass
\[
    A \cup B= (A^{c} \cap B^{c})^{c}
.\] 
ebenfalls in $\mathcal{F}$ sein wird. \\
Eine Menge $\mathcal{F}$ mit solchen Eigenschaften heißt \vocab{Algebra}.
\begin{notation}
Seien nun $A,B$ und $(A_i)_{i\in I}$  Ereignisse, wobei $I$ endlich oder abzählbar sei. Dann notieren wir die folgenden Ereignisse:
\begin{enumerate}[label=\protect\circled{\alph*}]
    \item \emphasize{$A \cup B$} : $ω\in A \cup B \iff  ω\in A \lor ω\in B$, d.h. $A\cup B$ tritt ein, genau dann, wenn  $A$ eintritt oder  $B$ eintritt.
        \item  \emphasize{$\bigcup_{i \in  I} A_i$}: $ω\in \bigcup_{i \in  I} A_i$, wenn es ein $i\in I$ gibt, sodass $\omega \in A_i$
    \item  \emphasize{$A \cap B$}: $\omega\in A \cap B \iff  $ A \underline{und} B treten ein.
        \item \emphasize{$\bigcap_{i \in I} A_i$}: $\omega\in \bigcap_{i \in I}A_i \iff \forall i \in I \colon$ $A_i$ tritt ein.
            \item \emphasize{$A = \emptyset$} ist das Ereignis, das  \underline{nie} eintritt. \\
                \emphasize{$A = \Omega$} ist das Ereignis, dass \underline{immer} eintritt.
\end{enumerate}
\end{notation}

\begin{definition}[$\sigma$-Algebra]\label{def:sigma-algebra}
    Eine  \vocab{$\sigma$-Algebra} ist eine nicht leere Menge $\mathcal{F}$ von Teilmengen von $\Omega$ mit den Eigenschaften:
    \begin{enumerate}[label=\protect\circled{\alph*}]
        \item $\Omega \in \mathcal{F}$
        \item $\forall A\in \mathcal{F} \colon A^{c}\in \mathcal{F}$.
        \item Falls $(A_i)_{i \in I}\in \mathcal{F}$, dann auch $\bigcup_{i=1} ^{\infty}A_i \in \mathcal{F}$
    \end{enumerate}
    Wir nennen $(\Omega,\mathcal{F})$ dann einen \vocab{Messraum}. 
\end{definition}

\begin{lemma}\label{lm:weitere-eigenschaften-einer-sigma-algebra}
    Sei $\mathcal{F}$ eine $\sigma$-Algebra, dann ist:
    \begin{enumerate}[label=\protect\circled{\alph*}]
        \item $\emptyset\in \mathcal{F}$
        \item $A,B \in \mathcal{F} \implies A \cup B \in \mathcal{F}$ und $A\cap B \in \mathcal{F}$.
        \item $(A_i)_{i \in I}\in \mathcal{F} \implies \bigcap_{i=1}^{\infty}A_i \in \mathcal{F}$.
    \end{enumerate}
\end{lemma}
\begin{proof}
    \begin{enumerate}[label=\protect\circled{\alph*}]
        \item $\emptyset = \Omega^{c} \in \mathcal{F}$ nach Eigenschaften \circled{a} und \circled{b} aus der Definition.
        \item $A \cup B = A \cup B \cup \emptyset \cup \emptyset \ldots \in \mathcal{F}$ nach Eigenschaften  \circled{b} und \circled{c}. $A \cap B = (A^{c}\cup B ^{c})^{c} \in \mathcal{F}$
        \item $\bigcap_{i=1}^{\infty}A_i = \left( \bigcup_{i=1}^{\infty}A_i^{c} \right) ^{c}\in \mathcal{F}$ nach \circled{b} und \circled{c}.
    \end{enumerate}
\end{proof}

Wir haben nun $(\Omega, \mathcal{F})$ näher untersucht, es fehlt nun noch $\mathbb{P}$. 
\begin{question}
    Welche Eigenschaften soll $\mathbb{P}$ (das Wahrscheinlichkeitsmaß bzw. die Wahrscheinlichkeitsverteilung) besitzen?
\end{question}
Seien $A,B \in \mathcal{F}$ mit $A\cap B = \emptyset$, d.h. $A$ und $B$ können nicht gleichzeitig eintreten. Dann fordern wir
\[
    \mathbb{P}(A \cap B) = \mathbb{P}(A) + \mathbb{P}(B) \quad \text{(endliche Additivität)}
.\] 
Dazu wollen wir, dass $\Omega \in \mathcal{F}$ immer eintritt, d.h. $\mathbb{P}(\Omega) = 1 \equiv  100\%$ (Normierung).

\begin{definition}[Wahrscheinlichkeitsverteilung]\label{def:wahrscheinlichkeitsverteilung}
Sei $(\Omega, \mathcal{F})$ ein Messraum. Eine Abbildung $\mathbb{P} : \mathcal{F} \to  \R_+$ ist eine \vocab{Wahrscheinlichkeitsverteilung} auf $(\Omega, \mathcal{F})$, falls
    \begin{enumerate}[(1)]
        \item $\mathbb{P}(\Omega) = 1$
        \item Sind $(A_i)_{i \in I}\in \mathcal{F}$ paarweise disjunkt, so ist:
            \[
                \mathbb{P}\left( \bigcup_{i=1}^{\infty}A_i \right) = \sum_{i=1}^{\infty} \mathbb{P}(A_i) \quad (\sigma\text{-Additivität})
            .\] 
    \end{enumerate}
\end{definition}
\begin{remark*}
    Die Definition macht implizit Gebrauch davon, dass die linke Seite überhaupt definiert ist. Dies folgt jedochdaraus, dass $\mathcal{F}$ eine $\sigma$-Algebra ist.
\end{remark*}

\begin{definition}[Wahrscheinlichkeitsraum]\label{def:wahrscheinlichkeitsraum}
    Ein \vocab{Wahrscheinlichkeitsraum $(\Omega, \mathcal{F},\mathbb{P})$} besteht aus einer Menge $\Omega$, einer  $\sigma$-Algebra $F\subset \mathbb{P}\mathcal{(\Omega)}$ und einem Wahrschenilichkeitsmass $\mathbb{P}$ auf $(\Omega, \mathcal{F})$ 
\end{definition}


\begin{lemma}\label{lm:weitere-eigenschaften-eines-wahrscheinlichkeitsraums}
    Sei $(\Omega, \mathcal{F}, \mathbb{P})$ ein Wahrscheinlichkeitsraum. Dann ist
\begin{enumerate}[label=\protect\circled{\alph*}]
    \item $\mathbb{P}(\emptyset)=0$ 
    \item $\forall A,B\in \mathcal{F}$ mit $A\cap B = \emptyset$ ist
        \[
            \mathbb{P}(A\cup B ) = \mathbb{P}(A) + \mathbb{P}(B)
        .\] 
    \item      $\forall A,B\in \mathcal{F}$ mit $A\subset B$ ist 
        \begin{equation*}
            \begin{split}
                \mathbb{P}(B) &= \mathbb{P}(A) + \mathbb{P}(B \setminus A)  \\
                \mathbb{P}(A^{c}) &= 1 - \mathbb{P}(A) \\
                \mathbb{P}(A) &\leq  \mathbb{P}(B) \leq  1
            \end{split}
        \end{equation*}
    \item $\forall A,B \in \mathcal{F}$ ist
        \begin{equation*}
            \begin{split}
                \mathbb{P}(A \cup B) &= \mathbb{P}(A) + \mathbb{P}(B) - \mathbb{P}(A\cap B) \\
                                     &\leq  \mathbb{P}(A) + \mathbb{P}(B)
            \end{split}
        \end{equation*}
    \item Wenn $A_n \nearrow A$ ,d.h. $A_1\subset A_2\subset \ldots$ mit $\bigcup_{i =1}^{\infty} A_i = A$ (monotone Konvergenz von Mengen), oder $A_n \searrow A$ (d.h.  $A_1\supset A_2 \supset \ldots$ mit $\bigcap_{i=1}^{\infty} A_i = A $ ), so ist
        \[
            \lim_{n \to \infty} \mathbb{P}(A_n) = \mathbb{P}\left( \lim_{n \to \infty} A_n \right)  = \mathbb{P}(A)
        .\] 
\end{enumerate}
\end{lemma}
\begin{proof}
    \begin{enumerate}[label=\protect\circled{\alph*}]
        \item Wir wissen:
            \[
                1= \mathbb{P}(\Omega) = \mathbb{P}\left( \Omega \cup \emptyset \cup \emptyset \cup \emptyset \ldots  \right)  = \mathbb{P}(\Omega) + \mathbb{P}(\emptyset) + \mathbb{P}(\emptyset) + \ldots
            .\] 
            subtrahieren von $\mathbb{P}(\Omega) =1$ liefert dann $\mathbb{P}(\emptyset) = 0$.
        \item Sei $A \cap B = \emptyset$, dann ist:
            \begin{equation*}
                \begin{split}
                    \mathbb{P}(A\cup B ) &= \mathbb{P}(A \cup B \cup \emptyset \cup \emptyset \cup \ldots) \\
                                         &\stackrel{σ-\text{Additivität}}{=} \mathbb{P}(A) + \mathbb{P}(B) + \mathbb{P}(\emptyset) + \mathbb{P}(\emptyset) + \ldots \\
                                         &= \mathbb{P}(A) + \mathbb{P}(B)
                \end{split}
            \end{equation*}
        \item Sei $A\subset B$. Dann ist $B = A \cup (B \setminus A)$ eine disjunkte Vereinigung, also erhalten wir
            \[
                \mathbb{P}(B) = \mathbb{P}(A) + \underbrace{\mathbb{P}(B \setminus A)}_{\geq 0} \geq  \mathbb{P}(A)
            .\] 
            Mit $B = \Omega$ ergibt sich  $1 = \mathbb{P}(A) + \mathbb{P}(A^{c})$
        \item Es ist
            \begin{equation}
                \begin{split}
                    \mathbb{P}(A \cup B) &= \mathbb{P}(A) + \mathbb{P}((A \cup B) \setminus A)  \\
                                         &= \mathbb{P}(A) + \mathbb{P}(B \setminus (A \cap B)) \\
                                         &= \mathbb{P}(A) + \mathbb{P}(B) - \underbrace{\mathbb{P}(A \cap B)}_{\geq 0} \\
                                         &\geq \mathbb{P}(A) + \mathbb{P}(B)
                \end{split}
            \end{equation}
        \item Übung
    \end{enumerate}
\end{proof}
\begin{corollary}[Einschluss-Ausschluss-Prinzip]\label{cor:einschluss-ausschluss-prinzip}
    Seien $A_1,\ldots,A_n \in \mathcal{F}$. Dann gilt
    \[
        \mathbb{P}(A_1 \cup \ldots \cup A_n) = \sum_{k=1}^{n} (-1)^{k-1} \sum_{1\leq i_1<i_2<\ldots<i_k \leq n} \mathbb{P}(A_{i_1} \cap A_{i_2} \cap \ldots \cap A_{i_k})
    .\] 
\end{corollary}
\begin{proof}
    Per Induktion, der Induktionsanfang lautet  $\mathbb{P}(A_1) = \mathbb{P}(A_1)$ und ist offensichtlich wahr. \\
    Die Aussage gelte nun für ein $n\in \N$, dann erhalten wir
    \begin{equation}
        \begin{split}
            \mathbb{P}\left( \bigcup_{i=1}^{n+1} A_i \right)  &= \mathbb{P}\left( \left(\bigcup_{i=1}^{n}A_i \right) \cup A_{n+1}\right)  \\
                                                              &= \mathbb{P}\left(\bigcup_{i=1}^{n} A_i\right) + \mathbb{P}(A_{n+1}) - \mathbb{P}\left( \left( \bigcup_{i=1}^n A_i \right) \cap A_{n+1} \right)  \\
                                                              &= \mathbb{P}\left( \bigcup_{i=1}^{n} A_i \right)  + \mathbb{P}(A_{n+1}) - \mathbb{P}\left( \bigcup_{i=1}^n \underbrace{(A_i \cap _{A_{n+1}})}_{=: \tilde{A}_i} \right)  \\
                                                              &= \sum_{k=1}^{n} (-1)^{k-1} \sum_{1\leq i<\ldots<i_k \leq n} \mathbb{P}(A_{i_1} \cap \ldots \cap A_{i_k}) + \mathbb{P}(A_{n+1}) \\
                                                              &\qquad -\sum_{k=1}^n (-1)^{k-1} \sum_{1\leq i_1 < \ldots < i_k \leq n} \mathbb{P}(\underbrace{\tilde{A}_{i_1} \cap \ldots \cap \tilde{A}_{i_k}}_{A_{i_1} \cap \ldots \cap A_{i_k} \cap A_{n+1}})
        \end{split}
    \end{equation}
    Andererseits ist aber auch:
    \begin{alignat*}{5}
           &\quad&  &\sum_{k=1} ^{n+1} (-1)^{k-1} \sum_{1\leq i_1<\ldots<i_k \leq  n+1} \mathbb{P}(A_{i_1} \cap \ldots \cap A_{i_k}) &\quad & \\
           &= & &\sum\limits_{k=1}^{n}(-1)^{k-1} \sum\limits_{1\leq i_1< \ldots < i_k \leq  \color{red} n} \mathbb{P}(A_{i_1} \cap \ldots \cap A_{i_k}) & \quad &\Big\}\text{\parbox{2cm}{Terme mit $i_k\leq n$}} \\
           &+& &\underbrace{\sum_{k=2}^{n+2} (-1)^{k-1} \sum_{1\leq i_1<\ldots<i_{k-1}\leq n} \mathbb{P}(A_i \cap \ldots \cap A_{i_{k-1}} \cap A_{n+1})}_{\stackrel{l := k-1}{=} -\sum\limits_{l=1}^n (-1)^{l-1}\sum\limits_{1\leq i_1<...<i_l\leq n} \mathbb{P}(A_{i_1} \cap \ldots \cap A_{i_l} \cap A_{n+1})}& \quad &\Bigg\}\text{\parbox{2cm}{\small Terme mit $i_k = n+1$ und  $k\geq 2$}}\\
           &+& & \mathbb{P}(A_{n+1}) \qquad& \quad  &\Big\}\text{\parbox{2cm}{\small Terme mit $i_k = n+1$ und  $k=1$}}
    \end{alignat*}
    und damit sehen wir, dass die beiden Ausdrücke übereinstimmen, also ist der Induktionsschritt erbracht.
\end{proof}


\subsection{Diskrete Verteilungen}
\begin{itemize}
    \item Sei nun $\Omega$ endlich oder abzählbar.
    \item Falls wir $\mathcal{F}$ nicht explizit angeben, dann wird $\mathcal{F} = \mathcal{P}(\Omega)$ gewählt, d.h.
        \[
            \operatorname{Card} (\mathcal{P}(\Omega)) \equiv \abs{\mathcal{P}(\Omega)} = 2 ^{ \abs{\Omega}} 
        .\] 
\end{itemize}
\begin{example}[Münzwurf]
    Es sei $\Omega = \left \{K,Z\right\}$, wobei $K$ für Kopf stehe und $Z$ für Zahl. Dann ist
    \[
    \mathcal{F} = \left \{\left \{K\right\} ,\left \{Z\right\} ,\left \{Z,K\right\} ,\emptyset\right\} 
    .\] 
    Sei $p\in [0,1]$ die Wahrscheinlichkeit, dass man Kopf erhält. Da $\mathbb{P}$ für alle Element aus $\mathcal{F}$ definiert sein muss, erhalten wir
    \[
        \mathbb{P}(\emptyset) = 0 \qquad \mathbb{P}(K) = p, \qquad \mathbb{P}(Z) = \mathbb{P}(K^{c}) = 1-p \qquad \mathbb{P}(\left \{Z,K\right\} ) = \mathbb{P}(\Omega) = 1
    .\] 
\end{example}
\begin{question}[Charakterisierung von diskreter Wahrscheinlichkeit]
Was müssen wir fordern, sodass $\mathbb{P}$ auf $\mathcal{P}(\Omega)$ gibt?.
\end{question}
\begin{example}
    $\Omega = \left \{1,2,\ldots,10\right\}$ würde genügen, da dann $\abs{\mathcal{P}(\Omega)}= 2^{\abs{\Omega} }=2^{10} = 1024 $
    endlich (diskret) ist.
\end{example}
