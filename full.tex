\documentclass[fancyfoot, a4paper, german, git]{mkessler-script}

\course{Einführung in die Geometrie und Topologie}
\lecturer{Daniel Kasprowski}
\assistant[f]{Arunima Ray}
\author{Maximilian Keßler}

\usepackage{mkessler-hypersetup}
\usepackage{mkessler-math}
\usepackage[lecturenumbers = true, number small environment = theorem]{mkessler-fancythm}
\usepackage[index]{mkessler-vocab}
\usepackage[bibfile=references/bibliography.bib, imagefile=references/images.bib]{mkessler-bibliography}
\usepackage[fancyhead]{mkessler-lectures}
\usepackage{castel-incfig}
\usepackage{mkessler-counters}
\usepackage{mkessler-proof}

\import{inputs/exercises/}{preambleBlatt.tex}

\usepackage{todonotes}
\setuptodonotes{disable}


%%%% From here on, the preamble is a single ugly hack to make this document work for now
\usepackage{caption}
\usepackage{subcaption}
\usepackage[shortlabels]{enumitem}

\newcommand{\emphasize}[1]{{\color{red} #1}}
\newenvironment{proof*}{\begin{proof}}{\end{proof}}
\makeatletter
\newcommand\setItemnumber[1]{\setcounter{enum\romannumeral\@enumdepth}{\numexpr#1-1\relax}}
\makeatother
\newcounter{blatt}
\declaretheorem[style = thmgreenmargin, numberwithin = blatt, name =Aufgabe]{aufgabe}
\newcommand\blatt{\refstepcounter{blatt}\subsection*{\theblatt. Übungsblatt}\addcontentsline{toc}{subsection}{\theblatt. Übungsblatt}}
\AtEndEnvironment{aufgabe}{\label{aufgabe-\theaufgabe}}
\AtBeginDocument{\def\itemautorefname{Aufgabe}}

\usepackage{float}
\def\cat#1{\mathcat{#1}}
\def\category#1{\operatorname{\textbf{#1}}}


\usepackage{import}


\begin{document}
    \maketitle

    \import{inputs/}{abstract.tex}

    %Table of contents
    \cleardoublepage
    \tableofcontents

    %List of lectures with their corresponding keywords
    \cleardoublepage
    \summaryoflectures

    \cleardoublepage
    % start lectures
    \import{inputs/lectures/}{lec_01.tex}
    \import{inputs/lectures/}{lec_02.tex}
    \import{inputs/lectures/}{lec_03.tex}
    \import{inputs/lectures/}{lec_04.tex}
    \import{inputs/lectures/}{lec_05.tex}
    \import{inputs/lectures/}{lec_06.tex}
    \import{inputs/lectures/}{lec_07.tex}
    \import{inputs/lectures/}{lec_08.tex}
    \import{inputs/lectures/}{lec_09.tex}
    \import{inputs/lectures/}{lec_10.tex}
    \import{inputs/lectures/}{lec_11.tex}
    \import{inputs/lectures/}{lec_12.tex}
    \import{inputs/lectures/}{lec_13.tex}
    \import{inputs/lectures/}{lec_14.tex}
    \import{inputs/lectures/}{lec_15.tex}
    \import{inputs/lectures/}{lec_16.tex}
    \import{inputs/lectures/}{lec_17.tex}
    \import{inputs/lectures/}{lec_18.tex}
    \import{inputs/lectures/}{lec_19.tex}
    \import{inputs/lectures/}{lec_20.tex}
    \import{inputs/lectures/}{lec_21.tex}
    \import{inputs/lectures/}{lec_22.tex}
    \import{inputs/lectures/}{lec_23.tex}
    % end lectures

    %Start appendix
    \cleardoublepage
    \appendix
    \part{Anhang}

    %Exercises of the lecture
    \import{inputs/exercises/}{exercises.tex}
    
    %Explanation of environments
    \cleardoublepage
%    \import{inputs/}{environments.tex}

    %Index
    \cleardoublepage
    \printvocabindex

    %Image attributions
    \cleardoublepage
    \printimageattributions

    %Literature
    \cleardoublepage
    \printliterature

\end{document}


