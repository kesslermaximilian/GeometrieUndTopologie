%! TEX root = ./master.tex
\lecture[Kontravariante Funktoren. Dualkategorie. Dualraum als Funktor. Natürliche Transformationen und natürliche Äquivalenz von Kategorien. Doppelter Dualfunktor als natürliche Transformation. Abelisierung von Gruppen. Diagramme. Initial- und Terminalobjekte. Diagramme. Kegel, Kegelkategorie. Limiten und Kolimiten: Produkt, Koprodukt, Faserprodukt, Pushout.]{Di 01 Jun 2021 12:15}{Funktoren, Limiten}
\begin{definition}[kontravarianter Funktor]\label{def:kontravarianter-funktor}
   Ein \vocab{kontravarianter} Funktor $\mathcal{F} : \cat{C} \to  \cat{D}$ besteht aus
   \begin{itemize}
       \item einer Abbildung $\mathcal{F}: \Ob(\mathcal{C}) \to  \Ob(\cat{D})$
       \item Abbildungen $\mathcal{F}\colon  \Mor_{\mathcal{C}}(X,Y) \to  \Mor_{\cat{D}}(\mathcal{F}(Y), \mathcal{F}(X))$ für $X,Y \in  \Ob(\cat{C})$
   \end{itemize}
   sodass
   \[
       \mathcal{F}(\id_X) = \id_{\mathcal{F}(X)}, \qquad \mathcal{F}(f \circ  g) = \mathcal{F} (g) \circ  \mathcal{F}(f)
   .\] 
\end{definition}

\begin{example}[Dualraum]
    Das Bilden des Dualraums eines Vektorraums bildet einen Funktor
        \begin{equation*}
            ()^* : \left| \begin{array}{c c l} 
        \Vect_{\R} & \longrightarrow & \Vect_{\R} \\
        V & \longmapsto &  V^*=\Hom_{\R}(V,\R) \\
        f\colon  V \to  W & \longmapsto & 
            f^*: \left| \begin{array}{c c l} 
            W^*  & \longrightarrow & V^* \\
            W\stackrel{g}{\to} \R & \longmapsto	 & V \stackrel{f}{\to }W \stackrel{g}{\to } \R \\
            g & \longmapsto & g \circ f
        \end{array} \right.
        
        \end{array} \right.
    \end{equation*}
\end{example}

\begin{remark*}
   Natürlich ist hier an $\R$ nichts besonders, das geht mit jedem Körper $K$. 
\end{remark*}

\begin{definition}[Dualkategorie]\label{def:dualkategorie}
    Ist $\cat{C}$ eine Kategorie, so ist $\mathcal{C}^{\op}$ die Kategorie mit $\Ob(\mathcal{C}^{\op}) = \Ob(\cat{C})$, sowie $\Mor_{\cat{C}^{\op}}(X,Y) \coloneqq \Mor_{\cat{C}}(Y,X)$, und 'denselben' Verknüpfungen.
\end{definition}

\begin{remark*}
    'Dieselben' Verknüpfungen, d.h für $X,Y,Z \in  \Ob(\cat{C}^{\op}) = \Ob(\cat{C})$ definieren wir:
        \begin{equation*}
        \circ _{\cat{C}^{\op}}: \left| \begin{array}{c c l} 
            \Mor_{\cat{C}^{\op}}(X,Y) \times  \Mor_{\cat{C}^{\op}}(Y,Z) & \longrightarrow & \Mor_{\cat{C}^{\op}}(X,Z) \\
            (f,g) & \longmapsto &  f \circ _{\cat{C}} g
        \end{array} \right.
    \end{equation*}
    indem wir uns erinnern, dass $f \in \Mor_{\cat{C}}(Y,X)$ und $g\in \Mor_{\cat{C}}(Z,Y)$
\end{remark*}

\begin{lemma}\label{lm:kovariante-funktoren-sind-funktoren-von-der-dualen-kategorie}
    Ein kontravarianter Funktor $\mathcal{F}\colon \cat{C} \to  \cat{D}$ ist ein (kovarianter) Funktor $\mathcal{F}\colon  \cat{C}^{\op} \to  \cat{D}$.
\end{lemma}
\begin{proof*}
    Nur Definitionen ausschreiben, wirklich nichts spannendes, sollte man aber einmal machen als Übung.
\end{proof*}

\begin{example}
$\Vect_{\R}^{\op} \stackrel{(\cdot )^*}{\longrightarrow} \Vect_{\R}$ als kovarianter Funktor, analog zu vorherigem Beispiel.
\end{example}

\begin{example}
    $\Cat$ ist die Kategorie der  \underline{kleinen} Kategorie und Funktoren. 
\end{example}
\todo{Das Beispiel sollte eigentlich früher kommen}

\begin{oral}
    Wir betrachten hier nur die kleinen Kategorien (also die, deren Objekte eine Menge bilden), damit wir nicht im Probleme wieder 'Klasse aller Klassen' laufen, solche Probleme könnte man aber auch wieder umgehen, wenn man weitere unerreichbare Kardinalzahlen fordert.
\end{oral}

\begin{definition}[natürliche Transformation und Äquivalenz]\label{def:natürliche-transformation-und-äquivalenz}
    Seien $\mathcal{F},\mathcal{G}\colon  \cat{C} \to  \cat{D}$ zwei Funktoren. 
    \begin{enumerate}[i)]
        \item 
    Eine \vocab{natürliche Transformation} $t\colon  \mathcal{F} \to  \mathcal{G}$ ist eine Familie von Morphismen (in $\cat{D}$)
    \[
        \left \{t_X \in  \Mor_{\cat{D}}(\mathcal{F}(X),\mathcal{G}(X))\right\} _{X\in \Ob(\cat{C})}
    .\] 
    sodass das Diagramm
    \begin{equation*}
        \begin{tikzcd}
            \mathcal{F}(X) \ar{r}{\mathcal{F}(f)} \ar[swap]{d}{t_X} & \mathcal{F}(Y) \ar{d}{t_{Y}} \\
            \mathcal{G}(X) \ar[swap]{r}{\mathcal{G}(f)} & \mathcal{G}(Y)
        \end{tikzcd}
    \end{equation*}
    kommutiert.
\item $t$ ist eine  \vocab{natürliche Äquivalenz} (auch natürlicher Isomorphismus genannt), falls jedes $t_X$ ein Isomorphismus ist. 
\item $\mathcal{F}\colon  \cat{C} \to  \cat{D}$ ist eine \vocab{Äquivalenz} (von Kategorien), falls es einen Funktor $\mathcal{G} \colon  \mathcal{D} \to  \cat{C}$ gibt, sodass $\mathcal{F} \circ  \mathcal{G}$ und $\mathcal{G} \circ  \mathcal{F}$ jeweils natürlich äquivalent zum Identitätsfunktor sind.
    \end{enumerate}
\end{definition}

\begin{remark*}
    Die Idee an obigem kommutativen Diagramm ist einfach, dass wir nun per se zwei Wege haben, von $\mathcal{F}(X)$ nach $\mathcal{G}(Y)$ zu gelangen, und wir wollen, dass diese gleich sind.
\end{remark*}

\begin{oral}
    Die Idee daran, natürliche Äquivalenz von Kategorien zuzulassen ist, dass wir uns nicht mehr um die Anzahl von (in einer Kategorie) zueinander isomorphen Objekten kümmern müssen, und davon ebenfalls wegabstrahieren.
\end{oral}

\begin{example}
    Es gibt eine natürliche Transformation $t\colon  \id_{\Vect_{\R}} \to  {()^*}^*$, gegeben durch Komponenten für jeden $\R$-Vektorraum $V$:
        \begin{equation*}
        t_V: \left| \begin{array}{c c l} 
            V & \longrightarrow & {V^*}^* = \Hom_{\R}(V^*,\R) \\
        v & \longmapsto &
            \ev_v: \left| \begin{array}{c c l} 
            V^* & \longrightarrow & \R \\
            f & \longmapsto	 &f(v)
        \end{array} \right.
        \end{array} \right.
    \end{equation*}
    Wir müssen prüfen, dass folgendes Diagramm für jede lineare Abbildung $f\colon  V \to  W$ kommutiert.
    \[
    \begin{tikzcd}
        V \ar[swap]{d}{t_V} \ar{r}{f} & W \ar{d}{t_W}\\
        {V^*}^* \ar[swap]{r}{{f^*}^*} & {W^*}^*
    \end{tikzcd}
    \]
    Sei hierzu $v$ beliebig, dann erhalten wir
     \[
    \begin{tikzcd}
        v \ar[swap]{d}{} \ar{r}{} & f(v) \ar{d}{}\\
        \ev_v \ar[swap]{r}{} & {\ev_v^*}^* = \ev_{f(v)}
    \end{tikzcd}
    \]
    Um zu sehen, dass die beiden Abbildungen, die wir rechts unten erhalten, gleich sind, sollte man sich kurz klarmachen, dass es sich dabei um Elemente aus ${W^*}^* = \Hom{W^*, \R}$ handelt, also können wir zeigen, dass die Abbildungen gleich sind, indem wir ein Element $g\in W^* = \Hom(W,\R)$ einsetzen, und die gleiche reelle Zahl erhalten. Dazu ist erstmal:
    \[
        \ev_{f(v)} (g) \stackrel{\text{Def. von $\ev$}}{=} g(f(v)) = (g \circ  f) (v)
    .\] 
sofort nach Definition. Die andere Rechnung ist etwas langwieriger, aber nur direktes Definitionen-Einsetzen:
\begin{IEEEeqnarray*}{rCl}
    {\ev_{v}^*}^* (g) & = & ((f^*)^*(\ev_v)) (g) \\
                      & \stackrel{a^*(b) = b \circ  a}{=} & ( \ev_v \circ  f^*) (g) \\
                      & = & \ev_v (f^*(g)) \\
                      & \stackrel{a^o(b) = b \circ  a}{=} & \ev_v(g \circ  f) \\
                      & \stackrel{\text{Def. von $\ev$}}{=} &(g \circ  f) (v)
\end{IEEEeqnarray*}
und damit erhalten wir die gleiche reelle Zahl, also schlussendlich $(\ev_v^*)^* = \ev_{f(v)}$ wie gewünscht, das Diagramm kommutiert und in der Tat handelt es sich bei $t$ um eine natürliche Transformation  $\id_{\Vect_{\R}} \to  {()^*}^*$
\end{example}

    \begin{example}
        Betrachte die Kategorie der endlich-dimensionalen Untervektorräume von $\R^{\infty}$ mit ihren linearen Abbildung und diese mittels eines Inklusionsfunktors in $\Vect_{\R}^{f.d.}$ (endlich-dimensionale $\R$-Vektorräume)
    \end{example}

    \begin{oral}[In der Pause]
        Es ist $\Vect_{\R}^{f.d.}$ keine kleine Kategorie, denn wir können z.B. für jede beliebige Menge $X$ die Menge  $\R^n \times \left \{X\right\} $ bilden und diese kanonisch wideer als $\R^n$ auffassen, damit erhalten wir also für jede Menge (mindestens) ein Objekt in $\Vect_{\R}^{f.d.}$, und damit können wir auch die Menge aller Mengen basteln. \\
        Umgekehrt ist aber die Kategorie der endlich dimensionalen Untervektorräume von $\R^{\infty}$ eine kleine Kategorie, weil jedes Objekt ja insbesondere eine Teilmenge von $\R^{\infty}$ (zumindest die Trägermenge) ist, und wir damit wieder eine Menge erhalten.
    \end{oral}

    \begin{oral}
        Man kann zeigen, dass ein Funktor $\mathcal{F}$ genau dann eine natürliche Äquivalenz ist, wenn er essentiell surjektiv  und eine Bijektion auf den Morphismen ist. \\
        Essentiell surjektiv bedeutet, dass für jedes Objekt $Y\in \cat{D}$ ein zu $Y$ isomorphes Objekt  $Z$ existiert, das im Bild von  $\mathcal{F}$ liegt. \\
        Bijektion auf Morphismen heißt, dass $\mathcal{F}$ einen Isomorphismus $\Mor_{\cat{C}}(X,Y) \cong \Mor_{\cat{D}}(\mathcal{F}(X), \mathcal{F}(Y))$ mittels $f \mapsto \mathcal{F}(f)$ induziert.
    \end{oral}

\begin{oral}
    Das tolle an obigem Beispiel ist, dass wir nun eine Äquivalenz zwischen $\Vect_{\R}^{f.d.}$ und einer kleinen Kategorie gefunden habe. Das ist toll, wenn wir z.B. mit $\Cat$ arbeiten wollen, weil wir dann hier zwar  $\Vect_{\R}^{f.d.}$ nicht als Objekt wiederfinden, allerdings eine dazu natürlich äquivalente Kategorie.
\end{oral}

\begin{example}
    Betrachte die natürliche Transformation $t\colon \id_{\Grp} \to  ()^{ab}$, gegeben durch Komponenten für jede Gruppe $G$:
        \begin{equation*}
        t_G: \left| \begin{array}{c c l} 
            G & \twoheadrightarrow & G / [G,G] \\
            g & \longmapsto &  [g]
        \end{array} \right.
    \end{equation*}
\end{example}

\begin{recap}
    Ist $G$ eine Gruppe und  $a,b\in G$, so ist $[a,b] \coloneqq  aba^{-1}b^{-1}$ der \vocab{Kommutator} der beiden Elemente (der Name kommt daher, dass $[a,b] = e$ genau dann, wenn  $a,b$ kommutieren), und  $[G,G]$ ist die Untergruppe von  $G$, die von den Kommutatoren in $G$ erzeugt wird, d.h. alle endlichen Verknüpfungen von solchen Kommutatoren. Man kann nun zeigen, dass $[G,G]\unlhd G$ eine normale Untergruppe ist, und  $G / [G,G]$ ist abelsch und die  \vocab{Abelisierung} von $G$. Wir erhalten ebenfalls die kanonische Projektion $G \twoheadrightarrow G / [G,G]$.
\end{recap}

\begin{recap}
Wir können Gruppen auch als $\left< E \mid  R \right> $ darstellen, wobei $E$ eine Menge an Symbolen und  $R$ eine Menge an Relationen ist. Die beschrieben Gruppe besteht dann aus allen Wörtern, die die Buchstaben der Menge  $E$ und deren Inversen bilden, modulo der erzeugten Untergruppe der Relationen. \\
\begin{itemize}
    \item $\left< a,b \mid \emptyset \right>  = \left< a,b \right>  \cong F_2$ ist die freie Gruppe mit 2 Elementen. Sie besteht aus Wörtern wie $aba^{-1}b^{-2}a^3$, bei der keine Relationen gelten (außer $a a^{-1} = a^{-1}a = e, b b^{-1} = b^{-1} b = e$, was immer gefordert  wird).
    \item $\left< a,b \mid  [a,b] \right> \cong \Z^2$, denn jetzt kommutieren beliebige zwei Elemente, und wir können jedes Wort aus  $a,b,a^{-1},b^{-1}$ als $a^nb^m$ mit  $n,m \in \Z$ umschreiben, und der Isomorphismus zu $\Z^2$ ist kanonisch.
    \item $\left< a,b \mid  [a,b], a^3, b^3 \right> \cong \left( \Z / 3\Z \right) ^2$, denn jetzt ist ja zusätzlich noch $a^3 = b^3 = e$ das neutrale Element.
\end{itemize}
\end{recap}

\begin{oral}
\Warning    Es sei noch darauf hingewiesen, dass nicht alle Relationen, die gelten, aufgeführt sein müssen. Im dritten Beispiel ist ja z.B. auch $a^6 = e$, weil  $a^6 = a^3 a^3 = e e = e$.
\end{oral}
\begin{remark*}
Es sei gesagt, dass wir jede Gruppe in obiger Form darstellen können, indem wir $E$ als die Trägermenge von  $G$ wählen, und nun einfach für jede Gleichung  $g_1g_2 = g_3$, die in $G$ gilt, die entsprechende Relation  $g_1g_2g_3^{-1}$ in $R$ fordern. (Das wird aber schnell hässlich, don't do it!)
\end{remark*}

\begin{remark*}
\Warning    Auch, wenn diese Darstellung von Gruppen sehr schön sein \textit{kann}, ist sie sicherlich nicht immer die Beste. Es ist z.B. ein NP-vollständiges Problem, für eine Darstellung $\left< E \mid  R \right> $ zu entscheiden, ob die dadurch definierte Gruppe trivial ist, d.h. nur aus einem Element besteht.
\end{remark*}
\begin{recap}
    Als nächstes kann man sich überlegen, dass wir morphismen von $G = \left< E \mid  R \right> $ in eine Gruppe $H$ genau angeben können durch  $f(g)$ für jedes Element  $g\in E$, sodass $f(r) = e$ für jede Relation  $r\in R$. (Dabei meinen wir mit $f(r)$ für  $r = g_1g_2,\ldots,g_n$ einfach $f(g_1)\circ f(g_2)\circ \ldots\circ f(g_n)$). 

    Damit kann man dann auch die Abbildung $G \to  G / [G,G]$ leicht angeben, denn haben wir $G$ dargestellt als  $\left< E \mid  R \right> $, so ist $G / [G,G]$ gegeben durch  $\left< E \mid R, \left \{[a,b]\right\} _{a,b\in G^2} \right> $, und den Morphismus geben wir durch $f(g) = g$ an.
\end{recap}

\subsection{Limiten und Kolimiten}
Um Limiten und Kolimiten zu betrachten, brauchen wir sogenannte \vocab{Diagramme}:

Sei $\cat{C}$ eine Kategorie und $\cat{I}$ eine kleine Kategorie. Wir stellen uns dann $\cat{I}$ als Diagramm vor, z.B. eines der folgenden drei Beispiele:

\begin{equation}
\begin{tikzcd}
    \star \ar{r} \ar{d} & \star \\
    \star
\end{tikzcd} 
\qquad
\begin{tikzcd}
    \ldots \ar{r} & \star \ar{r} & \star \ar{r} &\ldots
\end{tikzcd}
\qquad
\begin{tikzcd}
    \star \ar[bend left = 10]{r} \ar[bend right = 10]{r} & \star
\end{tikzcd}
\label{eq:diagramme}
\end{equation}

\begin{definition}[Terminales Objekt]
    Ein \vocab{terminales} Objekt in $\cat{C}$ ist ein Objekt $X\in \Ob(\cat{C})$, sodass es für alle $Y\in \Ob(\cat{C})$ enien eindeutigen Morphismus $Y \to  X$ gibt, d.h.
     \[
         \forall Y\in \Ob(\cat{C})\colon  \qquad \Mor_{\cat{C}}(Y,X) = \left \{\star\right\} 
    .\] 
\end{definition}
\begin{oral}
    Es gibt zwar nicht die Definition \textit{des} terminalen Objekts, allerdings überlegt man sich leicht, dass terminale Objekte bis auf eindeutigen Isomorphismus eindeutig sind, deswegen ist eine Unterscheidung nicht notwendig bzw. erfolgt normalerweise nicht.
\end{oral}

\begin{example}
    In $\Top$ und  $\Set$ sind die Einpunktmengen die terminalen Objekte.
\end{example}

\begin{remark*}
    In der Vorlesung wurde der Begriff 'Kegelkategorie' auf einmal definiert, ich splitte das ganze jedoch auf, weil auch Kegel einzeln danach noch Verwendung finden, und ich das so lieber mag, der Inhalt ist aber derselbe:
\end{remark*}

\begin{dabuse}
    Sei $\cat{I}$ eine kleine Kategorie, $\cat{C}$ eine beliebige Kategorie und $X\colon  \cat{I} \to  \cat{C}$ ein Funktor. Für $i\in \Ob(\cat{I})$ schreiben wir auch kurz $X_i \coloneqq  X(i)$, um die entsprechenden Bildobjekte des Funktors in $\cat{C}$ zu bezeichnen.
\end{dabuse}

\begin{remark*}
    So sollte man auch wirklich über diese Diagramme nachdenken, $\cat{I}$ ist einfach eine elegante Möglichkeit, eine Menge von Objekten in $\cat{C}$ mittels $\cat{I}$ zu indizieren (= Indizes zu verteilen), und gleichzeitig einige Morphismen zwischen diesen Objekten mit zu beschreiben.
\end{remark*}

\begin{definition**}[Kegel]\label{def:kegel}
    Sei $\cat{I}$ eine kleine Kategorie und $X: \cat{I} \to  \cat{C}$ ein Funktor. Ein \vocab{Kegel} über $X$ ist ein Objekt $N\in \Ob(\cat{C})$ mit einer Familie von Morphismen $\left \{f_i\right\} _{i\in \cat{I}}$, wobei $f_i \in  \Mor_{\cat{C}}(N, X_i)$, sodass für jeden Morphismus $h\in \Mor_{\cat{I}}(i,j)$ das folgende Diagramm kommutiert:
    \[
    \begin{tikzcd}
        & N \ar[swap]{dl}{f_i} \ar{dr}{f_j} \\
        X_i \ar[swap]{rr}{X(h)} & & X_j
    \end{tikzcd}
    .\] 
\end{definition**}

\begin{dremark}
    Der name Kegel kommt daher, dass $N$ gewissermaßen mit den Morphismen  $f_i$ einen Kegel über den  $X_i$ erzeugt. Ist z.B.
     \[
         \cat{I} = 
    \begin{tikzcd}
        \star \ar{rr} \ar{dr} & & \star \ar{dl} \\
                              & \star
    \end{tikzcd}
\]
so ergibt sich als Ausschnitt der Kategorie $\cat{C}$ ein \textit{kommutatives} Diagramm (kommutativ wegen der Kommutativitätsbedingung in der Definition):
\[
\begin{tikzcd}
    & Y\ar[bend right = 10, dashed,swap]{ddl}{f_1} \ar[dashed, "f_3" description, near start]{ddd} \ar[bend left = 10, dashed]{ddr}{f_2} \\
    \\
    X_1 \ar{dr}\ar{rr} & & X_2 \ar{dl}\\
                & X_3
\end{tikzcd}
.\]

\end{dremark}

\begin{definition}[Kegelkategorie]\label{def:kegelkategorie}
    Sei $X: \cat{I} \to  \cat{C}$ ein Funktor. Die \vocab{Kegelkategorie} $\cat{C} / X$ hat als Objekte genau die Kegel über $X$. Die Morphismen zwischen Kegeln $(Y, \left \{f_i\right\}) \to (Y', f_i')$ sind gegebene durch $g\in \Mor_{\cat{C}}(Y,Y')$ (also durch übliche Morphismen zwischen $Y,Y'$ als Objekten von  $\cat{C}$), sodass für alle $i\in \Ob(\cat{I})$ folgendes kommutiert:
    \[
    \begin{tikzcd}
        Y \ar{rr}{g} \ar[swap]{dr}{f_i} & & Y' \ar{dl}{f_i'} \\
                                        & X_i
    \end{tikzcd}
    .\] 
\end{definition}

\begin{oral}
    Ist $\cat{I}$ einelementig, so erhalten wir die Kategorie über  $X\in \Ob(\cat{C})$, vergleiche hierzu \autoref{aufgabe-6.4}.
\end{oral}

\begin{remark*}
    Morphismen zwischen Kegeln lassen im Wesentlichen einfach 'alles' kommutieren, d.h. zeichnen wir die beiden Kegel und den entsprechenden Morphismus zwischen ihnen ein, so erhalten wir ein kommutatives Diagramm.
\end{remark*}

\begin{definition}[Limes]\label{def:limes}
    Ein Limes für $X \colon  \cat{I} \to  \cat{C}$ ist ein terminales Objekt in $\cat{C} / X$.
    \[
    \begin{tikzcd}
        & {\color{blue}Y}\ar[bend right = 10, blue, swap]{ddl}{h_1} \ar[blue, bend left = 10]{ddr}{h_2} \ar[violet, dashed, "\exists !" description, near start]{d}  \\
        & {\color{red}\lim X} \ar[red,swap]{dl}{s_1} \ar[red]{dr}{s_2} \ar[dashed, "s_0" description, near start, red]{dd}\\
        X_1 \ar[swap]{dr}{f_1} & & X_2\ar{dl}{f_2} \\
         & X_0
    \end{tikzcd}
\]
Zur Erklärung des Diagramms: In rot und blau die beiden Kegel $\lim X$ und  $Y$ über dem Diagramm  $X$ (in schwarz). In violett der induzierte Morphismus  $Y \to  \lim X$ (der sogar ein Morphismus in $\cat{C} / X$ ist!), der eindeutig existiert, weil $\lim X$ ein Terminalobjekt in  $\cat{C} / X$ ist.
\end{definition}
\begin{dnotation}
    Existiert 'der' (eigentlich ein) Limes von $X \colon  \cat{I} \to  \cat{C}$, so notieren wir ihn mit $\lim X$.
\end{dnotation}


\begin{warning}
    Es muss nicht immer ein Limes existieren (d.h. es gibt nicht immer ein terminales Objekt in jeder Kategorie)
\end{warning}

\begin{oral}
    Limiten sind eindeutig, wenn sie existieren (bis auf eindeutigen Isomorphismus)
\end{oral}
\begin{proof*}
    Das ganze ist ein 'Standard'-Beweis für Objekte mit universeller Eigenschaft. Seien $Y, Y'$ beides Limiten über  $X \colon  \cat{I} \to  \cat{C}$, wir werden zeigen, dass diese isomorph mit eindeutigem Isomorphismus sind. Nach universeller Eigenschaft existiert zunächst eine (eindeutige) Abbildung $g\colon  Y \to  Y'$ in der Kegelkategorie, sodass das Diagramm
    \[
    \begin{tikzcd}
        & {\color{blue}Y}\ar[bend right = 10, blue, swap]{ddl}{h_1} \ar[blue, bend left = 10]{ddr}{h_2} \ar[violet, dashed, "g" description]{d}  \\
        & {\color{red}Y'} \ar[red,swap]{dl}{s_1} \ar[red]{dr}{s_2} \ar[dashed, "s_0" description, near start, red]{dd}\\
        X_1 \ar[swap]{dr}{f_1} & & X_2\ar{dl}{f_2} \\
         & X_0
    \end{tikzcd}
\]
kommutiert. Analog finden wir jedoch $f\colon  Y' \to  Y$ als Morphismus in der Kegelkategorie, denn $Y$ ist ja ebenfalls ein Limes und  $Y'$ ist ein Kegel. Nun ist  $f \circ  g \colon  Y \to  Y$ eine Abbildung von $Y$ (ein Kegel) nach  $Y$ (der Limes) in der Kegelkategorie. Allerdings ist solch eine Abbildung eindeutig bestimmt (Definition des Terminalobjekts), und wir wissen auch, dass offensichtlich die Identitätsabbildung  $\id : Y \to  Y$ ein Morphismus ist, der ebenfalls zur Kegelkategorie gehört. Also muss bereits $f \circ  g = \id_Y$ gelten. Völlig analog zeigt man $g \circ  f = \id_{Y'}$, und damit sind $g$ bzw.  $f$ die geforderten Isomorphismen. \\
Für die Eindeutigkeit von  $f,g$ reicht es zu bemerken, dass ja bereits die Abbildung $g\colon Y\to Y'$ eindeutig bestimmt war, also kann es erst recht keine weiteren Isomorphismen geben.
\end{proof*}


\begin{example}
    \begin{enumerate}[1)]
        \item Sei $\cat{I} = \emptyset$ die leere Kategorie, dann ist $\cat{C} / X \cong \cat{C}$. Also ist $\lim X$ ein terminalse Objekt.
        \item Sei $\cat{I} = \left \{1,2\right\} $ (mit nur den Identitätsabbildungen). Sei $X\colon  \mathcal{I}t \to  \cat{C}$ gegeben ducrh $X_1,X_2\in \Ob(\cat{C})$, dann ist $\lim X$ das  \underline{Produkt} von $X_1$ und $X_2$. \\
            In $\Top$ und  $\Set$ ist dieses das klassiche / bekannte Produkt  $X_1\times X_2$ (mit Projektionen!)
        \item Sei 
            \[
            \cat{I} = 
            \begin{tikzcd}
                & \star \ar{d} \\
                \star \ar{r} & \star 
            \end{tikzcd}
            \qquad
            X = 
            \begin{tikzcd}
                & X_2 \ar{d}{f_2} \\
                X_1 \ar[swap]{r}{f_1} & X_0
            \end{tikzcd}
            .\] 
            dann ist $\lim X$ das  \vocab{Faserprodukt} (Pullback). In $\Top$ bzw.  $\Set$ ist
             \[
                 \lim X = \left \{(x_1,x_2) \in  X_1\times X_2 \mid  f_1(x_1) = f_2(x_2)\right\} 
            .\] 
            mit der Teilraumtopologie des Produktes und inklusive der entsprechenden Projektionen.
    \end{enumerate}
\end{example}

\begin{oral}
    Zu einem Limes gehören immer auch die entsprechenden Abbildungen, oft unterdrücken wir diese jedoch implizit.
\end{oral}

\begin{definition}[Initiales Objekt]\label{def:initiales-objekt}
    Ein initiales Objekt in $\cat{C}$ ist ein Objekt $X\in \Ob(\cat{C})$, sodass für alle $Y\in \Ob(\cat{C})$ eine eindeutige Abbildung von $X$ nach  $Y$ existiert, d.h.
     \[
         \forall Y\in \Ob(\cat{C})\colon  \qquad \Mor_{\cat{C}}(X,Y) = \left \{\star\right\} 
    .\] 
\end{definition}

\begin{example}
    In $\Top$ bzw.  $\Set$ ist  $\emptyset$ das initiale Objekt.
\end{example}

\begin{definition}[Kokegelkategorie]\label{def:kokegelkategorie}
    Die Definition ist völlig analog zu der des Kegels, nur dass wir die Richtungen der Morphismen umdrehen:


    Ein \vocab{Kokegel} über $X\colon  \cat{I} \to  \cat{C}$ ist ein Paar $(Y, \left \{f_i\right\} _{i\in \cat{I}})$, sodass $f_i \in  \Mor_{\cat{C}}(X_i, Y)$ (die Morphismen gehen jetzt in die andere Richtung) und für jeden Morphismus $h\in \Mor_{\cat{I}}(i,j)$ das entsprechende Diagramm
    \[
    \begin{tikzcd}
        X_i \ar{rr}{X(h)} \ar[swap]{dr}{f_i} & & X_j \ar{dl}{f_j} \\
         & Y
    \end{tikzcd}
    .\] 
    kommutiert. Die Kokegelkategoriebesteht $X / \cat{C}$ (sprich: '$\cat{C}$ unter $X$') besteht nun aus allen Kokegeln, wobei ein Morphismus $(Y, \left \{f_i\right\} ) \to  (Y', \left \{f_i'\right\} )$ gegeben ist durch $g\in \Mor_{\cat{C}}(Y,Y')$, sodass für alle $i\in \Ob(\cat{I})$ folgendes kommutiert:
    \[
        \begin{tikzcd}
    & X_i \ar[swap]{dl}{f_i} \ar{dr}{f_j} \\
            Y \ar[swap]{rr}{g}& & Y'
        \end{tikzcd}
    .\] 
\end{definition}

\begin{definition}[Kolimes]\label{def:kolimes}
Ein \vocab{Kolimes} ist ein Initialobjekt in der Kokegelkategorie $X / \cat{C}$. 
\end{definition}

\begin{warning}
    Auch hier gilt wieder: Initialobjekte und damit Kolimiten müssen nicht immer existieren.
\end{warning}

\begin{oral}
    Auch das Konzept des Kolimes kann man wieder auf das des Limes zurückführen, indem wir die duale Kategorie einführen.
\end{oral}

\begin{lemma**}[Kolimes als Limes in der Dualkategorie]
    TODO
\end{lemma**}


\todo{ausarbeiten}

\begin{example}
    \begin{enumerate}[1)]
        \item Ist $\cat{I} = \emptyset$, so erhalten wir ein initiales Objekt.
        \item Ist $\cat{I} = \left \{1,2\right\} $ und $X = X_1,X_2$, dann ist $\colim X$ das  \vocab{Koprodukt}. In $\Top$ bzw.  $\Set$ ist dieses  $X_1 \coprod X_2$ (mit den entsprechenden Inklusionen).
        \item Betrachte
            \[
            \cat{I} = 
            \begin{tikzcd}
                \star \ar{r} \ar{d} & \star \\
                \star
            \end{tikzcd}
            \qquad
            X = 
            \begin{tikzcd}
                X_0 \ar{r}{f_2} \ar[swap]{d}{f_1}&X_2 \\
                X_1
            \end{tikzcd}
            .\] 
            Dann ist $\colim X$ das  \vocab{Pushout} (Kofaserprodukt). In $\Top$ bzw.  $\Set$ ist das  $X_1 \bigcup\limits_{X_0} X_2$ 
    \end{enumerate}
\end{example}
