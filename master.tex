\documentclass[a4paper, german, lecturenumbers = true, number small environments = theorem, hide version]{mkessler-script}

\course{Einführung in die Geometrie und Topologie}
\lecturer{Daniel Kasprowski}
\assistant[f]{Arunima Ray}
\author{Maximilian Keßler}

\usepackage{mkessler-geotopo}
\restatesetup{strict=false}

\begin{document}
    \maketitle
    \begin{abstract}
    {\color{red} Achtung:} Diese Version des Skripts benutze ich zur Bearbeitung! Einige Dinge fehlen, dafür gibt es TODO-Notes. Für Inhalte, benutzt die \href{https://kesslermaximilian.github.io/LectureNotesBonn/2021_Topologie.pdf}{normale Version}
    \end{abstract}
    \newpage
    \listoftodos
    \newpage
    \summaryoflectures
    \newpage
    % start lectures
    \setcounter{abspage}{7}
    \setcounter{page}{8}
    \setcounter{bookmark@seq@number}{1}
    \import{inputs/lectures/}{lec_01.tex}
    \import{inputs/lectures/}{lec_02.tex}
    \import{inputs/lectures/}{lec_03.tex}
    \import{inputs/lectures/}{lec_04.tex}
    \import{inputs/lectures/}{lec_05.tex}
    \import{inputs/lectures/}{lec_06.tex}
    \import{inputs/lectures/}{lec_07.tex}
    \import{inputs/lectures/}{lec_08.tex}
    \import{inputs/lectures/}{lec_09.tex}
    \import{inputs/lectures/}{lec_10.tex}
    \import{inputs/lectures/}{lec_11.tex}
    \import{inputs/lectures/}{lec_12.tex}
    \import{inputs/lectures/}{lec_13.tex}
    \import{inputs/lectures/}{lec_14.tex}
    \import{inputs/lectures/}{lec_15.tex}
    \import{inputs/lectures/}{lec_16.tex}
    \import{inputs/lectures/}{lec_17.tex}
    \import{inputs/lectures/}{lec_18.tex}
    \import{inputs/lectures/}{lec_19.tex}
    \import{inputs/lectures/}{lec_20.tex}
    \import{inputs/lectures/}{lec_21.tex}
    \import{inputs/lectures/}{lec_22.tex}
    \import{inputs/lectures/}{lec_23.tex}
    \import{inputs/lectures/}{lec_24.tex}
    % end lectures
\end{document}
