\documentclass[a4paper, german, lecturenumbers = true, number small environments = theorem, hide version]{mkessler-script}

\course{Einführung in die Geometrie und Topologie}
\lecturer{Daniel Kasprowski}
\assistant[f]{Arunima Ray}
\author{Maximilian Keßler}

\usepackage{mkessler-hypersetup}
\usepackage{mkessler-math}
\usepackage{mkessler-fancythm}
\usepackage{mkessler-vocab}
\usepackage{mkessler-bibliography}
\usepackage[fancyhead]{mkessler-lectures}
\usepackage{castel-figures}

\RequirePackage{float}

% for wrapping text around figures
\RequirePackage{wrapfig}

% Don't indent paragraphs, leave some space between them
\RequirePackage{parskip}

% Hide page number when page is empty
\RequirePackage{emptypage}

% Put x \to \infty below \lim
\let\svlim\lim\def\lim{\svlim\limits}

\import{inputs/exercises/}{preambleBlatt.tex}

\usepackage{todonotes}
\setuptodonotes{disable}


\usepackage{import}


\begin{document}
    \maketitle
    \begin{abstract}
    {\color{red} Achtung:} Diese Version des Skripts benutze ich zur Bearbeitung! Einige Dinge fehlen, dafür gibt es TODO-Notes. Für Inhalte, benutzt die \href{https://kesslermaximilian.github.io/LectureNotesBonn/2021_Topologie.pdf}{normale Version}
    \end{abstract}
    \newpage
    \listoftodos
    \newpage
    \summaryoflectures
    \newpage
    % start lectures
    \setcounter{section}{20}
    \setcounter{dummy}{8}
    \setcounter{smalldummy}{0}
    \setcounter{figure}{29}
    \setcounter{claim}{1}
    \setcounter{lecture}{22}
    \import{inputs/lectures/}{lec_23.tex}
    % end lectures
\end{document}
