\textbf{Erinnerung:} Ab jetzt sind alle Abbildungen stetig.
\begin{aufgabe}
	\begin{enumerate}[i)]
		\item Die Homotopiekategorie $\hTop$ ist tatsächlich eine Kategorie, d.h. Komposition von Homotopieklassen von Abbildungen ist wohldefiniert.
		\item Die Isomorphismen in $\hTop$ sind gerade die Homotopieäquivalenzen, d.h. für $f\colon X\to Y$ ist $[f]\in\Mor_{\hTop}(X,Y)$ genau dann ein Isomorphismus, falls $g\colon Y\to X$ existiert mit $g\circ f\sim \id_X$ und $f\circ g\sim \id_Y$.
	\end{enumerate}	
\end{aufgabe}

\begin{aufgabe}
	Es seien $(X,x)$ und $(Y,y)$ zwei punktierte topologische Räume. Zeige, dass es einen kanonischen Isomorphismus 
	\[ \pi_1(X\times Y, (x,y)) \cong \pi_1(X,x) \times \pi_1(Y,y) \]
	gibt.
\end{aufgabe}

\begin{aufgabe} Sei $X$ ein topologischer Raum.
	\begin{enumerate}[i)]
		\item Der Kegel $CX:=(X\times[0,1])/(X\times\{1\})$ ist \emph{zusammenziehbar}, d.h. homotopieäquivalent zum Einpunktraum.
		\item Eine Abbildung $f\colon X\to Y$ ist homotop zu einer konstanten Abbildung genau dann, wenn $F\colon CX\to Y$ existiert mit $f(-)=F(-,0)$.
		\item $CS^{n}$ ist homöomorph zu $D^{n+1}$, insbesondere ist eine Abbildung $f\colon S^n\to X$ genau dann homotop zu einer konstanten Abbildung, wenn $f$ sich auf $D^{n+1}$ fortsetzen lässt.
	\end{enumerate}
\end{aufgabe}

\begin{aufgabe}
	In der Vorlesung wurde gezeigt, dass punktierte Abbildungen $f\colon (X,x)\to (Y,y)$ Gruppenhomomorphismen $f_*\colon \pi_1(X,x)\to \pi_1(Y,y)$ induzieren. 
	\begin{enumerate}[i)]
		\item Wenn zwei punktierte Abbildungen $f, g\colon (X,x) \to (Y,y)$ punktiert homotop sind, dann stimmen die induzierten Abbildungen $f_*$ und $g_*$ überein.
		\item Für punktiert homotopieäquivalente Räume $(X,x)$ und $(Y,y)$, d.h. es gibt punktierte Abbildungen in beide Richtungen, deren Kompositionen punktiert homotop zu den Identitäten sind, sind die Fundamentalgruppen $\pi_1(X,x)$ und $\pi_1(Y,y)$ isomorph.
		\item $\pi_1(\mathbb R^n, 0) \cong 0$.
	\end{enumerate}
\end{aufgabe}
