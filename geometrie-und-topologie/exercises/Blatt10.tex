
\begin{aufgabe}\label{freieSchlingen}
		Wir bezeichnen mit $\mathcal S(X)$ die Menge
	der {\itshape freien} Homotopieklassen von Schleifen in einem Raum $X$,
	also Homotopieklassen stetiger Abbildungen $f\colon S^1\to X$,
	wobei Basispunkte keine Rolle spielen.
	\begin{enumerate}[i)]
		\item Für jeden Basispunkt $x\in X$ induziert das 
		"`Vergessen des Basispunktes"' eine wohldefinierte Abbildung
		\[ \pi_1(X,x) \ \to \ \mathcal S(X)\ . \]
		\item Die Vergiss-Abbildung ist genau dann surjektiv,
		wenn $X$ wegzusammenhängend ist.
		\item Zwei Elemente aus $\pi_1(X,x)$ haben genau dann dasselbe Bild
		unter der Vergiss-Abbildung, wenn sie in der
		Gruppe $\pi_1(X,x)$ konjugiert sind. Das heißt, dass $w, \, w^\prime \in \pi_1(X,x)$ genau dann dasselbe Bild in $\mathcal S(X)$ haben, wenn ein $\gamma\in \pi_1(X,x)$ existiert, sodass $w= \gamma \ast w^\prime \ast \gamma\inverse$.
		\item Folgere, dass die Vergiss-Abbildung
		\[ \pi_1(S^1,1) \ \to \ \mathcal S(S^1)\  \]
		für $X=S^1$ eine Bijektion ist.
	\end{enumerate}
\end{aufgabe}

\begin{aufgabe}[Brouwer'scher Fixpunkt-Satz] Jede stetige
	Selbstabbildung $f\colon D^2 \to D^2$ der Kreisscheibe $D^2$ hat einen Fixpunkt.\\~\\	
	\textbf{Hinweise:}
	\begin{enumerate}[i)]
		\item Zeige, dass der Kreis $S^1$ nicht Retrakt der Kreisscheibe $D^2$ ist, dass also keine stetige Abbildung $r\colon D^2\to S^1$ existiert, sodass $r|_{S^1} = \id_{S^1}$ die Identität auf $S^1$ ist.
		
		\item Angenommen es gäbe eine stetige Abbildung $f\colon D^2 \rightarrow D^2$ ohne Fixpunkt. Konstruiere aus diesem $f$ eine neue Abbildung
		$g \colon D^2 \rightarrow S^1$ wie folgt: $g$ bildet einen Punkt $x \in D^2$ auf	denjenigen Kreispunkt ab, der auf der Geraden durch $x$ und $f(x)$ auf der Seite von $x$ liegt. Zeige, dass dieses $g$ eine wohldefinierte und stetige Abbildung ist.
	\end{enumerate}
	\textbf{Bemerkung}
		Der Brouwer'sche Fixpunktsatz gilt auch für Selbstabbildungen von $D^n$. Man kann denselben Beweis verwenden, sobald man eine angemessene Verallgemeinerung der Fundamentalgruppe auf höhere Dimensionen hat. Mit zwei solchen Invarianten, Homologie- und Homotopiegruppen, werden sich die nächsten Topologiekurse beschäftigen.
\end{aufgabe}

%\begin{aufgabe}[Fundamentalsatz der Algebra]
%	Zeigen Sie, dass  jedes	nicht-konstante Polynom mit Koeffizienten in den komplexen Zahlen eine Nullstelle hat.
%	
%	Der Beweis soll wie folgt geführt werden: Für ein normiertes Polynom $f(x) \in \mathbb{C}[x]$ ohne Nullstelle und $t\geq 0$
%	betrachten wir die Abbildung
%	\[ \tilde f_t \colon S^1 \rightarrow \mathbb C\setminus \{0\}, \, \tilde f_t(z) = f(tz)\]
%	
%	\begin{enumerate}[i)]
%		\item Für jedes $t\geq 0$ ist die Abbildung $\tilde f_t$ homotop zu einer konstanten Abbildung.
%		
%		\item  Es sei $k$ der Grad des Polynoms $f$, d.\ h.\ es ist
%		\[f(z) = z^k + \sum_{i=0}^{k-1} a_{i} \cdot z^{i}.\]
%		Es sei $C \geq 0$ eine Konstante, für die die Ungleichung
%		\[C^k > \sum_{i=0}^{k-1} |a_{i}|C^{i} \]
%		gilt. Zeige, dass  $\tilde f_C$ homotop zur Abbildung $z \mapsto C^kz^k$ ist.
%		\item Benutze die Teile i) und ii) sowie die Projektionsabbildung 
%		\[ \mathbb C\setminus \{0\} \to S^1, \, z \mapsto \frac{z}{|z|}, \]
%		und die Berechnung von ${\mathcal S}(S^1)$ aus Aufgabe \ref{freieSchlingen}, um zu schließen, dass $f$ ein konstantes Polynom sein muss.
%	\end{enumerate}
%\end{aufgabe}

\begin{aufgabe}
	Klassifiziere bis auf Isomorphie alle Überlagerungen $p\colon E\to S^1\vee D^1$, deren Totalraum $E$ kompakt und wegzusammenhängend ist. Dabei soll jeweils ein	explizites Model konstruiert werden.
\end{aufgabe}

\begin{aufgabe}
	Es sei $G$ eine topologische Gruppe (z.B. eine Lie-Gruppe), das heißt eine Gruppe $(G, \cdot, e)$ mit einer Topologie auf der unterliegenden Menge von $G$, sodass die Multiplikation 
	\[ \cdot\colon G\times G\to G \]
	und die Bildung von Inversen \[ ( \_\ )\inverse\colon G\to G\]
	stetig sind. Zeige, dass $\pi_1(G,e)$ eine abelsche Gruppe ist.
	
	\textbf{Hinweis:} Zeige folgendes Resultat, welches als {\itshape Eckmann-Hilton Argument} bekannt ist:\\
	Es sei $G$ eine Menge mit zwei Verknüpfungen $\cdot$ und $\ast$ und Elementen $e$ und $e^\prime$, sodass sowohl $(G,\cdot, e)$ als auch $(G, \ast, e^\prime)$ Gruppen sind. Es gelte zudem für alle $a,\, b,\, c,\, d\in G$ die Austauschrelation
	\[ (a\cdot b)\ast (c\cdot d)= (a\ast c)\cdot (b\ast d).\]
	Zeige, dass dann $e=e^\prime$ gilt, die Multiplikationen $\cdot$ und $\ast$ übereinstimmen und die Gruppe $G$ mit dieser Multiplikation abelsch ist.
\end{aufgabe}

%\begin{aufgabe}
%	Es sei $G$ eine Gruppe und $E$ ein $G$-Raum, d.h. $E$ ist ein $G$-Objekt in $\Top$ oder anders ausgedrückt es gibt einen Gruppenhomomorphismus $c$ von $G$ in die Homeomorphismen von $E$, also ist für jedes $g\in G$ die Abbildung $c_g\colon E\to E,\ e\mapsto ge$ stetig und $c_{gh}=c_{g}\circ c_{h}$ für alle $g,h\in G$.
%	Die Wirkung ist \emph{frei und eigentlich diskontinuierlich}
%	falls jeder Punkt $e\in E$ eine Umgebung $U$ besitzt, so dass
%	für alle $g\in G\backslash\{1\}$ gilt, dass $gU\cap U=\emptyset$.
%	Wir bezeichnen mit $G\backslash E$ den Quotientenraum von $E$ nach
%	der Äquivalenzrelation $e\sim ge$ für alle $e\in E$ und $g\in G$.
%	
%	\begin{enumerate}[i)]
%%		\item Sei $G$ eine endliche Gruppe, die stetig und {\em frei} auf einem
%%		Hausdorffraum $E$ wirkt 
%%		(d.h. aus $eg=e$ f"ur $e\in E, g\in G$ folgt schon $g=1$). Zeigen Sie,
%%		dass die Wirkung von $G$ auf $E$ eigentlich diskontinuierlich ist.
%%		%\item Finden Sie eine Gruppe $G$ (notwendigerweise unendlich) und eine
%%		%stetige, freie Wirkung von~$G$ auf einem Hausdorffraum, die nicht
%%		%eigentlich diskontinuierlich ist.
%		\item F"ur jede freie und eigentlich diskontinuierliche Wirkung
%		einer Gruppe $G$ auf einem Raum $E$ ist die Quotientenraumprojektion
%		\[ E \ \to \ E/G \]
%		eine Überlagerung ist. %Was ist die Blätterzahl?
%		\item Sei $E$ ein einfach-zusammenhängender Raum, auf dem eine Gruppe $G$ frei und eigentlich diskontinuierlich wirkt. Für $e\in E$ ist die Fundamentalgruppe $\pi_1(E/G, e)$ isomorph zu $G$. 
%	\end{enumerate}
%\end{aufgabe}
