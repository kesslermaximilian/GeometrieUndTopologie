\begin{aufgabe}Es sei $X$ ein normaler topologischer Raum und $A\subset X$ ein Unterraum.
\begin{enumerate}[i)]
	\item Ist $A$ abgeschlossen, so ist $A$ normal.
	\item Gib ein Beispiel für $X$ und $A$, so dass $A$ nicht normal ist.\\
	\textbf{Hinweis:} Für diese Teilaufgabe heißt normal nur folgendes: Für disjunkte abgeschlossene Teilmengen $B_1,B_2$ gibt es disjunkte offene Umgebungen. D.h. es genügt ein Beispiel in dem $X$ nicht notwendig Hausdorffsch ist. Beispiele mit $X$ Hausdorffsch gibt es auch, sind aber deutlich schwieriger zu finden.
	\item Es sei $f\colon X\to Y$ eine stetige, surjektive und abgeschlossene Abbildung. Dann ist auch $Y$ normal.
\end{enumerate}
\emph{Hinweis:} Schritte im Beweis von iii), die genau analog sind zu solchen im Beweis von Satz 5.11, müssen nicht neu bewiesen werden. Ein Verweis genügt.
\end{aufgabe}
\begin{aufgabe}
	\begin{enumerate}[i)]
		\item Es sei $X$ eine Menge und $\cS$ eine Menge von Teilmengen von
		$X$. Sei
		\[ \mathcal T(\cS) := \left\{ \bigcup_{i\in I} \bigcap_{k=1}^{n_i} S_{i,k}\mid S_{i,k}\in \cS, n_i\geq 0 \right\}. \]
		\emph{Hinweis: $X$ ist als leerer Schnitt (d.h. $n_i=0$ für ein $i\in I$) in $\cT(\cS)$ enthalten.}
		\begin{enumerate}[a)]
			\item ${\mathcal T}(\cS)$ ist eine Topologie auf $X$.
			\item Ist $\cO$ eine Topologie auf $X$ mit $S\in \cO$ für alle $S\in\cS$, so gilt $\cT(\cS)\subseteq\cO$.
		\end{enumerate}
		\item Es sei $X$ ein metrischer Raum, $t>0$. Welche der folgenden Mengen von Teilmengen von $X$ bilden eine Basis der induzierten Topologie auf $X$?
		\begin{enumerate}[a)]
			\item $\mathcal U_t :=\{ U(x, \varepsilon)\mid x\in X, \varepsilon < t\} $.
			\item $\mathcal U'_t :=\{ U(x, \varepsilon)\mid x\in X, \varepsilon > t\} $.
			\item $\mathcal U'' :=\{ U(x,\tfrac{1}{n})\mid x\in X, n\in\N_{>0}\} $.
		\end{enumerate}
	\end{enumerate}
\end{aufgabe}
\begin{aufgabe}
	Zeige, dass $\cS:=\{(-\infty,a)\mid a\in\R\}\cup\{(a,\infty)\mid a\in\R\}$ eine Subbasis der euklidischen Topologie auf $\R$ ist. Benutze diese um nochmal zu zeigen, dass das Einheitsintervall $[0,1]$ kompakt ist.
\end{aufgabe}
\begin{aufgabe}
	Für eine Abbildung $f\colon X \rightarrow Y$ zwischen topologischen
	Räumen bezeichne
	\[\Gamma(f) = \{ \ (x,f(x)) \mid x \in X\} \ \subset\ X \times Y\]
	den {\it Graph} von $f$, versehen mit der Unterraumtopologie der
	Produkttopologie auf $X \times Y$.
	\begin{enumerate}[i)]
		\item Die Abbildung \[p_X|_{\Gamma(f)}\colon\Gamma(f) \rightarrow X\ ;\quad (x,y) \mapsto x \]
		ist eine stetige Bijektion.
		\item $p_X|_{\Gamma(f)}$ ist genau dann offen (also ein Homöomorphismus), wenn $f$ stetig ist.
	\end{enumerate}
\end{aufgabe}
