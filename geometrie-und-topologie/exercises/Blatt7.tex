\begin{aufgabe} Sei $\cC$ eine Kategorie.
	\begin{enumerate}[(i)]
		\item Terminale Objekte in $\cC$ sind eindeutig bis auf einen eindeutigen Isomorphismus.
		\item Sei $I$ eine kleine Kategorie und $X\colon I\to \cC$ ein Funktor. Sind $(L,s_i)$ und $(L',s_i')$ Limiten von $X$, so gibt es einen eindeutigen Isomorphismus $f\colon L\to L'$ mit $s_i'\circ f=s_i$ für alle $i\in\ob(I)$.
	\end{enumerate}
	\emph{Hinweis:} Die analoge Aussage gilt für initiale Objekte und Kolimiten.
\end{aufgabe}
\begin{aufgabe}
	Für $X\in\ob(\Top)$ sei $\pi_0(X)\in\ob(\Set)$ die Menge der Wegekomponenten von $X$, d.h. die Menge der maximalen wegzusammenhängenden Teilräume. 
	\begin{enumerate}[(i)]
		\item Definiere $\pi_0$ auf Morphismen derart, dass $\pi_0\colon \Top\to \Set$ ein Funktor ist.
		\item Gib eine natürliche Transformation vom Vergissfunktor $F\colon \Top\to\Set$ nach $\pi_0$ an.
	\end{enumerate}
\end{aufgabe}
\begin{aufgabe}
	Ein Funktor $F\colon\cC\to\cD$ ist genau dann eine Äquivalenz, wenn
	\begin{enumerate}[i)]
		\item $F$ ist \emph{essenziell surjektiv}, d.h. für alle $X\in\ob(\cD)$ exisitiert ein $Y\in\ob(\cC)$, so dass $F(Y)$ und $X$ isomorph sind, und
		\item $F$ ist \emph{volltreu}, d.h. für alle $X,Y\in\ob(\cC)$ ist $F\colon \Mor_\cC(X,Y)\to \Mor_\cD(F(X),F(Y))$ eine Bijektion.
	\end{enumerate}
	\emph{Hinweis:} Zur Konstruktion des Inversen wähle für jedes $X$ ein $Y$ und einen Isomorphismus $f\colon F(Y)\to X$.
\end{aufgabe}
\begin{aufgabe}
	Sei $I$ eine kleine Kategorie und $X\colon I\to\Top$ ein Funktor. 
	\begin{enumerate}[(i)]
		\item Der Raum $\{(x_i)_{i\in I}\in \prod_{i\in I} X_i\mid h(x_i)=x_j~\forall h\in\Mor_I(i,j)\}$ zusammen mit den von den Projektionen induzierten Abbildungen ist ein Limes von $X$. Also ist $\Top$ \emph{vollständig}, d.h. alle Limiten in $\Top$ existieren.
		\item Der Raum $\coprod_{i\in I}X_i/\sim$ mit der Äquivalenzrelation erzeugt von $h(x_i)\sim x_j$ für alle $h\in\Mor_I(i,j)$ zusammen mit den von den Inklusionen induzierten Abbildungen ist ein Kolimes von $X$. Also ist $\Top$ \emph{kovollständig}, d.h. alle Kolimiten in $\Top$ existieren.
	\end{enumerate}
\end{aufgabe}
