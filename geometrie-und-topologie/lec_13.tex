%! TEX root = ./master.tex
\lecture[]{Di 08 Jun 2021 12:15}{}
\begin{example}
    \begin{enumerate}[1)]
        \item Wir können eine Gruppe $G$ als Kategorie auffassen, indem wir  $\Ob{\cat{G}} = \left \{\star\right\} $ und $\Mor_{\cat{G}}(\star,\star) = G$ setzen, wobei natürlich $g \circ  h = gh$. Man könnte das in etwa so skizzieren:
            \[
\begin{tikzcd}
            \bullet \ar[loop left]{}{f} \ar[loop above]{}{\id} \ar[loop right]{}{g} \ar[loop below]{}{f \circ g}
\end{tikzcd}
\]
\item Ein \vocab{G-Objekt} in einer Kategorie $\cat{C}$ ($G$ ist eine Gruppe) ist ein Funktor  $\cat{G} \to  \cat{C}$ (dieser Funktor besteht aus einem Objekt von $\cat{C}$ zusammen mit Endomorphismen dieses Objekts für jedes $g\in G$)
    \begin{remark*}
        Ein typisches Beispiel sind Gruppenwirkungen, wählen wir hier $\cat{C} = \Set$, so sind die $G$-Objekte genau  $G$-Mengen bzw.  $G$ wirkt dann auf die entsprechende Menge  $\mathcal{F}(\cat{G})\in \Set$.
    \end{remark*}
\item Sind $\cat{C}$, $\cat{D}$ Kategorien, so gibt es die \vocab{Produktkategorie} $\cat{C} \times  \cat{D}$, die wir erhalten, indem wir $\Ob{\cat{C} \times \cat{D}} = \Ob{\cat{C}} \times \Ob{\cat{D}} $ (dieses Produkt müssen wir potenziell als das von Klassen auffassen) setzen und als Morphismen
    \[
        \Mor_{\cat{C}\times \cat{D}}((X,Y),(X',Y')) \coloneqq  \Mor_{\cat{C}}(X,X') \times \Mor_{\cat{D}}(Y,Y') 
    .\] 
    setzen mit komponentenweiser Komposition, d.h. $(f,g) \circ  (f', g') = (f \circ  f', g \circ  g')$.
\item Das Wedge-Produkt ($\twedge$) können wir nun als Funktor
     \[
    \twedge \Top_{\star} \times \Top_{\star} \to  \Top_{\star}
    .\] 
    auffassen, mittels
    \[
        ((X,x_0),(Y,y_0)) \mapsto X \bigcup\limits_{\left \{\star\right\} }Y / (x_0 \sim  y_0)
    .\] 
wobei der Punkt $x_0 \sim y_0$ der Basispunkt des neuen Raumes ist.


    Analog geht das auch für das Wedge-Produkt, d.h. $\tsmash \Top_{\star} \times  \Top_{\star} \to  \Top_{\star}$ 
    \begin{example}
        \todo{Fehlt}
    \end{example}
\item Die Funktorkategorie $\Mor(\cat{C},\cat{D})$ mit $\Ob(\Mor(\cat{C},\cat{D}))$ als der Menge der Funktoren von $\cat{C}$ nach $\cat{D}$, und $\Mor_{\Mor_{\cat{C},\cat{D}}}(\mathcal{F},\mathcal{G})$ als die Menge der natürlichen Transformationen von $\mathcal{F}$ nach $\mathcal{G}$
    \end{enumerate}
\end{example}



\section{Homotopien und die Fundamentalgruppen}
\begin{oral}
    Wir reden von nun an über 'Abbildungen', wobei wir damit meinen, dass jede Abbildung automatisch stetig ist.
\end{oral}

\begin{definition}[Homotopie]\label{def:homotop}
    Zwei Abbildungen $f,g\colon  X \to  Y$ heißen \vocab{homotop}, falls es eine Abbildung
    \[
        H \colon  X \times [0,1] \to  Y
    .\] 
    gibt mit 
    \[
        H_0 \coloneqq  H(-,0) = f \qquad H_1 = g
    .\] 
    Die Abbildung $H$ nennen wir dann  \vocab{Homotopie}. 
\end{definition}

\begin{dnotation}
    Ab nun notieren wir $I \coloneqq  [0,1]$ als das Einheitsintervall
\end{dnotation}

\begin{lemma}
    Homotopie ist eine Äquivalenzrelation auf $\Mor_{\Top}(X,Y)$.
\end{lemma}
\begin{proof}
    \begin{description}
        \item[Reflexivität] Ist $f\colon  X \to  Y$ stetig, so auch  $H \colon  X \times I \stackrel{\pr_X}{\to } X \stackrel{f}{\to } Y $ und klarerweise ist $H_0 = H_1 = f$.
        \item[Symmetrie] Ist $H \colon  X \times I \to  Y$ stetig, so auch
            \[
                H' \colon  X \times I \stackrel{(\id, 1-t)}{\to}  X\times I \to  Y
            .\] 
        \item[Transitivität] Sind $H,G \colon  X \times I \to  Y$ stetig und $H_1 = G_0$, so ist auch die Abbildung
                \begin{equation*}
                HG: \left| \begin{array}{c c l} 
                X\times I & \longrightarrow & Y \\
                (x,t) & \longmapsto &  \begin{cases}
                    H(x,2t) & 0 \leq  t \leq  \frac{1}{2} \\
                    G(x, 2t-1) & \frac{1}{2}\leq t\leq 1
                \end{cases}
                \end{array} \right.
            \end{equation*}
            und wir prüfen leicht $(HG)_0 = H_0$ sowie $(HG)_1 = G_1$.
    \end{description}
\end{proof}

\begin{definition}[Homotop relativ einer Menge]\label{def:homotopie-relativ-einer-menge}
    \begin{enumerate}[i)]
        \item 
    Zwei punktierte Abbildungen $f,g \colon  (X,x_0) \to  (Y,y_0)$ heißen (punktiert) \vocab{homotop}, falls es eine Abbildung
    \[
    H \colon  X \times I \to  Y
    .\] 
    mit $H_0 = g, H_1 = g$ gibt, und zusätzlich $H_t(x_0) = y_0 \forall t\in I$ (wir lassen also den Basispunkt zu jedem Zeitpunkt fest).
\item Sei $A\subset X$ und $f,g\colon  X \to  I$. Die Abbildungen $f,g$ heißen  \vocab{homotop relativ $A$}, falls es
    \[
    H \colon  X \times I \to  Y
    .\] 
    mit $H_0 = g, H_1 = g$ und $H(a,t) = H(a,t')$ für alle  $a\in A, t\in I$ gibt (d.h, die Homotopie bleibt auf $A$ konstant). Inbesondere gilt dies nur, wenn  $f|_A = g|_A$
    \end{enumerate}
\end{definition}

\begin{definition}[Homotopiekategorie]\label{def:homotopiekategorie}
    Die \vocab{(naive) Homotopiekategorie $\hTop$}      ist die Kategorie mit $\Ob(\hTop) = \Ob(\Top)$ und 
    \[
        \Mor_{\hTop}(X,Y) = \Mor_{\Top}(X,Y) / \sim
    \]
    d.h. wir identifizieren Abbildungen modulo Homotopie.
\end{definition}

\begin{definition}[Homotopieäquivalenz]\label{def:homotopieäquivalenz}
    \begin{enumerate}[a)]
        \item 
    Eine Abbildung $f\colon  X \to  Y$ heißt \vocab{Homotopieäquivalenz}, falls $f$ ein Isomorphismus in  $\hTop$ ist, d.h. falls es eine Abbildung $g\colon  Y \to  X$ gibt, so dass $g \circ  f \sim  \id_X$ und $f \circ  g \sim \id_Y$ jeweils homotop zu den Identitäten sind.
\item Existiert eine Homotopieäquivalenz $f\colon  X \to Y$, so heißen $X$ und  $Y$  \vocab{homotopieäquivalent}. 
    \end{enumerate}
\end{definition}

\begin{example}
    Der Einpunktraum $\left \{\star\right\}$ ist homotopieäquivalent zu $\R^n$ mittels
        \begin{equation*}
        f: \left| \begin{array}{c c l} 
        \left \{\star\right\}  & \longrightarrow & \R^n \\
        \star & \longmapsto &  0
        \end{array} \right.
    \end{equation*}
        \begin{equation*}
        g: \left| \begin{array}{c c l} 
        \R^n & \longrightarrow & \left \{\star\right\} \\
        x & \longmapsto &  \star
        \end{array} \right.
    \end{equation*}
    Hierbei ist $g \circ  f = \id_{\left \{\star\right\} }$ sowieso schon die Identität, und es ist $f \circ  g = c_0$ (die konstante Nullabbildung). Mittels
        \begin{equation*}
        H: \left| \begin{array}{c c l} 
        \R^n\times I & \longrightarrow & \R^n \\
        (x,t) & \longmapsto &  tx
        \end{array} \right.
    \end{equation*}
    erhalten wir auch $H_0 = c_0$ und $H_1 = \id_{\R^n}$, sodass wir eine Homotopie $c_0 \sim \id$ gefunden haben.
\end{example}



