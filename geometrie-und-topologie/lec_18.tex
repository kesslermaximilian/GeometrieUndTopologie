%! TEX root = ./master.tex
\lecture[Beweis des Existenzsatzes für universelle Überlagerungen: $H$-äquivalente Wege. Basis der Topologie von semilokal einfachzusammenhängenden, (lokal) wegzusammenhängenden Räumen. Decktransformationen]{Do 24 Jun 2021 10:20}{Existenzsatz für universelle Überlagerungen}

Um die verbleibende Richtung $1)\implies 3)$ zeigen zu können, müssen wir uns im Folgenden recht viel erarbeiten. In allen nachfolgenden Sätzen des Kapitels gehen wir davon aus, dass wir uns in der Situation von \autoref{thm:universelle-überlagerungen-existieren-genau-für-semilokal-einfachzusammenhängende-lokal-wegzusammenhängenden-zusammenhängende-räume} befinden, und dass im Laufe des Kapitels eingeführte Konstruktionen vorhanden sind.

\begin{definition*}\label{def:h-äquivalent-für-untegruppen}
    Sei $H\subset \pi_1(X,x_0)$ eine Untergruppe. Zwei Wege $w,w'$ in  $X$ mit  $w(0) = w'(0) = x_0$ heißen \vocab{$H$-äquivalent}, falls $w(1) = w'(1)$ und  $[w \star \overline{w}']\in H$. 
\end{definition*}

\begin{oral}
Das ist keine gängige Notation, und dient einfach nur, den Beweis besser aufschreiben zu können.    
\end{oral}

\begin{lemma}\label{lm:h-äquivalent-ist-äquivalenzrelation}
    Dies definiert eine Äquivalenzrelation auf
    \[
        \left \{w\colon  I \to  X \mid w(0) = x_0\right\} 
    .\] 
\end{lemma}

\begin{notation*}
    Wir notieren $[w]_H$ für die korrespondierenden Äquivalenzklassen.
\end{notation*}

\begin{proof}[Beweis von \autoref{lm:h-äquivalent-ist-äquivalenzrelation}]
    \begin{description}
        \item[Reflexivität] Klar, denn für $w$ beliebig ist  $[w \star \overline{w}] = [c_{x_0}]\in H$ das neutrale Element.
        \item[Symmetrie] Ist $v \simeq_H w$, also  $[w \star \overline{v}] \in H$, so ist
            \[
                [v \star \overline{w}] = [\overline{w \star \overline{v}}] = [ w \star \overline{v}]^{-1} \in H
            .\] 
            weil $H$ unter Inversen abgeschlossen ist.
        \item[Transitivität] Ist $w \simeq_H w'$ und  $w' \simeq H w''$, so ergibt sich
    \[
        [w \star \overline{w''}] =         [\underbrace{w \star \overline{w'}}_{\in H} \star \underbrace{w' \star \overline{w''}}_{\in H}]
    .\] 
aber $H$ ist unter Komposition abgeschlossen.
    \end{description}
\end{proof}

Nun können wir unseren Überlagerungsraum - als Menge - bereits durch
\[
    E(H) \coloneqq  \faktor{\left \{w\colon  I \to  X \mid  w(0) = x_0\right\} }{\text{H-Äquivalenz}}
.\] 
definieren, zusammen mit der Projektion
    \begin{equation*}
    p: \left| \begin{array}{c c l} 
        E(H) & \longrightarrow & X \\
        \left[w\right]_H & \longmapsto &  w(1)
    \end{array} \right.
\end{equation*}

\begin{remark*}
    Man beachte, dass die Projektion wohldefiniert ist, weil nach \autoref{def:h-äquivalent-für-untegruppen} zwei $H$-äquivalente Wege insbesondere den gleichen Endpunkt besitzen. 
\end{remark*}


 \begin{example}
     Ist $H = \left \{\star\right\} $ trivial, so sind $w$ und  $w'$  $H$-äquivalent genau dann, wenn  $w(1) = w'(1)$ und  $w \simeq w'$ relativ Endpunkten.

     Ist  $H = \pi_1(X,x_0)$, so sind $w$ und  $w'$  $H$-äquivalent genau dann, wenn  $w(1) = w'(1)$.
\end{example}

Unser nächstes Ziel ist es, den (zukünftigen) Überlagerungsraum $E(H)$ mit einer Topologie zu versehen. Dazu bedarf es einiger Vorbereitungen über die Topologie auf $X$ selbst.

\begin{lemma}\label{lm:basis-vonsemilokal-einfachzusammenhängendem-zusammenhängendem-raum}
    Sei $X$ semilokal einfachzusammenhängend, lokal wegzusammenhängend und wegzusammenhängend. Dann ist
     \[
         \mathcal{B} \coloneqq  \left \{U\subset X \text{ offen}\mid  U \text{ ist wegzusammenhängend}, \pi_1(U) \to  \pi_1(X) \text{ ist trivial}\right\} 
    .\] 
    eine Basis der Topologie auf $X$.
\end{lemma}


\begin{proof}
    Sei $V\subset X$ offen, $x\in V$. Es ist zu zeigen, dass es ein $U\in \mathcal{B}$ mit $x\in U\subset V$ gibt. 

    Weil $X$ semilokal einfachzusammenhängend ist, gibt es eine Umgebung  $U'$ von  $x$ mit  $U'$ wegzusammenhängend und $\pi_1(U',x) \to  \pi_1(X,x)$ trivial sowie $U'\subset V$. Da $U'$ Umgebung ist, existiert ein  $U''\subset X$ offen mit $x\in U'' \subset U'$, und weil $X$ lokal wegzusammenhängend und  $U''$ offen ist auch  $U''$ lokal wegzusammenhängend. Die Wegekomponenten von  $U''$ sind nun offen in  $U''$ (nach  \autoref{thm:wegkomponenten-in-lokal-wegzusammenhängendem-raum-sind-offen}) und damit auch in $X$. 

    Sei $U$ die Wegekomponente von  $U''$, die  $x$ enthält. Dann ist  $U\subset X$ offen, wegzusammenhängend, $x\in U$, $U\subset U''\subset U'\subset V$, und
    \[
        \pi_1(U,x) \to  \pi_1(U',x) \stackrel{g \mapsto 0}{\longrightarrow}   \pi_1(X,x)
    .\] 
    ist somit auch die triviale Abbildung. Also ist $U$ ein Basiselement mit $x\in U\subset V$.
\end{proof}

\todo{Brauchen wir Wegzusammenhang im Beweis überhaupt?}

Wir wollen unsere Überlagerung letztendlich so bauen, dass $p$ auf den ebigen Basiselementen der Topologie auf $X$ trivial ist. Wir benötigen also für ein solches Basiselement  $U\in \mathcal{B}$ mit $x\in U$ eine offene Umgebung um jedes Urbild in $p^{-1} (x)$, das homöomorph ist zu $U$, dann liegt folgende Definition nahe:

\begin{dlemmadef}\label{def:offene-mengen-in-universeller-überlagerung}
    Sei $U\in \mathcal{B}$ ein Basiselement, und sei $w\colon  I \to  X$ ein Weg mit $w(1) \in U$ (d.h. insbesondere $p([w]_H) \in U$), so definiere
\[
    \mathcal{B}([w]_H,U) = \left \{[v]_H \in E(H) \mid v = w \star u, u\colon I\to U\right\}  
.\]
\end{dlemmadef}

\begin{proof}[Beweis der Wohldefiniertheit]
    Die Definition hängt nur von $[w]_H$ ab, nicht von der Wahl des Vetreters. Ist nämlich  $w \simeq_H w'$, so auch  $w \star u \simeq_H w' \star u$ wegen
     \[
         w \star u \star \overline{u} \star \overline{w'} \simeq w \star \overline{w'} \quad \implies\quad [(w \star u) \star\overline{w' \star u}] = [w \star \overline{w'}] \in H
    .\] 
\end{proof}

\begin{remark*}
    Intuitiv haben wir also für jedes $U$ und ein Urbild unter  $p$ eines Punktes aus  $U$, d.h. für  $[w]_H$, die Menge der Wege definiert, die 'nicht weit Weg von  $w$' sind, d.h. die sich nur um einen Weg in  $U$ unterscheiden. 

    Das nächste Lemma zeigt, dass sich die so definierte Umgebung nicht ändert, wenn wir $w$ nur 'ein bisschen', d.h. mit einem Weg in  $U$, ändern:
\end{remark*}

\begin{lemma}\label{lm:b-umgebungen-in-e-h-sind-gleich-für-wege-die-sich-nur-um-weg-in-u-unterscheiden-bzw-invariant-unter-ersetzung-durch-element-der-umgebung}
    Sei $[v]_H\in \mathcal{B}([w]_H,U)$. Dann ist
    \[
        \mathcal{B}([v]_H,U) = \mathcal{B}([w]_H,U)
    .\] 
\end{lemma}

\begin{proof}
    '$\supseteq$'. Wir können für $[v]$ einen Repräsentanten der Form $w \star u$ für  $u\colon  I \to U$ wählen, durch ersetzen von $v$ durch diesen Repräsentanten ändert sich die Aussage nicht, also sei OBdA  $v = w \star u$ für ein  $u\colon  I \to  U$.

    Sei $[\tilde{v}]_H \in \mathcal{B}([w]_H,U)$ mit $\tilde{v} = w \star \tilde{u}$ mit $\tilde{u}\colon  I \to  U$ beliebig.

    Dann ist
    \[
        \tilde{v} = w \star \tilde{u} \simeq w \star u \star \overline{u} \star \tilde{u} = v \star (\underbrace{\overline{u} \star \tilde{u}}_{\text{Weg in $U$}})
    .\] 
    Also $[\tilde{v}]_H = [v \star (\overline{u} \star \tilde{u})]_H \in \mathcal{B}([v]_H, U)$.

    Insbesondere ist somit $[w]_H \in  \mathcal{B}([v]_H, U)$. Die Inklusion '$\subset $' folgt also analog.
\end{proof}

\begin{propositiondef}
    Die Mengen $\mathcal{B}([w]_H, U)$ definieren eine Basis einer Topologie auf $E(H)$.
\end{propositiondef}

\begin{proof}
    Nach \autoref{thm:subbasis-ist-basis-wenn-schnitt-generiert-wird} müssen wir nur zeigen, dass der Schnitt zweier solcher Mengen wieder Vereinigung solcher ist, bzw. dass wir um jeden Punkt im Schnitt eine solche Umgebung im Schnitt finden.


    Seien also $B([w]_H, U)$ und  $B([w']_H, U')$ zwei solche offenen Mengen, und wähle  
    \[
        [v]_H \in \mathcal{B}([w]_H, U) \cap \mathcal{B}([w']_H, U')
    \]
    beliebig. Nach \autoref{lm:b-umgebungen-in-e-h-sind-gleich-für-wege-die-sich-nur-um-weg-in-u-unterscheiden-bzw-invariant-unter-ersetzung-durch-element-der-umgebung} sind also die entsprechenden Mengen gleich:
    \[
        \mathcal{B}([w]_H, U) = \mathcal{B}([v]_H, U), \qquad \mathcal{B}([w']_H, U') = \mathcal{B}([v]_H, U')
    .\] 
    Sei $U'' \subset U \cap  U'$, $U'' \in \mathcal{B}$ mit $v(1) \in U''$. Dann ist
    \begin{IEEEeqnarray*}{rCl}
        \mathcal{B}([v]_H, U'') & \subset  & \mathcal{B}([v]_H, U) \cap  \mathcal{B}([v]_H, U') \\
                                & = & \mathcal{B}([w]_H, U) \cap \mathcal{B}([w']_H, U')
    \end{IEEEeqnarray*}
    Die Inklusion ergibt sich dabei unmittelbar aus \autoref{def:offene-mengen-in-universeller-überlagerung}, weil $U'' \subset U, U'$. Offensichtlich ist auch $[v]_H$ noch in dieser Umgebung enthalten.
\end{proof}

Wir statten $E(H)$ nun mit dieser Topologie aus.

 \begin{corollary}
     Mit dieser Topologie ist die Abbildung $p\colon  E(H) \to  X$ stetig.
\end{corollary}


\begin{proof}
    Wir prüfen die Stetigkeit auf der Basis $\mathcal{B}$.


    Es ist zu zeigen, dass für $U\in \mathcal{B}$ auch $p^{-1} (U)\subset E(H)$ offen ist. Es ist
    \[
        p^{-1} (U) = \left \{[w]_H \in E(H) \mid  w(1) \in U\right\} = \bigcup_{[w]_H, w(1) \in U}  \mathcal{B}([w]_H, U)
    .\] 
\end{proof}

\begin{lemma}\label{lm:homöomorphismus-zwischen-u-und-b-u-in-universeller-überlagerung}
    Sei $U\in \mathcal{B}$ und $w\colon  I \to  X$ mit $w(1)\in U$, also $[w]_H \in E(H)$.
    \[
        p|_{B([w]_H, U)} \colon  \mathcal{B}([w]_H, U) \to  U
    .\] 
    ein Homöomorphismus.
\end{lemma}

\begin{proof}
    \begin{description}
        \item[Surjektivität]. Sei $x\in U$. Weil $U$ wegzusammenhängend gibt es einen Weg  $u\colon  I \to  U$ mit $u(0) = w(1)$ und  $u(1) = x$. Dann ist  $[w \star u]_H \in  \mathcal{B}([w]_H, U)$ und $p([w \star u]_H) = ( w \star u) (1) = u(1) = x$. 
        \item[Injektivität] Seien $[v]_H, [\tilde{v}]_H\in \mathcal{B}([w]_H, U)$ mit gleichem Bild unter $p$, d.h.  $v(1) = \tilde{v}(1)$.

            Wir können annehmen, dass $v = w \star u$ und  $\tilde{v} = w \star \tilde{u}$. Dann ist
            \[
                [ v \star \overline{\tilde{v}}] = [ w \star \underbrace{u \star \overline{\tilde{u}}}_{\text{Schleife in $U$}}\star \overline{w}]
            .\] 
            Weil $\pi_1(U) \to  \pi_1(X)$ trivial, ist diese Schleife in $U$ also in  $X$ nullhomotop, und somit
             \[
                 [v \star \overline{\tilde{v}}] = [ w \star \overline{w}] = [c_{x_0}]
            .\] 
            also $[v \star \overline{\tilde{v}}] = 0 \in H$, also wie gewünscht $[v]_H = [\tilde{v}]_H$.
    \end{description}
    Insbesondere ist $p$ offen, da  Basiselemente auf Basiselemente gehen (und Abbildungen mit Vereinigungen vertauschen). Da $p$ auch stetig ist und  $B([w]_H, U)$ offen, ist auch 
     \[
         p|_{B([w]_H, U)}
    .\] 
    offen und stetig, und somit ein Homöomorphismus.
\end{proof}

\begin{proposition}
    $p$ ist eine Überlagerung.
\end{proposition}

\begin{proof}
    Sei $x\in X$, wähle ein Basiselement  $U\in \mathcal{B}$ mit $x\in U$.
    \begin{claim}
        Es ist
        \[
            p^{-1} (U) = \bigsqcup_{[w]_H \in p^{-1} (x)} \mathcal{B}([w]_H, U)
        .\] 
    \end{claim}
    \begin{subproof}
        Die Richtung '$\supseteq$' ist klar.

        Für ' $\subset $' sei $[v]_H \in p^{-1} (U)$ beliebig. Nach \autoref{lm:homöomorphismus-zwischen-u-und-b-u-in-universeller-überlagerung} ist $p|_{\mathcal{B}([v]_H, U)}\colon  \mathcal{B}([v]_H, U) \to  U$ ein Homöomorphismus, also insbesondere surjektiv und wir finden $[w]_H \in \mathcal{B}([v]_H, U)$ mit $p([w]_H) = x$.

        Nach \autoref{lm:b-umgebungen-in-e-h-sind-gleich-für-wege-die-sich-nur-um-weg-in-u-unterscheiden-bzw-invariant-unter-ersetzung-durch-element-der-umgebung} ist nun  $\mathcal{B}([v]_H, U) = B([w]_H, U)$, also 
        \[
            [v]_H \in \mathcal{B}([w]_H, U)
        .\] 

        Die Vereinigung ist auch disjunkt: Angenommen, $[v]_H \in \mathcal{B}([w]_H, U)\cap \mathcal{B}([w']_H, U)$ mit $[w]_H, [w']_H \in p^{-1} (x)$, d.h. $w(1) = w'(1) = x$. Dann erhalten wir nach \autoref{lm:b-umgebungen-in-e-h-sind-gleich-für-wege-die-sich-nur-um-weg-in-u-unterscheiden-bzw-invariant-unter-ersetzung-durch-element-der-umgebung}, dass  
        \[
            B([w]_H,U) = \mathcal{B}([v]_H, U) = \mathcal{B}([w']_H, U)
        \]
        und nach \autoref{lm:homöomorphismus-zwischen-u-und-b-u-in-universeller-überlagerung} ist $p|_{B([w']_H, U)}$ injektiv, also $[w]_H = [w']_H$, denn beide werden auf $w(1) = w'(1) = x$ abgebildet.
    \end{subproof}
    Also können wir unseren gewünschten Homöomorphismus durch
        \begin{equation*}
        \varphi : \left| \begin{array}{c c l} 
            p^{-1} (U) = \bigsqcup\limits_{\left[w\right]_H \in p^{-1} (x)}\mathcal{B}(\left[w\right]_H, U) & \longrightarrow & U\times p^{-1} (x) \\
            \left[v\right]_H \in \mathcal{B}(\left[w\right]_H,U) & \longmapsto &  (p(\left[v\right]_H), \left[w\right]_H)
        \end{array} \right.
    \end{equation*}
    definieren. Dass es sich um einen Homöomorphismus handelt, ergibt sich unmittelbar aus \autoref{lm:homöomorphismus-zwischen-u-und-b-u-in-universeller-überlagerung}. Zudem ist $\varphi $ klarerweise eine Abbildung über $U$, und die Phaser  $p^{-1} (x)$ kann mit der diskreten Topologie versehen werden, weil auf der linken Seite alle Vereinigungsmengen offen sind und jede dieser genau zu einem Punkt der Faser korrespondiert.

    Also ist $p$ eine Überlagerung.
\end{proof}

Es bleibt final zu zeigen, dass unsere Überlagerung charakteristische Untergruppe $H$ hat. Dazu benötigen wir noch eine Vorbereitung über die Wege in  $E(H)$.

\begin{lemma}\label{lm:wege-partiell-zu-laufen-ist-ein-weg-in-e(h)}
    Sei $w\colon  I \to X$ ein Weg mit $w(0) = x_0$. Dann ist 
        \begin{equation*}
        \tilde{w}: \left| \begin{array}{c c l} 
            I & \longrightarrow & E(H) \\
            t & \longmapsto &  [w_t]_H
        \end{array} \right.
    \end{equation*}
    mit $w_t(s) = w(ts)$ eine Hebung von  $w$ mit Anfangspunkt  $e_0 \coloneqq  [c_{x_0}]_H$.
\end{lemma}

\begin{remark*}
    Der Weg $w_t$ ist einfach derjenige Weg, der nur das Anfangsstück von  $w$ durchläuft, das man normalerweise im Intervall  $[0,t]$ zurücklegt. Da  $E(H)$ aus Wegen besteht, ist ein Weg in  $E(H)$ demnach ein Weg von Wegen.

    Was läge also näher, als einen Weg aus $X$ im Zeitverlauf stückweise mehr entlangzulaufen, um ihn nach $E(H)$ zu heben?
\end{remark*}

\begin{proof}[Beweis von \autoref{lm:wege-partiell-zu-laufen-ist-ein-weg-in-e(h)}]
    \begin{description}
        \item[Stetigkeit von $\tilde{w}$:] 
            Wir überprüfen dies für Basiselemente um Bildpunkte von $\tilde{w}$, d.h. es sei $t\in [0,1]$ beliebig, wähle dann $U\in \mathcal{B}$ mit $w(t) \in U$.

            Dann ist $\mathcal{B}([w_t]_H, U)$ eine offene Umgebung von $[w_t]_H$. Es genügt zu zeigen, dass  $\tilde{w}^{-1}(B([w_t]_H, U))\subset I$ offen ist.

    Da $w$ stetig ist, ist $w^{-1}(U)\subset I$ offen, also gibt es $0\leq a < t < b \leq 1$ mit $w((a,b)) \subset U$.
\begin{claim}
    Es ist $\tilde{w}((a,b)) \subset B([w_t]_H, U)$.
\end{claim}
    \end{description}

    \begin{subproof}
        Sei $s\in (a,b)$. Falls $s\geq t$ ist, so ist $w_s \simeq w_t \star w_{t,s}$ wobei $w_{t,s}(r) = w(t + (s-t)r)$ und damit $w_{t,s}([0,1]) = w([t,s]) \subset w((a,b)) \subset U $.

        Also ist $[w_s]_H \in \mathcal{B}([w_t]_H, U)$. Analoges gilt für $s\leq t$ mit $w_s \simeq w_t \star \overline{w_{s,t}}$.
    \end{subproof}
    Also ist $\tilde{w}$ stetig. Nun ist
    \begin{itemize}
        \item $\tilde{w}(0) = [w_0]_H = [c_{x_0}]_H = e_0$.
        \item $p(\tilde{w}(t)) = w_t (1) = w(t)$
    \end{itemize}
    Also hebt $\tilde{w}$ den Weg  $w$.
\end{proof}


\begin{proof}[Fortsetzung des Beweises von \autoref{thm:universelle-überlagerungen-existieren-genau-für-semilokal-einfachzusammenhängende-lokal-wegzusammenhängenden-zusammenhängende-räume}]
    Es fehlt noch die Richtung $1) \implies 3)$.  Wir haben bereits gezeigt, dass $p\colon  E(H) \to  X$ eine Überlagerung ist, es verbleibt zu zeigen, dass sie wegzusammenhängend ist und charakteristische Untergruppe $H$ besitzt.

    Wir definieren der Einfachheit halber $e_0\coloneqq [w_{0}]\in p^{-1} (x_0) \subset E(H)$ als kanonisches Element aus der Faser von $x_0$.

    \underline{1. Schritt}: $E(H)$ ist wegzusammenhängend. Sei  $[w]_H \in E(H)$ beliebig. Nach \autoref{lm:wege-partiell-zu-laufen-ist-ein-weg-in-e(h)} ist
        \begin{equation*}
        \tilde{w}: \left| \begin{array}{c c l} 
            I & \longrightarrow & E(H) \\
            t & \longmapsto &  [w_t]_H
        \end{array} \right.
    \end{equation*}
    eine Hebung von $w$. Es ist  $\tilde{w}(0) = [w_0] = [c_{x_0}] = e_0, \tilde{w}(1) = [w_1]_H = [w]_H$, also ist $\tilde{w}$ ein Weg von $e_0$ nach $[w]_H$, also ist  $E(H)$ wegzusammenhängend, denn wir können von überall nach  $e_0$ laufen.

    \underline{2. Schritt} Es ist $p_*(\pi_1(E(H),e_0)) = H$.

    Nach dem Wegehebungssatz und dem Homotopiehebungssatz gilt:
    \[
        \forall [w] \in \pi_1(X,x_0) \colon  [w] \in p_*(\pi_1(E(H),e_0)) \iff  \tilde{w} \text{ ist eine Schleife an $e_0$}
    .\] 
    $\tilde{w}$ ist gegeben durch  $\tilde{w}(t) = [w_t]_H$. Also ist $\tilde{w}(1) = [w]_H$. Es ist zudem $[w]_H = [c_{x_0}] = e_0$ genau dann wenn $[w]\in H$, d.h. $\tilde{w}$ ist eine Schleife an $e_0$ genau dann, wenn $[w] \in H$. Also folgt zusammen
    \[
        p_*(\pi_1(E(H),e_0)) = H
    .\] 
    wie gewünscht.
\end{proof}

\section{Decktransformationen}
\begin{restatable}[Decktransformation]{lemmadef}{decktransformation}\label{def:decktransformation}
    Sei $p\colon  E \to  X$ eine Überlagerung. Eine \vocab{Decktransformation} von $p$ ist ein Homöomorphismus
     \[
    \begin{tikzcd}[column sep = tiny]
        E \ar{rr}{f}[swap]{\cong} \ar[swap]{dr}{p} & & E \ar{dl}{p} \\
    & X
    \end{tikzcd}
    \]
    über $X$.  Wir bezeichnen mit  $\Delta(p)$ die Menge der Decktransformationen, und diese bilden eine Gruppe.
\end{restatable}

\begin{proof}
    Sind $f,g\in \Delta(p)$ ist $(p \circ  g) \circ  f = p \circ  f = p$, und somit $g \circ  f \in \Delta(p)$. Für $f\in \Delta(p)$ ist auch $f^{-1}\in \Delta(p)$.
\end{proof}
