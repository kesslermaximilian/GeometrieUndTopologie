%! TEX root = ./master.tex
\lecture[]{Do 24 Jun 2021 10:20}{}

\begin{definition*}\label{def:h-äquivalent-für-untegruppen}
    Sei $H\subset \pi_1(X,x_0)$ eine Untergruppe. Zwei Wege $w,w'$ in  $X$ mit  $w(0) = w'(0) = x_0$ heißen \vocab{$H$-äquivalent}, falls $w(1) = w'(1)$ und  $[w \star \overline{w}']\in H$. 
\end{definition*}

\begin{oral}
Das ist keine gängige Notation, und dient einfach nur, den Beweis besser aufschreiben zu können.    
\end{oral}

\begin{lemma}\label{lm:h-äquivalent-ist-äquivalenzrelation}
    Dies definiert eine Äquivalenzrelation auf
    \[
        \left \{w\colon  I \to  X \mid w(0) = x_0\right\} 
    .\] 
\end{lemma}

\begin{notation*}
    Wir notieren $[w]_H$ für die korrespondierenden Äquivalenzklassen.
\end{notation*}

\begin{proof}
    Reflexivität ist klar, weil $[w \star \overline{w}] = [c_{x_0}] \in H$.

    Für Symmetrie benötigen wir nur, dass aus $g\in H$ auch $g^{-1}\in H$ folgt.

    Für Transitivität benötigen wir, dass $g,g'\in H$ auch $gg'\in H$ impliziert, denn so ist
    \[
        [\underbrace{w \star \overline{w'}}_{\in H} \star \underbrace{w' \star \overline{w''}}_{\in H}] = [ w \star \overline{w''}]
    .\] 
\end{proof}

Wir definieren
\[
    E(H) \coloneqq  \faktor{\left \{w\colon  I \to  X \mid  w(0) = x_0\right\} }{\text{H-Äquivalenz}}
.\] 

und $p\colon E(H) \to  X$ durch $p([w]_H) = w(1)$.

 \begin{example}
     Ist $H = \left \{\star\right\} $ trivial, so sind $w$ und  $w'$  $H$-äquivalent genau dann, wenn  $w(1) = w'(1)$ und  $w \simeq w'$ relativ Endpunkten.

     Ist  $H = \pi_1(X,x_0)$, so sind $w$ und  $w'$  $H$-äquivalent genau dann, wenn  $w(1) = w'(1)$.
\end{example}

\begin{lemma}\label{lm:basis-vonsemilokal-einfachzusammenhängendem-zusammenhängendem-raum}
    Sei $X$ semilokal einfachzusammenhängend, lokal wegzusammenhängend und wegzusammenhängend. Dann ist
     \[
         \mathcal{B} \coloneqq  \left \{U\subset X \text{ offen}\mid  U \text{ ist wegzusammenhängend}, \pi_1(U) \to  \pi_1(X) \text{ ist trivial}\right\} 
    .\] 
    eine Basis der Topologie.
\end{lemma}


\begin{proof}
    Sei $V\subset X$ offen, $x\in V$. Es ist zu zeigen, dass es ein $U\in \mathcal{B}$ mit $x\in U\subset V$ gibt. 

    Weil $X$ semilokal einfachzusammenhängend ist, gibt es eine Umgebung  $U'$ von  $x$ mit  $U'$ wegzusammenhängend und $\pi_1(U') \to  \pi_1(X)$ trivial sowie $U'\subset V$. Da $U'$ Umgebung ist, existiert ein  $U''\subset X$ offen mit $x\in U'' \subset U'$, und weil $X$ lokal wegzusammenhängend und  $U''$ offen ist auch  $U''$ lokal wegzusammenhängend. Die Wegekomponenten von  $U''$ sind nun offen in  $U''$ (nach  \autoref{thm:wegkomponenten-in-lokal-wegzusammenhängendem-raum-sind-offen}) und damit auch in $X$. 

    Sei $U$ die Wegekomponente von  $U''$, die  $x$ enthält. Dann ist  $U\subset X$ offen, wegzusammenhängend, $x\in U$, $U\subset U''\subset U'\subset V$, und
    \[
        \pi_1(U,x) \to  \pi_1(U',x) \stackrel{g \mapsto 0}{\longrightarrow}   \pi_1(X,x)
    .\] 
    ist somit auch die triviale Abbildung. Also ist $U''$ ein Basiselement mit $x\in U''\subset V$.
\end{proof}
\todo{basispunkte der gruppen}

Sei $w\colon I\to x$         mit $w(0) = x_0$, $U\in \mathcal{B}$ mit $w(1) \in U$, so definiere
\[
    \mathcal{B}([w]_H,U) = \left \{[v]_H \in E(H) \mid v = w \star u, u\colon I\to U\right\}  
.\]

\begin{recap}
    Das hängt wirklich nur von $[w]_H$ ab, nicht von der Wahl des Vetreters. Ist nämlich  $w \simeq_H w'$, so auch  $w \star u \simeq_H w' \star u$ wegen
     \[
    w \star u \star \overline{u} \star \overline{w'} \simeq w \star \overline{w'} \simeq_H c_{x_0}
    .\] 
\end{recap}

\begin{lemma}
    Sei $[w]_H\in \mathcal{B}([w]_H,U)$. Dann ist
    \[
        \mathcal{B}([v]_H,U) = \mathcal{B}([w]_H,U)
    .\] 
\end{lemma}

\begin{proof}
    '$\supseteq$'. Schreibe $v = w \star u$ mit  $u\colon  I \to  U$ nach Voraussetzung. Sei $[\tilde{v}]_H \in \mathcal{B}([w]_H,U)$ mit $\tilde{v} = w \star \tilde{u}$ mit $\tilde{u}\colon  I \to  U$ beliebig.

    Dann ist
    \[
        \tilde{v} = w \star \tilde{u} \simeq w \star u \star \overline{u} \star \tilde{u} = v \star (\underbrace{\overline{u} \star \tilde{u}}_{\text{Weg in $U$}})
    .\] 
    Also $[\tilde{v}]_H = [v \star (\overline{u} \star \tilde{u})]_H \in \mathcal{B}([v]_H, U)$.

    Insbesondere ist somit $[w]_H \in  \mathcal{B}([v]_H, U)$. Die Inklusion '$\subset $' folgt also analog.
\end{proof}

\begin{proposition}
    Die Mengen $\mathcal{B}([w]_H, U)$ bilden eine Basis einer Topologie auf $E(H)$.
\end{proposition}

\begin{proof}
    Seien $B([w]_H, U)$ und  $B([w']_H, U')$ zwei solche offenen Mengen, und wähle  $[v]_H \in \mathcal{B}([w]_H, U) \cap \mathcal{B}([w']_H, U')$. Nach Lemma 18.6 ist dann
    \[
        \mathcal{B}([w]_H, U) = \mathcal{B}([v]_H, U), \qquad \mathcal{B}([w']_H, U') = \mathcal{B}([v]_H, U')
    .\] 
    Sei $U'' \subset U \cap  U'$, $U'' \in \mathcal{B}$ und $v(1) \in U''$. Dann ist
    \[
        \mathcal{B}([v]_H, U'') \subset \mathcal{B}([w]_H, U) \cap \mathcal{B}([w']_H, U')
    .\] 
    und enthält immer noch $[v]_H$.
\end{proof}

Wir statten $E(H)$ nun mit dieser Topologie aus.

 \begin{corollary}
     Mit dieser Topologie ist die Abbildung $p\colon  E(H) \to  X$ stetig.
\end{corollary}


\begin{proof}
    Es ist zu zeigen, dass für $U\in \mathcal{B}$ auch $p^{-1} (U)\subset E(H)$ offen ist. Es ist
    \[
        p^{-1} (U) = \left \{[w]_H \in E(H) \mid  w(1) \in U\right\} = \bigcup_{[w]_H, w(1) \in U}  \mathcal{B}([w]_H, U)
    .\] 
\end{proof}
