%! TEX root = ./master.tex
\lecture[Metrisierbarkeit des Hilbertraums. Beweis des Metrisierungssatzes von Urysohn. Motivation für die Einführung von Algebraischer Topologie sowie Kategiorientheorie. Mengen und Klassen. Anfänge der Kategorientheorie: Kategorien, Unterkategorien, Isomorphismen, Funktoren.]{Do 20 Mai 2021 10:07}{Kategorientheorie}

\begin{lemma}[Hilbert-Raum ist metrisierbar]\label{lm:hilbert-raum-ist-metrisierbar}
    Der Raum $\prod_{i=1}^{\infty}[0,1]$ ist metrisierbar (in der Produkttopologie).
\end{lemma}

\begin{proof}
    Übung. Die Metrik ist hierbei gegeben durch:
    \[
        D((x_n)_{n\in \N}, (y_n)_{n\in \N}) = \sup \left \{\frac{\abs{x_n-y_n}}{n}\mid n\in \N \right\} 
    .\] 
\end{proof}

\begin{lemma}\label{lm:abzählbare-trennungsfamilie-in-normalem-raum-mit-abzählbarer-basis}
   Sei $X$ ein normaler Raum mit abzählbarer Basis
   \[
   \mathcal{B} = \left \{B_1,B_2,\ldots\right\} 
   .\] 
   Dann gibt es eine abzählbare Familie
   \[
       \left \{f_i \colon X \to  [0,1] \mid  f_i \text{ stetig}\right\} 
   .\] 
   sodass für jedes $x\in X$ und jede offene Umgebung $x\in U$ ein $i\in \N$ existiert, sodass $f_i(x) = 1$ und  $f_i(y) = 0$ für  $y\not\in U$.
\end{lemma}

\begin{remark}
    Wir wissen schon, dass $X$ normal  $\implies$ $X$ vollständig regulär, dass wir also solche Funktionen finden, ist bereits klar. Das wichtige am Beweis ist, dass wir abzählbar viele Funktionen finden können, die das schon für alle (!) Punkte tun.
\end{remark}

\begin{proof}[Beweis von \autoref{lm:abzählbare-trennungsfamilie-in-normalem-raum-mit-abzählbarer-basis}]
    Für jedes $n,m$ mit  $\abs{B_n} \subset B_m$ wenden wir das \nameref{thm:urysohn}.
    an, also gibt es Funktionen
    \begin{IEEEeqnarray*}{rCl}
        g_{n,m}\colon X & \to  & [0,1] \\
        g_{n,m} (\overline{B_n}) &=& \left \{1\right\}  \\
        g_{n,m}(X \setminus B_m) & = & \left \{0\right\} 
    \end{IEEEeqnarray*}
    Wir stellen zudem fest, dass diese Familie von Funktionen abzählbar ist, wegen $\N\times \N \cong \N$.
    \begin{claim}
        Die $(g_{n,m})_{n,m\in \N}$ erfüllen bereits die gewünschte Bedingung.
    \end{claim}
    \begin{subproof}
        Sei $x\in X$ mit einer Umgebung $x\in U$ gegeben. Da $\mathcal{B}$ eine Basis ist, finden wir $m\in \N$ mit $x\in B_m\subset U$, da $U$ offen ist. Da $X$ normal ist, finden wir zudem eine offene Menge  $V$ mit  $x\in V \subset \overline{V} \subset B_m$ (\autoref{trennung-von-mengen-in-normalem-raum-für-urysohn-lemma}, wir erinnern uns, dass Punkte in normalen Räumen abgeschlossen sind nach \autoref{thm:hausdorff-impliziert-t1}). Analog finden wir nun $B_n\in \mathcal{B}$ mit $x\in B_n \subset V$, erneut, weil $\mathcal{B}$ eine Basis ist. \\
        Dann ist $\overline{B_n}\subset \overline{V}\subset B_m$, und $g_{n,m}(x) =1$ wegen $x\in B_n \subset \overline{B_n}$ und $g_{n,m}(y) = 0$ für $y\not\in U$, da dann $y\not\in B_m$.
    \end{subproof}
\end{proof}

\begin{doral}
    Für den Beweis von \autoref{thm:metrisierungsssatz-von-urysohn} brauchen wir nicht wirklich, dass wir eine abzählbare Basis finden, sondern es genügt die Eigenschaft ebigen Lemmas. Die abzählbare ist jedoch die einfachste Eigenschaft das zu garantieren.
\end{doral}

\begin{proof}[Beweis des \nameref{thm:metrisierungsssatz-von-urysohn}]
    Seien $(f_i \colon X \to  [0,1])_{i\in \N}$ wie in \autoref{lm:abzählbare-trennungsfamilie-in-normalem-raum-mit-abzählbarer-basis}. Definiere
        \begin{equation*}
        F: \left| \begin{array}{c c l} 
            X & \longrightarrow & \prod_{i=1}^{\infty}[0,1] \\
            x & \longmapsto &  (f_i(x))_{i\in \N}
        \end{array} \right.
    \end{equation*}
    Nach der universellen Eigenschaft der Produkttopologie ist $f$ stetig.
     \begin{claim}
         $F$ ist eine Einbettung (d.h. ein Homöomorphismus mit dem Bild, siehe \autoref{def:einbettung}).
    \end{claim}
    \begin{subproof}
        Wir zeigen, dass $F$ injektiv und $F\colon X\to F(X)$ offen ist, dann ist  $F$ eine Einbettung.
        \begin{itemize}
            \item Seien $x\neq y\in X$. Da $X$ normal ist, finden wir eine offene Menge $x\in U$, $y\not\in U$ (erneut, indem wir uns erinnern, dass normale Räume Hausdorff sind, und dann \autoref{thm:hausdorff-impliziert-t1} anwenden). Wegen \autoref{lm:abzählbare-trennungsfamilie-in-normalem-raum-mit-abzählbarer-basis} gibt es also $n\in \N$ mit $f_n(x) \subset f_n(U) = 1$ und  $f_n(X\setminus U) = 0$, also 
                \[
                    f_n(x) = 1 \neq  f_n(y) \implies F(x) \neq  F(y)
                .\] 
                Also ist $F$ injektiv.
            \item Sei $U\subset X$ offen. Wir zeigen: $F(U)\subset \prod_{\N}$ ist offen. Sei $z\in F(U)$ mit (eindeutigem) Urbild $x\in U$. Wir konstruieren eine Menge $V\subset \prod_{\N}[0,1]$ offen, sodass $z\in V \cap F(X)\subset F(U)$, dann ist $F(U)$ offen in  $F(X)$. \\
                Erneut nach \autoref{lm:abzählbare-trennungsfamilie-in-normalem-raum-mit-abzählbarer-basis} erhalten wir ein $n$ mit $f_n(x) = 1$ und  $f_n (X\setminus U) = 0$. Setze nun
                \[
                    V = [0,1] \times  \ldots \times  (0,1] \times [0,1] \times \ldots
                .\] 
                als offene Teilmenge von $\prod_{\N}[0,1]$, wobei $(0,1]$ im  $n$-ten Faktor stehe.
                \begin{claim}
                    $z\in V\cap F(X) \subset F(U)$ ist eine offene Umgebung (in $F(X)$) von  $z$.
                \end{claim}
                \begin{subproof}
                    Sei $z' = F(x')\in V\cap F(X)$. Es ist $z'\in V$, also $z_n' \coloneqq  f_n(x') \neq 0$, allerdings wissen wir auch $f_n(X\setminus U) = 0$, d.h. $x' \not\in X\setminus U$, also folgt $x'\in U$ und somit $z' = F(x') \in F(U)$. Zudem ist wegen $z_n = f_n(x)=1 \neq 0$ auch $z \in V \cap F(X)$, also handelt es sich um eine offene Umgebung von $z$.
                \end{subproof}
                Also ist $F(U)$ offen in  $F(X)$ und somit  $F\colon X \to  F(X)$ offen.
        \end{itemize}
        Also ist $F\colon X \to  F(X)$ offen und injektiv, und somit eine Einbettung.
    \end{subproof}
    Nun stellen wir also fest, dass $X \cong F(X)$ (wegen der Einbettung), aber  $F(X) \subset  \prod_{\N}[0,1]$ ist metrisierbar als Teilraum eines metrisierbaren Raums, also ist $X$ metrisierbar und das wollten wir zeigen.
\end{proof}

\begin{oral}
    Wo haben wir jetzt wirklich benutzt, dass das Produkt abzählbar war?. Man überlegt sich, dass wir den exakt gleichen Beweis für jede Kardinalität einer Basis hätten durchführen können, um nach $\prod_{\aleph} [0,1]$ einzubetten. Das wirkliche Problem ergibt sich dann erst, wenn wir zeigen (in der Übung), dass $\prod_{\N}[0,1]$ metrisierbar ist. Es stellt sich heraus, dass das nur für $\aleph\leq \omega $, dh. für abzählbare Indexmengen der Fall ist.
\end{oral}

\newpage
\part{Algebraische Topologie}

\section*{Motivation}\addcontentsline{toc}{section}{Motivation}

Bisher haben wir Topologische Räume und ihre Eigenschaften wie Hausdorff, normal, Kompakt oder zsuammenhängend gesehen, um diese zu unterscheiden. Im 2. Teil der Vorlesung kümmern wir uns nun um weiter Topologische Invarianten.
\begin{example}
    Setze $\pi_0(X) \coloneqq $ als die Menge der Wegkomponenten von  $X$, d.h.  $\pi_0(X) \subset \mathcal{P}(X)$ und $U\in \pi_0(X)$ genau dann, wenn $U$ wegzusammenhängend und inklusionsmaximal, d.h. $\not \exists \,V$ wegzusammenhängend mit $U\subsetneq  V$. Dann ist $\pi_0(X)$ eine topologischen Invariante.
\end{example}

\begin{remark*}
    Mit 'topologischen Invariante' meinen wir natürlich immer eine Eigenschaft eines topologischen Raumes, die von Homöomorphismen erhalten wird, also nicht von der konkreten Wahl des Raumes abhängt.
\end{remark*}

\begin{example}
    \begin{itemize}
        \item $\pi_0(\R) = \left \{\R\right\}$ , da $\R$ wegzusammenhängend
        \item $\pi_0(\N) = \left \{\left \{n\right\} \mid n\in \N\right\}$, weil die einzigen wegzusammenhängenden Teilmengen von $\N$ die einpunktigen Mengen sind
        \item Betrachte die Sinuskurve des Topologen (vgl. \autoref{aufgabe-5.1}). Diese ist definiert als der Abschluss des Graphen $G$ von $x \mapsto \sin  \frac{1}{x}$ für $x>0$. Dann sind die Wegzusammenhangskomponenten genau $G$ selbst (blau) sowie  $\overline{G} \setminus G$ (rot). \\
            \begin{minipage}{\textwidth}
            \centering
            \begin{tikzpicture}[domain=0.001:1, xscale = 6]
                \draw[color=blue!30!white,smooth,samples=100,domain=0.001:0.01,line width = 0.1pt] plot[id=gnuplots/topologists-sine-curve-1] function{sin(1/x)};
                \draw[color=blue!30!white,smooth,samples=1000,domain=0.01:0.1, line width = 0.1pt] plot[id=gnuplots/topologists-sine-curve-2] function{sin(1/x)};
                \draw[color=blue!30!white,smooth,samples=100,domain=0.1:1, line width = 0.1pt] plot[id=gnuplots/topologists-sine-curve-3] function{sin(1/x)};
                \draw[color=red,thick] (0,-1) -- (0,1);
                \draw[->] (0,0) -- (1,0);
                \foreach \x in {1,2,3,4,5,6,7,8,9} {
                    \draw (0.1*\x,-0.1) node[anchor=north]{0,\x} -- (0.1*\x, 0.1);
                }
                \draw (-0.01,-1) node[anchor = east] {-1} -- (0.01,-1);
                \draw (-0.01,1) node[anchor = east] {1} -- (0.01,1);
            \end{tikzpicture}
            \captionof{figure}{Sinuskurve des Topologen}
            \end{minipage}
    \end{itemize}
\end{example}

\begin{example}
    Eine Invariante, die wir noch nicht kennen, ist $\pi_1$. Hierzu definiere für $x_0\in X$:
    \[
        \pi_1(X,x_0) = \left \{\text{Abbildungen } f\colon S^1 \to  X, 1 \mapsto x_0\right\} / \text{'Verschieben'}
    .\] 
\end{example}

\begin{oral}
    'Verschieben' ist an dieser Stelle (bewusst) noch nicht präzisiert. Wir werden sehen, dass wir damit 'Homotopie' meinen, dazu aber später mehr, wenn wir das ganze detailliert behandeln.
\end{oral}

\begin{fact}    
Ist $f: X \to Y$ stetig, so induziert $f$ Abbildungen
 \begin{IEEEeqnarray*}{rCl}
     f_*: \pi_0(X) &\to&  \pi_0(Y) \\
     f_*: p_1(X,x_0) & \to  & \pi_1(Y,f(x_0))
\end{IEEEeqnarray*}
\end{fact}

Formal sind $\pi_0, \pi_1$ sogenannte \vocab[Kategorie!Funktor]{Funktoren}, deswegen wollen wir uns im Folgenden etwas genauer die sogenannte \vocab{Kategorientheorie} ansehen, die solche Konzepte behandelt.


\section{Kategorien}
\subsection{Einschub: Mengentheorie}

\begin{remark*}
    Das folgende Kapitel ist sehr formal und holt weit aus, was wir nicht (wirklich) verwenden. Es dient nur dazu, unserem folgenden Handeln eine formale Grundlage zu verleihen, hat aber keinen (wirklichen) weiteren Einfluss auf die Vorlesung. \\
    Eine Einführung in die Logik und Mengenlehre, die auf die Topologie hinarbeitet, findet sich in \cite[Kapitel 1]{point-set-topology}. \\
    Für Formaleres zu Kardinalitäten sei auf auf das Vorlesungsskript \cite{set-theory} verwiesen, dort wird allerdings auch vieles behandelt, das hier nicht relevant ist.
\end{remark*}

Wir fordern neben den üblichen Axiomen von \textbf{ZFC} (hierbei steht \textbf{C} für das sogenannte \vocab{Auswahlaxiom}), noch die Existenz mehrerer unerreichbarer Kardinalzahlen (d.h. welche außer $\aleph_0 \coloneqq  \card(\N)$, die wir $\aleph_0<κ<κ'$ nennen)

\begin{ddefinition}[unerreichbare Kardinalzahl]\label{def:unerreichbare-kardinalzahl}
    $κ$ ist eine \vocab[Kardinalzahl!unerreichbare]{unerreichbare Kardinalzahl}, falls 
     \begin{itemize}
         \item $\card\left( \bigcup_{i \in  I} X_i \right) <κ$ für alle $I,X_i$ mit  $\card(I), \card(X_i) < κ$.
         \item  $\card \left \{f\colon X \to Y \mid  f \text{ Abbildung}\right\} <κ$ für alle Mengen $X,Y$ mit  $\card(X), \card(Y) < κ$.
    \end{itemize}
\end{ddefinition}

\begin{remark*}[Logik-Spam]
    Eine überabzählbare unerreichbare Kardinalzahl liefert uns zugleich ein Modell von \textbf{ZFC}, wir können deren Existenz also nicht innerhalb von \textbf{ZFC} zeigen (vgl. Gödelscher Unvollständigkeitssatz), deren Existenz ist jedoch konsistent genau dann, wenn \textbf{ZFC} selbst Widerspruchsfrei ist (wovon wir ausgehen). Insbesondere sollte man über  $κ$ so nachdenken, dass alles, das man definieren kann / betrachtet, bzw. alle 'interessanten' Mengen  $<κ$ sind. So sollte man auch über  $κ$ nachdenken: Durch keine Begriffsbildungen, die Dinge  $<κ$ verwenden, können wir  $κ$ erreichen, also bildet  $κ$ etwas wie den Horizont des Universums (der Mengen).
\end{remark*}

\begin{definition}[Menge,Klasse]\label{def:menge-klasse}
    \begin{itemize}
        \item 
            Der Begriff \vocab[Menge]{Mengen} heißt für uns ab nun \textit{alle Mengen mit Kardinalität $<κ$}. 
   \item Der Begriff \vocab{Klasse} steht für alle Mengen mit Kardinalität $<κ'$. 
    \end{itemize}
\end{definition}

\begin{remark*}
    Diese Begrifflichkeiten dienen nur dazu, dass wir über die Klasse aller Mengen $V$ reden können, die keine Menge ist. (auch nicht im herkömmlichen Sinne). Es gibt auch andere Ansätze, um das zu ermöglichen, wie etwa das Beschreiben von Klassen mittels Formeln, für uns ist obiger Ansatz jedoch am einfachsten. Merken sollte man sich vor allem
    \begin{itemize}
        \item Jede Menge ist eine Klasse, aber nicht zwingend umgekehrt. $V = \left \{M \mid  M \text{ ist Menge}\right\} $ ist die Klasse aller Mengen, oder auch das Universum aller Mengen.
        \item Wir können Mengen beliebig zu einer Klasse zusammenfassen, d.h. ist $\varphi (M)$ eine Formel (Eigenschaft) einer Menge $M$, so ist
             \[
                 \left \{M \mid  \varphi (M)\right\} \coloneqq  \left \{M  \mid  \varphi (M) , M \text{ ist Menge}\right\} 
            .\] 
            eine Klasse.
        \item Für Klassen gilt das nicht mehr, d.h. 
            \[
            \left \{K \mid K \text{ ist Klasse}\right\} 
            .\] 
            ist \underline{keine} Klasse (und damit [für uns]) kein definierter Ausdruck. Sonst könnten wir das \href{https://en.wikipedia.org/wiki/Russell%27s_paradox}{Russel'sche Paradoxon} herleiten.
    \end{itemize}
\end{remark*}


\subsection{Kategorien}

\begin{remark*}
    Für eine ausführlichere Einführung zur Kategorientheorie siehe z.B. \cite{category-theory}.
\end{remark*}

\begin{definition}[Kategorie]\label{def:kategorie}
    Eine \vocab{Kategorie} $\cat{C}$ besteht aus
    \begin{itemize}
        \item Einer Klasse von \vocab[Kategorie!Objekt]{Objekten}, notiert $\Ob(\cat{C})$.
        \item $\forall X,Y \in \Ob(\cat{C})$ eine Menge $\Mor_{\cat{C}}(X,Y)$ von \vocab[Kategorie!Morphismus]{Morphismen}
        \item Für $X,Y,Z \in \Ob(\cat{C})$ Verknüpfungsabbildungen
                \begin{equation*}
                \circ : \left| \begin{array}{c c l} 
                    \Mor_{\cat{C}}(X,Y)\times \Mor_{\cat{C}}(Y,Z) & \longrightarrow & \Mor_{\cat{C}}(X,Z) \\
                    (f,g) & \longmapsto &  g\circ f
                \end{array} \right.
            \end{equation*}
            mit $(f,g) \mapsto g\circ f$, sodass $\circ $ assoziativ ist.
        \item Jede Menge $\Mor_{\cat{C}}(X,X)$ enthält eine Identität $\id_{X}$, sodass 
            \[
                f \circ  \id_{X} = f \qquad \id_X \circ  g = g
            .\] 
            für $Y\in \Ob(\cat{C}), f\in \Mor_{\cat{C}}(X,Y)$ und $g\in \Mor_{\cat{C}}(Y,X)$ beliebig. 
    \end{itemize}
\end{definition}

\begin{abuse*}
    Wir schreiben $X\in \cat{C}$ für $X\in \Ob(\cat{C})$.
\end{abuse*}

\begin{notation**}
    Aus naheliegenden Gründen notieren wir $f: X \to Y$ für $f\in \Mor_{\cat{C}}(X,Y)$
\end{notation**}

\begin{remark*}[Assoziativität und kommutative Diagramme]
    Mit Assoziativität meinen wir das folgende: Sind $X,Y,Z,W\in \Ob(\cat{C})$ und $f\in \Mor_{\cat{C}}(X,Y)$, $g\in \Mor_{\cat{C}}(Y,Z), h\in \Mor_{\cat{C}}(Z,W)$, so ist $h \circ  (g \circ f) = (h \circ  g) \circ  f$. Wir veranschaulichen dies in einem \vocab[Kommutatives Diagramm]{kommutativen Diagramm} (das ist typische für die Kategorientheorie, wir malen Objekte als Punkte und Elemente von $\Mor_{\cat{C}}(X,Y)$ als Pfeile von $X\to Y$, so sollte man sich das vorstellen):
    \[
    \begin{tikzcd}[column sep = 4em, row sep = 4em]
        X \ar[bend left = 60]{rrr}{h \circ  (g \circ  f)} \ar[bend right = 60, swap]{rrr}{(h \circ  g) \circ f}\ar[bend left = 40,swap]{rr}{g \circ  f}\ar{r}{f} & Y\ar[bend right = 40]{rr}{h \circ  g} \ar{r}{g} & Z \ar{r}{h} & W
    \end{tikzcd}
    .\] 
    Wir fordern also, dass beide Möglichkeiten, sich eine Abbildung $X\to Z$ zusammenzubauen, die gleichen sind.
\end{remark*}

\begin{remark}
    \begin{itemize}
        \item Ist $\Ob(\mathcal{C})$ eine Menge, so heißt $\cat{C}$ klein.
        \item Da $\Mor_{\cat{C}}(X,Y)$ Mengen sind, heißt $\cat{C}$ in der Literatur manchmal lokal klein, manche Autoren lassen für $\Mor_{\cat{C}}(X,Y)$ auch Klassen zu, wir jedoch nicht.
    \end{itemize}
\end{remark}

\begin{example}
    \begin{itemize}
        \item $\category{Set}$ ist die Kategorie der Mengen und all ihrer Abbildungen dazwischen.
        \item $\Top X$ ist die Kategorie der topologischen Räume und ihren stetigen Abbildungen.
        \item $\Grp$ ist die Kategorie der Gruppen und ihren Gruppenhomomorphismen
        \item  $\Vect_{\R}$ ist die Kategorie der $\R$-Vektorräume und den linearen Abbildungen dazwischen.
        \item $\Top_{\star}$ ist die Kategorie der punktierten topologischen Räume. Wir setzen
            \[
                \Ob(\Top_{\star}) = \left \{(X,x_0) \mid X \text{ topologischer Raum, } x_0\in X\right\} 
            .\] 
            d.h. Objekte sind topologische Raume mit einem ausgezeichneten \vocab{Basispunkt} $x_0\in X$. Von den Morphismen fordern wir, dass sie diesen Basispunkt erhalten, d.h.
            \[
                \Mor_{\Top_{\star }}((X,x_0),(Y,y_0)) \coloneqq  \left \{f \colon X \to  Y \text{ stetig}\mid  f(x_0) = y_0\right\} 
            .\] 
    \end{itemize}
\end{example}

\begin{dremark}
    Die Kategorientheorie bildet zunächst eine Sprache, mit der wir sehr vieles präziser ausdrücken können. Wir sollten auch so über sie nachdenken, d.h. die Kategorientheorie hilft uns, Dinge aus vielen verschiedenen Teildisziplinen (siehe Liste der Beispiele oben) elegant und knapp zusammenzufassen und Beweise, die gleich geführt werden, zu vereinheitlichen.
\end{dremark}

\begin{definition}[Unterkategorie]\label{def:unterkategorie}
    \begin{itemize}
        \item $\cat{U}$ ist eine \vocab[Kategorie!Unter-]{Unterkategorie} von $\cat{C}$, falls $\cat{U}$ eine Kategorie ist mit $\Ob(\cat{U}) \subset \Ob(\cat{C})$, und $\forall X,Y\in \Ob(\cat{U})$ ist
    \[
        \Mor_{\cat{U}}(X,Y) \subset \Mor_{\cat{C}}(X,Y)
    .\] 
\item Ist zudem für $X,Y \in \Ob(\cat{U})$ obige Inklusion sogar eine Gleichheit, d.h. $\Mor_{\cat{U}}(X,Y) = \Mor_{\cat{C}}(X,Y)$, so heißt $\cat{U}$ \vocab[Kategorie!Unter-!volle]{volle} Unterkategorie. 
    \end{itemize}
\end{definition}

\begin{example}
    \begin{itemize}
        \item $\Fin \subset \Set$ ist die Unterkategorie der endlichen Mengen (jede endliche Menge ist eine Menge)
        \item $\CHaus \subset \Top$ ist die Unterkategorie der kompakten Hausdorffräume (jeder kompakte Hausdorff-Raum ist ein topologischer Raum)
        \item $\Ab\subset \Grp$ ist die Unterkategorie der abelschen Gruppen (jede abelsche Gruppe ist eine Gruppe)
    \end{itemize}
    Alle 3 Beispiele sind volle Unterkategorien.
\end{example}

\begin{definition}[Isomorphismus]\label{def:isomorphismus}
    $f\in \Mor_{\cat{C}}(X,Y)$ ist ein \vocab[Kategorie!Isomorphismus]{Isomorphismus}, wenn es ein $g\in \Mor_{\cat{C}}(Y,X)$ gibt mit $f \circ  g = \id_Y$ und $g \circ f = \id_X$ (d.h. eine Umkehrabbildung).
\end{definition}

\begin{recap}
    Ist $f\colon X\to Y$ ein Isomorphismus, d.h. es gibt ein Inverses  $g$, so ist  $g$ bereits eindeutig. Seien hierzu  $g,g'$ beides Inverse zu  $f$, dann ergibt sich
    \[
        g = g \circ  \id_{Y} = g \circ  (f \circ  g') = (g \circ  f) \circ  g' = \id_{X} \circ  g' = g' 
    .\] 
\end{recap}

\begin{example}
    $f\in  \Top$ ist ein Isomorphismus, wenn $f$ ein Homöomorphismus ist.
\end{example}

\begin{definition}[Funktor]\label{def:funktor}
    Seien $\cat{C},\cat{D}$ Kategorien. Ein (kovarianter) \vocab[Kategorie!Funktor]{Funktor} $\mathcal{F} \colon \cat{C} \to  \cat{D}$ besteht aus:
    \begin{itemize}
        \item einer Abbildung $\mathcal{F} \colon \Ob(\cat{C}) \to  \Ob(\cat{D})$.
        \item Abbildungen $\mathcal{F}\colon \Mor_{\cat{C}}(X,Y) \to  \Mor_{\cat{D}}(\mathcal{F}(X),\mathcal{F}(Y))$ für alle Objekte $X,Y\in \Ob(\cat{C})$.
    \end{itemize}
    sodass
    \begin{itemize}
        \item $\mathcal{F}(f \circ  g) = \mathcal{F}(f) \circ \mathcal{F}(g)$ 
        \item $\mathcal{F}(\id_X) = \id_{\mathcal{F}(X)}$
    \end{itemize}
\end{definition}

\begin{definition}[Isomorphismus von Kategorien]\label{def:funktor-isomorphismus}
    Ein Funktor $\mathcal{F} \colon \cat{C} \to  \cat{D}$ ist ein \vocab{Isomorphismus}, falls es einen Funktor $\mathcal{G} \colon \cat{D} \to  \cat{C}$ gibt, sodass $\mathcal{G} \circ  \mathcal{F}$ und $\mathcal{F} \circ  \mathcal{G}$ die Identitäten sind.
\end{definition}
\begin{oral}
    Ein Funktor $\mathcal{F}$ ist hierbei die Identität ('der Identitätsfunktor'), wenn er sowohl Objekte als auch Abbildungen auf sich selbst schickt.
\end{oral}

\begin{doral}
    Im Gegensatz zu den Begrifflichkeiten ist ein Isomorphismus von Kategorien nicht die gängigste Version von 'Gleichheit'. Nur in Seltenen Fällen sind Kategorien tatsächlich isomorph (nach ebiger Definition), allerdings oft \textit{äquivalent}. Mehr dazu später.
\end{doral}

\begin{example}
    \begin{itemize}
        \item Es gibt einen Funktor $\Top \to  \Set$, der jeden topologischen Raum auf seine Trägermenge sendet, und (stetige) Abbildungen zwischen den topologischen Räumen auf die zugehörigen Abbildungen zwischen den Trägermengen. Dieser Funktor heißt oft \vocab[Kategorie!Funktor!vergesslicher]{vergesslicher Funktor}, auch wenn das kein Fachbegriff ist, sondern eher ein gängiges Schema. Wir können im wesentlichen alle möglichen Strukturen vergessen. 
        \item $\Ab \to  \Grp$ als Inklusion ist ein Funktor, da jede abelsche Gruppe auch Gruppe ist, und Gruppenhomomorphismen von abelschen Gruppen natürlich auch Gruppenhomomorphismen.
        \item Betrachte den Funktor
                \begin{equation*}
                \mathcal{F}: \left| \begin{array}{c c l} 
                \Set & \longrightarrow & \Ab \\
                X & \longmapsto &  \Z[X] \\
                f: X \to  Y &\longmapsto & 
                    \left| \begin{array}{c c l} 
                        \Z[X] & \longrightarrow & \Z[Y] \\
                        \sum\limits_{i=1}^k n_i x_i& \longmapsto	 & \sum\limits_{i=1}^k n_i f(x_i)
                \end{array} \right.
                
                \end{array} \right.
            \end{equation*}
            Wir senden hierbei eine Menge $X$ auf die \vocab{freie abelsche Gruppe} , die von $X$ generiert wird. Dabei ist $\Z[X]$ das (freie) $\Z$-Modul, das als Basis  von $X$ erzeugt wird.
    \end{itemize}
\end{example}
