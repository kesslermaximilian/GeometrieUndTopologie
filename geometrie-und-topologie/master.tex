\makeatletter
% patched command of loading a package
\def\find@load#1[#2]#3[#4]{%
    % load the package
    \find@fileswith@pti@ns{#1}[#2]{#3}[#4]%
    % check if commands exists now
    \ifcsname\find@command\endcsname
        \typeout{Package #3 introduces command \find@command.}%
        % command has been found, revert to original version without checks
        \let\@fileswith@pti@ns\find@fileswith@pti@ns
    \else
        % somehow, \@fileswith@pti@ns is restored after loading a package
        % thus, patch it again
        \let\@fileswith@pti@ns\find@load
    \fi
}%

\newcommand*{\findpackagebycommand}[1]{%
    % using this multiple times - esp. when the package has not been found yet -
    % will break things. Thus, check first that \find@command has never been defined before
    \ifx\find@command\undefined
        \def\find@command{#1}%
        % first, check if this command is already defined
        \ifcsname\find@command\endcsname
            % in this case, just issue a warning and do nothing
            \@latex@warning@no@line{Command \find@command \space is already defined}%
        \else
            % overwrite the internal \@fileswith@pti@ns command, which does the actual loading
            % \@fileswith@pti@ns is used internally by \usepackage and \RequirePackage
            \let\find@fileswith@pti@ns\@fileswith@pti@ns
            \let\@fileswith@pti@ns\find@load
        \fi
    \else
        % used multiple times - prevent and give a warning
        \@latex@warning@no@line{You can use \protect\findpackagebycommand \space only once}%
        \errmessage{Invalid use of command findpackagebycommand.}%
    \fi
}

\makeatother


\findpackagebycommand{aside}



\documentclass[a4paper, german, lecturenumbers = true, number small environments = theorem, hide version]{mkessler-script}

\course{Einführung in die Geometrie und Topologie}
\lecturer{Daniel Kasprowski}
\assistant[f]{Arunima Ray}
\author{Maximilian Keßler}

\RequirePackage{mkessler-math}
\RequirePackage{mkessler-fancythm}
\usepackage{epsfig}
%\usepackage{psfrag}
%\usepackage{sseq} (if you need to draw spectral sequences, please use this package, available at http://wwwmath.uni-muenster.de/u/tbauer/)
\usepackage{mathrsfs}
\usepackage{amscd}
\usepackage{amsbsy}
\usepackage{verbatim}
\usepackage{moreverb}

\newtheorem{prop}[theorem]{Proposition}
\newtheorem{cor}[theorem]{Corollary}
\newtheorem{conj}[theorem]{Conjecture}


\theoremstyle{definition}
\newtheorem{hw}{Homework}
\newtheorem{exercise*}[exercise]{$\star$ Exercise}

\theoremstyle{remark}
\newtheorem{aside}[theorem]{Aside}

\newcommand{\nn}{\nonumber}
\newcommand{\nid}{\noindent}
\newcommand{\ra}{\rightarrow}
\newcommand{\la}{\leftarrow}
\newcommand{\xra}{\xrightarrow}
\newcommand{\xla}{\xleftarrow}
\newcommand{\tto}{\longrightarrow}

\newcommand{\weq}{\xrightarrow{\sim}}
\newcommand{\cofib}{\rightarrowtail}
\newcommand{\fib}{\twoheadrightarrow}

\newcommand{\IRep}{\mathrm{IRep}}
\newcommand{\IHom}{\mathrm{IHom}}

\def\llarrow{   \hspace{.05cm}\mbox{\,\put(0,-2){$\leftarrow$}\put(0,2){$\leftarrow$}\hspace{.45cm}}}
\def\rrarrow{   \hspace{.05cm}\mbox{\,\put(0,-2){$\rightarrow$}\put(0,2){$\rightarrow$}\hspace{.45cm}}}
\def\lllarrow{  \hspace{.05cm}\mbox{\,\put(0,-3){$\leftarrow$}\put(0,1){$\leftarrow$}\put(0,5){$\leftarrow$}\hspace{.45cm}}}
\def\rrrarrow{  \hspace{.05cm}\mbox{\,\put(0,-3){$\rightarrow$}\put(0,1){$\rightarrow$}\put(0,5){$\rightarrow$}\hspace{.45cm}}}

\def\cA{\mathcal A}\def\cB{\mathcal B}\def\cC{\mathcal C}\def\cD{\mathcal D}
\def\cE{\mathcal E}\def\cF{\mathcal F}\def\cG{\mathcal G}\def\cH{\mathcal H}
\def\cI{\mathcal I}\def\cJ{\mathcal J}\def\cK{\mathcal K}\def\cL{\mathcal L}
\def\cM{\mathcal M}\def\cN{\mathcal N}\def\cO{\mathcal O}\def\cP{\mathcal P}
\def\cQ{\mathcal Q}\def\cR{\mathcal R}\def\cS{\mathcal S}\def\cT{\mathcal T}
\def\cU{\mathcal U}\def\cV{\mathcal V}\def\cW{\mathcal W}\def\cX{\mathcal X}
\def\cY{\mathcal Y}\def\cZ{\mathcal Z}

\def\sA{\mathscr A}\def\cB{\mathcal B}\def\cC{\mathcal C}\def\cD{\mathcal D}
\def\cE{\mathcal E}\def\cF{\mathcal F}\def\sG{\mathscr G}\def\cH{\mathcal H}
\def\cI{\mathcal I}\def\cJ{\mathcal J}\def\cK{\mathcal K}\def\cL{\mathcal L}
\def\cM{\mathcal M}\def\cN{\mathcal N}\def\cO{\mathcal O}\def\cP{\mathcal P}
\def\cQ{\mathcal Q}\def\cR{\mathcal R}\def\cS{\mathcal S}\def\cT{\mathcal T}
\def\cU{\mathcal U}\def\cV{\mathcal V}\def\cW{\mathcal W}\def\cX{\mathcal X}
\def\cY{\mathcal Y}\def\cZ{\mathcal Z}

\def\fG{\mathfrak G}\def\fH{\mathfrak H}
\def\fS{\mathfrak S}\def\fN{\mathfrak N}\def\fX{\mathfrak X}\def\fY{\mathfrak Y}

\def\op{\textrm{op}}\def\ob{\textrm{ob}}

%\def\Iso{\mathcal Iso}\def\cInn{\mathcal Inn}

\def\fg{\mathfrak g}\def\fh{\mathfrak h}\def\fri{\mathfrak i}\def\fp{\mathfrak p}
\def\fA{\mathfrak A}\def\fU{\mathfrak U}

\def\AA{\mathbb A}\def\BB{\mathbb B}\def\CC{\mathbb C}\def\DD{\mathbb D}
\def\EE{\mathbb E}\def\FF{\mathbb F}\def\GG{\mathbb G}\def\HH{\mathbb H}
\def\II{\mathbb I}\def\JJ{\mathbb J}\def\KK{\mathbb K}\def\LL{\mathbb L}
\def\MM{\mathbb M}\def\NN{\mathbb N}\def\OO{\mathbb O}\def\PP{\mathbb P}
\def\QQ{\mathbb Q}\def\RR{\mathbb R}\def\SS{\mathbb S}\def\TT{\mathbb T}
\def\UU{\mathbb U}\def\VV{\mathbb V}\def\WW{\mathbb W}\def\XX{\mathbb X}
\def\YY{\mathbb Y}\def\ZZ{\mathbb Z}

\def\TOP{\mathcal{TOP}}\def\GRP{\mathcal{GRP}}\def\GRPD{\mathcal{GRPD}} \def\CAT{\mathcal{CAT}} \def\SET{\mathcal{SET}}

\def\id{\mathrm{id}}\def\Id{\mathrm{Id}}
\def\inverse{^{-1}}



\begin{document}
    \maketitle
    \begin{abstract}
    {\color{red} Achtung:} Diese Version des Skripts benutze ich zur Bearbeitung! Einige Dinge fehlen, dafür gibt es TODO-Notes. Für Inhalte, benutzt die \href{https://kesslermaximilian.github.io/LectureNotesBonn/2021_Topologie.pdf}{normale Version}
    \end{abstract}
    \newpage
    \listoftodos
    \newpage
    \summaryoflectures
    \newpage
    % start lectures
    \setcounter{section}{18}
    \setcounter{dummy}{11}
    \setcounter{smalldummy}{0}
    \setcounter{figure}{28}
    \setcounter{claim}{1}
    \setcounter{lecture}{21}
    %! TEX root = ./master.tex
\lecture[]{Do 08 Jul 2021 10:11}{}

\begin{warning}[Anmerkung d. Editors]
    Die Nummerierung ist hier noch völlig kaputt, weil wir jetzt wieder in Kapitel 20 weitermachen, wo wir letzte Woche Donnerstag aufgehört haben, und Aru aber am Dienstag schon mit Kapitel 21 begonnen hatte. Die nächste Proposition sollte 20.5 sein (die 5 stimmt zufällig), ich muss mir noch überlegen, wie genau ich das löse, bis dahin ist das einfach kaputt.
\end{warning}


Ziel der heutigen Vorlesung ist es, den \nameref{} zu beweisen.

Wir erinnerns uns daran, dass die Wirkung von $\pi_1(X,x_0)$ auf $p^{-1} (x_0)$ durch
\[
    e.[w] \coloneqq  L(w,e)(1)
.\] 
gegeben ist.
\todo{Nummerierung fixen}
\begin{proposition}
    Sei $p\colon E\to X$ eine Überlagerung,  $X$ wegzusammenhängend sowie  $x_0\in X$. Dann induziert die Inklusion $p^{-1} (x_0) \hookrightarrow E$ eine Bijektion
    \[
        \left \{\pi_1(X,x_0) - \text{Bahnen von } p^{-1} (x_0) \right\}  \stackrel{1:1}{\longleftrightarrow} \left \{\text{Wegekomponenten von } E\right\} 
    .\] 
\end{proposition}

\begin{proof}
    \begin{description}
        \item[Wohldefiniertheit] Zu zeigen: Für $e\in p^{-1} (x_0)$ liegen $e$ und  $e.[w]$ in der gleichen Wegekomponente. Es ist aber
             \[
                 e.[w] = L(w,e)(1)
            .\] 
            und damit ist $L(w,e)$ ein Weg von  $e$ nach  $e.[w]$, und somit liegen die Punkte in der gleichen Wegekomponenten von  $E$.
        \item[Injektivität] Seien  $e,e'\in p^{-1} (x_0)$, so dass diese auf die gleiche Wegkomponenten abgebildet werden, dann gibt es einen Weg $v$ von  $e$ nach  $e'$. Dann ist
             \[
                 e.\underbrace{[p \circ  v]}_{\in \pi_1(X,x_0)}  = L(p \circ  v, e)(1) = v(1) = e'
            .\] 
            also liegen $e,e'$ in der gleichen Bahn.
        \item[Surjektivität] Sei  $\tilde{E}$ eine Wegekomponente von $E$ sowie  $e\in \tilde{E}$. Aufgrund des Wegzusammenhangs von $X$ finden wir einen Weg  $v$ von  $p(e)$ nach  $x_0$. Dann ist $L(v,e)$ ein Weg von  $e$ mit Endpunkt in  $p^{-1} (x_0)$, also $\tilde{E} \cap  p^{-1} (x_0) \neq  \emptyset$.
    \end{description}
\end{proof}

Ist $e\in p^{-1} (x_0)$, so ist der Orbit
\[
    e.\pi_1(x,x_0) \cong_{\pi_1(X,x_0)-\text{Menge}} \pi_1(X,x_0)_e \backslash \pi_1(X,x_0)
.\] 

Also interessieren wir uns auch für Elemente aus dem Stabilisator $\pi_1(x,x_0)_e$. Hierzu ist
\[
    e.[w] = e \iff  L(w,e)(1) = e \iff  L(w,e) \text{ ist Schleife an } e \iff  [w] \in  p_*(\pi_1(E,e))
.\] 

Also ist $e.\pi_1(X,x_0) \cong p_*(\pi_1(E,e)) \backslash \pi_1(X,x_0)$.

\begin{proof}[Beweis von \autoref{thm:hauptsatz-der-überlagerungstheorie}]
    \underline{1. Schritt} Wir zeigen die essentielle Surjektivität. Sei $M$ eine  $\pi_1(X,x_0)$-Menge. Dann ist $M$ isomorph zu einer disjunkten Vereinigung
    \[
        M = \bigsqcup_{i\in I} H_i \backslash \pi_1(X,x_0)
    .\] 
    Mit $H_i \leq  \pi_1(X,x_0)$. Nach \autoref{thm:universelle-überlagerungen-existieren-genau-für-semilokal-einfachzusammenhängende-lokal-wegzusammenhängenden-zusammenhängende-räume} existieren Räume $E(H_i)$, sodass
     \[
         p(H_i) \colon  E(H_i) \to  X
    .\] 
    sowie $e_i\in p(H_i)^{-1}(x_0)$ mit 
    \[
        p(H_i)_* \pi_1(E(H_i),e_i) = H_i
    .\] 
\todo{Referenzen}
    Dann ist $p(H_i)^{-1}(x_0)$ isomorph zu $H_i \backslash \pi_1(X,x_0)$ nach ebiger Proposition und der Vorüberlegung. Wir betrachten nun die disjunkte Vereinigung
    \[
        p\coloneqq  \coprod p(H_i) \colon  \coprod _{i \in I} E(H_i) \to  X
    .\] 
    so ist
    \[
        p^{-1} (x_0) = \bigsqcup p(H_i)^{-1}(x_0) \cong \coprod_{i \in I} H_i \backslash \pi_1(X,x_0) \cong M
    .\] 
    \begin{remark}
        Es fehlt noch zu zeigen, dass $p$ überhaupt eine Überlagerung ist, im allgemeinen ist das Koprodukt von Überlagerungen nämlich \textit{nicht} zwingend wieder eine Überlagerung. Wir müssen das also in diesem konkreten Fall noch zeigen.
    \end{remark}
    \begin{claim}
        Sei $x\in X$ und $x\in U\subset X$ eine wegzusammenhängende Umgebung mit $\pi_1(U) \to  \pi_1(X)$ trivial. Dann ist $U$ eine trivialisierende Umgebung für alle  $p(H_i)$ und damit auch für  $p$.
    \end{claim}
    \begin{subproof}
        Folgt unmittelbar aus der Konstruktion, die wir gewählt hatten, denn wir haben gezeigt, dass die trivialisierenden Umgebungen genau diejenigen Basiselement von $X$ aus  \autoref{lm:basis-vonsemilokal-einfachzusammenhängendem-zusammenhängendem-raum} sind, und diese waren unabhängig von der Überlagerung.
    \end{subproof}
    \underline{2. Schritt} Wir zeigen die volltreue.

    \underline{Injektivität} Seien $f,\hat{f}$ zweie Überlagerungsabbildungen, d.h.
    \[
    \begin{tikzcd}[column sep = tiny]
        E \ar{rr}{f}[swap]{\hat{f}} \ar[swap]{dr}{p} & & E' \ar{dl}{p'} \\
    & X
    \end{tikzcd}
    \]
    die unter dem Funktor das gleiche Bild haben, d.h. $f|_{p^{-1} (x_0)} = \hat{f}|_{p^{-1} (x_0)}$. Wir wollen zeigen, dass dann auch schon $f \equiv  \hat{f}$. Sei $\tilde{E} \subset E$ eine beliebige Wegekomponenten. Es genügt zu zeigen, dass $f|_E = \hat{f}|_{\tilde{E}}$.

    Da $X$ lokal wegzusammenhängend ist  $p|_{\tilde{E}}\colon \tilde{E} \to  X$ bereits eine Überlagerung nach \autoref{thm:überlagerung-über-lokal-wegzusammenhängendem-raum-zerfällt-in-wegzusammenhängende-komponenten-von-e}.

    Es ist $\tilde{E} \cap  p^{-1} (x_0) \neq  \emptyset$. Sei $e\in \tilde{E} \cap  p^{-1} (x_0)$. Dann sind $f|_{\tilde{E}}$ und $\hat{f}|_{\tilde{E}}$ Hebungen von
    \[
        \begin{tikzcd}[column sep = large, row sep = large]
        & E' \ar{d}{p'} \\
        \tilde{E} \ar[shift left]{ur}{f|_{\tilde{E}}} \ar[shift right, swap]{ur}{\hat{f}|_{\tilde{E}}} \ar{r}{p|_{\tilde{E}}} & X
    \end{tikzcd}
    .\] 

    \underline{Surjektivität}. Sei $\tilde{f} \colon  p^{-1} (x_0) \to  p'^{-1}(x_0)$ ein Homomorphismus von $\pi_1(X,x_0)$-Mengen. Wir möchten zeigen, dass dieser auch schon von einer Überlagerungsabbildung $f\colon  E \to  E'$ induziert wird. 

    Sei wieder $\tilde{E} \subset E$ eine Wegekomponente, dann ist $p^{-1} (x_0) \cap \tilde{E}$ genau eine Bahn von $p^{-1} (x_0)$ nach ebiger Proposition.

    Sei $e\in p^{-1} (x_0) \cap \tilde{E}$, dann ist
    \[
        p^{-1} (x_0) \cap  \tilde{E} \cong \underbrace{p_*(\pi_1(E,e))}_{\coloneqq H} \backslash \pi_1(X,x_0)
    .\] 
    nach der Vorüberlegung. Es ist $\tilde{f}(e) \in p'^{-1}(x_0)$.

    \begin{claim}
        Es ist $H\leq  p_*'(\pi_1(E',\tilde{f}(e)))$.
    \end{claim}
    \begin{subproof}
        Sei $h\in H$. Dann ist gerade
        \[
            \tilde{f}(e).  h = \tilde{f}(e.h) = \tilde{f}(e) \implies h\in \pi_1(x,x_0)_{\tilde{f}(e)}
        .\] 
        Also ergibt sich
        \[
            H \leq  \pi_1(X,x_0)_{\tilde{f}(e)} = p_*'(\pi_1(E',\tilde{f}(e)))
        .\] 
    \end{subproof}

    Nach dem \nameref{thm:allgemeiner-liftungssatz} existiert also eine Abbildung $f|_{\tilde{E}}\colon \tilde{E} \to  E'$ mit $f|_{\tilde{E}}(e) = \tilde{f}(e)$.
    \[
    \begin{tikzcd}
        & E \ar{d}{p'} \\
        \tilde{E} \ar[dashed]{ur}{f|_{\tilde{E}}} \ar[swap]{r}{p|_{\tilde{E}}} & X
    \end{tikzcd}
    .\]
    \begin{claim}
        Es gilt nun sogar 'automatisch' $f|_{\tilde{E}}(e') = \tilde{f}(e')$ für alle $e' \in p^{-1} (x_0) \cap  \tilde{E}$.
    \end{claim}
    \begin{subproof}
        Mit \autoref{cor:f-und-f-til-sind-gleich-im-hauptsatz}  nach der Pause.
    \end{subproof}
    Definiere nun $f\colon  E \to  E'$ durch
    \[
    E = \coprod \tilde{E} \stackrel{\coprod f|_{\tilde{E}}}{\longrightarrow} E'
    .\] 
    Dann ist $f|_{p^{-1} (x_0)} = \tilde{f}$ nach ebiger Behauptung.
\end{proof}


\begin{lemma}\label{lm:morphismus-von-g-mengen-von-transitiver-menge-ist-auf-einem-element-bestimmt}
    Sei $G$ eine Gruppe,  $M$ eine transitive  $G$-Menge.  $N$ eine  $G$-Menge,  $m\in M$ und $\varphi ,\varphi '\colon M \to  N$ Morphismen von $G$-Mengen. Dann ist bereits $\varphi  \equiv  \varphi '$.
\end{lemma}
\begin{proof}
    Sei $m' \in M$ beliebig. Wegen Transitivität existiert $g\in G$ mit $m.g = m'$. Also rechnen wir einfach nach:
     \[
         \varphi (m') = \varphi (m.g) = \varphi (m).g = \varphi' (m).g = \varphi' (m.g) = \varphi '(m')
    .\] 
\end{proof}

\begin{corollary}\label{cor:f-und-f-til-sind-gleich-im-hauptsatz}
    Es gilt $f|_{\tilde{E}}(e') = \tilde{f}(e')$ für alle $e'\in p^{-1} (x_0) \cap  \tilde{E}$.
\end{corollary}

\begin{proof}
    Folgt als Spezialfall von \autoref{lm:morphismus-von-g-mengen-von-transitiver-menge-ist-auf-einem-element-bestimmt} mit $G = \pi_1(X,x_0)$, $M = p^{-1} (x_0) \cap \tilde{E}$, $N = p'^{-1}(x_0)$ sowie $m = e$ und 
     \[
         \varphi  = (f|_{\tilde{E}})|_{p^{-1} (x_0)\cap \tilde{E}} \qquad \varphi ' = \tilde{f}|_{p^{-1} (x_0) \cap \tilde{E}}
    .\] 
\end{proof}

Damit ist nun der Beweis von \autoref{thm:hauptsatz-der-überlagerungstheorie} abgeschlossen.

Wir erinnern uns an \autoref{def:decktransformation} 

%\decktransformation

und an den verbundenen \autoref{thm:isomorphismus-von-decktransformationen-mit-nebenklassengruppe-von-charakteristischer-untergruppe-in-seinem-normalisator}.

\missingfigure{Überlagerung von $S^1 \twedge S^1$}


\underline{Alternativer Beweis}

Es sind $\Delta(p)$ genau die Automorphismen der Überlagerung  $p\colon  E \to X$. Nach dem Hauptsatz sind diese also isomorph zu den Automorphismen von $p^{-1} (x_0)$ als $\pi_1(X,x_0)$-Mengen, die wiederum isomorph sind zu den Automorphismen $p_*(\pi_1(E,e_0)) \backslash \pi_1(X,x_0)$, d.h.
\[
    \Delta(p) = \Aut \left(
    \begin{tikzcd}
        E \ar{d}{p} \\ X
    \end{tikzcd}\right) 
    \cong \Aut_{\pi_1(X,x_0)-\text{Mengen}} (p^{-1} (x_0)) \cong \Aut \left( p_*(\pi_1(E,e_0)) \backslash \pi_1(x,x_0) \right) 
.\] 

\begin{proposition}
    Es ist
    \[
        \Aut_{G-\text{Mengen}}(H \backslash G) \cong H \backslash N_GH
    .\] 
\end{proposition}
\begin{proof}
    Sei $f\colon  H \backslash G \to  H \backslash G\in \Aut (H \backslash G)$ ein solcher Automorphismus.
    \begin{enumerate}[1)]
    \item Dann ist $f$ bereits eindeutig bestimmt durch  $f(H)$ nach \autoref{lm:morphismus-von-g-mengen-von-transitiver-menge-ist-auf-einem-element-bestimmt}.
    \item $f(H.1) \in H \backslash N_GH$. ist $f(H.1) = H.g_0$, dann ist für alle $h\in H$ auch $H.g_0 = f(H.1) = f(H.h) = f(H.1).h = H.g_0.h$.

        Daraus folgt bereits, dass $\exists h_1,h_1 \in H$ mit $h_1g_0 = h_2g_0h$, also nach umformen
        \[
       h_2^{-1}h_1 = g_0hg_0^{-1}
        .\] 
        wegen $h\in H$ beliebig ergibt sich also bereits $g_0Hg_0^{-1}\subset H$. Das Inverse schickt $Hg_0$ auf $H$. Also schickt se  $H$ auf  $Hg_0^{-1}$. Analog zeigen wir, dass $g_0^{-1}Hg_0 \subset H$, also auch $H\subset g_0Hg_0^{-1}\subset H$ und somit schlussendlich
        \[
        H = g_0Hg_0^{-1} \qquad \implies \qquad g_0\in N_GH
        .\] 
        Die Abbildung
        \[
            f \mapsto f(H\cdot 1)
        .\] 
        ist also wohldefiniert und injektiv nach 1)

    \item Wir zeigen noch Surjektivität. Ist $g_0\in N_GH$, so behaupten wir, dass
        \[
        Hg \mapsto Hg_0g
        .\] 
        wohldefiniert und $G$-äquivariant ist. Die Äquivarianz ist nach Definition offensichtlich. Für Wohldefiniertheit bilden wir ab
         \[
             Hhg \mapsto H g_0hg \stackrel{g_0\in N_GH}{=} H h'g_0g = Hg_0g
        .\] 
        Da $N_GH$ eine Untergruppe ist, ist auch  $g_0^{-1}\in N_GH$, also hat die Abbildung ein Inverses, und zwar
        \[
        Hg \mapsto Hg_0^{-1}g
        .\] 
\end{enumerate}
\end{proof}

Wir wollen nun noch den Satz von Seifert van Kampen beweisen.
\begin{proof}[Beweis von \autoref{thm:seifert-van-kampen}]
    Wir wollen zunächst eine Surjektion
    \[
        \pi_1(U_1,x_0) \star p_1(U_2,x_0) \twoheadrightarrow p_1(x,x_0)
    .\] 
    finden, wie auch schon in der letzten Vorlesung angedeutet. Da $I$ kompakt ist, existiert  für jeden Weg $[w] \in \pi_1(X,x_0)$ ein $k\in \N$, sodass
    \[
w_l \coloneqq         w|_{\left[ \frac{l}{k}, \frac{l+1}{k} \right) } \subset U_i
    .\] 
    für alle $l$. Für alle  $l\in \left \{1,k-1\right\}$ wähle einen Weg $v_l$ von  $w\left( \frac{l}{k} \right) $ nach $x_0$ in 
    \[
    \begin{cases}
        U_3 & \text{falls } w\left( \frac{l}{k} \right) \in U_3 \\
        U_1 & \text{falls } w\left( \frac{l}{k} \right) \not\in U_2 \\
        U_2 & \text{falls } w\left( \frac{l}{k} \right)  \not\in U_1
    \end{cases}
    .\] 
    Wir können nun schreiben
    \[
        [w] = [w_0\star v_1] \star [v_1^{-1} w_1 v_2] \star \ldots \star [v_{k-1}^{-1}w_{k-1}]
    .\] 
    und jeder der geklammerten Wege verläuft nun in einem der $U_i$.
\begin{claim}
    Es ist $\ker(p) = \left< \varphi 1([w])\varphi 2([w])^{-1} \mid  [w] \in \pi_1(U_3,x_0) \right> $.
\end{claim}
\begin{subproof}
    Sei $[w]$ ein beliebiger Weg in  $X$, und schreibe diesen wieder in der Form
     \[
         [w] = [w_1] \star \ldots \star [w_k] = [v_1] \star \ldots \star [v_l]
    .\] 
    mit $w_i, v_i \in U_i$. OBdA sei $k=l$, sonst füge triviale Wgee ein. Sei  $H$ eine Homotopie von 
     \[
    w_1 \star \ldots \star w_k \simeq v_1 \star \ldots \star v_l
    .\] 
    \missingfigure{Skizze der Homotopie}
\end{subproof}
\end{proof}

    %! TEX root = ./master.tex
\lecture[]{Di 13 Jul 2021 12:12}{Beweis des Satzes von Seifert-van-Kampen}


\begin{proof}[Fortsetzung des Beweises zu Seifert van Kampen]
    Wir haben schon bewiesen, dass $\psi $ surjektiv ist.
    \begin{notation*}
        Seien $a,b$ Wege in  $X$ (an  $x_0$). Dann schreiben wir
        \begin{itemize}
            \item $a \sim _{U_i} b\coloneqq $ $a$ und  $b$ sind homotope Wege in  $U_i$, für $i=1,2,3$.
            \item  $a\sim_X b\coloneqq $ $a$ und  $b$ sind homotope Wege in  $X$
            \item $[a]_{U_i}$ für die Klasso von $a$ in  $\pi_1(U_i,x_0)$, wobei wir implizit fordern, dass $a$ bereits in  $U_i$ liegt.
            \item  $[a]_X\coloneqq $ die Klasse von $a$ in  $\pi_1(X,x_0)$
        \end{itemize}
    \end{notation*}
    z.B. ist nun $\varphi 1([a]_{U_3}) = [a]_{U_1}$ sowie $\varphi_2 ([a]_{U_3}) = [a]_{U_2}$.
    \begin{notation*}
        Es bezeichne $\cdot$ die Wegemultiplikation, und es bezeichne  $\star$ die Multiplikation im freine Produkt  $\pi_1(U_1,x_0) \star \pi_1(U_2,x_0)$.
    \end{notation*}
    Damit ergibt sich nun z.B.
    \begin{IEEEeqnarray*}{rCl}
        \psi ([a_1]_{U_1} \star [a_2]_{U_2} \star \ldots \star [a_m]_{U_2}) & = & \psi_1([a_1]_{U_1}) \cdot \psi_2 [a_2]_{U_2}  \ldots
    \end{IEEEeqnarray*}

    Sei $N$ der normale Abschluss
     \[
         N\coloneqq  \overline{F(\pi_1(U_3,x_0))}
    .\] 
    Es ist dann noch zu zeigen, dass $N = \ker \psi $.

    \underline{1. Schritt}: Wir zeigen, dass $N \leq  \ker \psi $. Es genügt zu zeigen, dass $F(\pi_1(U_3,x_0)) \subset  \ker \psi $, denn $\ker \psi $ ist normal.

    Sei also $[a]_{U_3}\in \pi_2(U_3,x_0)$, dann ist
    \begin{comment}
    \begin{IEEEeqnarray*}{rCl}
        \psi  \circ  F([a]_{U_3}) & = & \psi (\varphi_1 ([a]_{U_3}) \star \varphi _2[a]_{U_3}^{-1}) \\
                                  & = & \psi ([a]_{U_1} \star [a]_{U_2}^{-1}) \\
                                  & = & \psi _1 [a]_{U_1} \cdot \psi _2 [a]^{-1}_{U_2} \\
                                  & = & [a]_X \cdot [a]^{-1}_X \\
                                  [ a \cdot a^{-1}]_X = 1
    \end{IEEEeqnarray*}
    \end{comment}

    \underline{2. Schritt}: Es ist $\ker \psi  \leq  N$. 

    Sei $\gamma = [a_1]_{U_1}\star [a_2]_{U_2} \star \ldots \star [a_k]_{U_2}\in \pi_1(U_1,x_0) \star \pi_1(U_2,x_0)$ mit $\psi (\gamma) = 1$ ein generisches Element aus dem Kern. Wir können $\gamma$ stets in diese Form bringen, indem wir Buchstaben aus der gleichen Fundamentalgruppe miteinander verknüpfen, und am Anfang bzw. Ende evtl. mit trivialen Wegen auffüllen.

    Es ist also  $[a_1 \cdot \ldots \cdot a_k]_X = 1 \iff  a_1\cdot \ldots \cdot a_k \sim _X c_{X_0}$. Zu zeigen ist, dass $\gamma \in N$.

    Sei $H\colon [0,1]\times [0,1]\to X$ eine Homotopie (relativ Endpunkten) von $a_1 \cdot \ldots \cdot a_k$ nach $c_{x_0}$. Setze nun für $n$ groß
     \[
    S_{ij} \coloneqq  \left[ \frac{i-1}{n}, \frac{i}{n} \right] \times \left[ \frac{j-i}{n}, \frac{j}{n} \right] 
    .\] 
    Da $[0,1] \times [0,1]$ kompakt ist, $\exists n\in \N$, sodass jedes $S_{ij}$ durch $H$ in  $U_1$ oder $U_2$ abgebildet wird. Zudem wählen wir $n$ groß genug (bzw. vor allem korrekt als Vielfaches), sodass die Endpunkte von  $a_i$ von der Form $\frac{i'}{n}$ für ein geeignetes $i'$ sind.

    Setze  $a_{ij}\coloneqq H|_{\left[ \frac{i-}{n}, \frac{i}{n} \right]\times \left \{\frac{j}{n}\right\}  }$. Damit stellen wir fest:
    \begin{IEEEeqnarray*}{rCl}
        H|_{[0,1]\times 0} & = & a_1 \cdot a_2\cdot a_3\cdot \ldots\cdot a_k \\
                           & = & (\underbrace{a_{10}\cdot a_{20}\cdot \ldots \cdot a_{p_0}}_{ = a_1}) \cdot (a_{p+1,0}\cdot \ldots)\cdot \ldots\cdot \underbrace{(a_{q,0}\cdot \ldots\cdot a_{n,0})}_{=a_k}
    \end{IEEEeqnarray*}

    Setze zudem $v_{ij}\coloneqq H\left( \frac{i}{n}, \frac{j}{n} \right) $ und $b_{ij}\coloneqq H|_{\left \{\frac{i}{n}\right\} \times  \left[ \frac{j-1}{n}, \frac{j}{n} \right] }$.

    Also ergibt sich in $\pi_1(U_1,x_0) \star \pi_1(U_2,x_0)$
    \[
        \gamma = [a_{10} \cdot \ldots \cdot a_{p_0}]_{U_1} \star [a_{p+1,0}\cdot \ldots] \star \ldots \cdot [a_{r,0} \cdot  \ldots \cdot a_{n,0}]_{U_2}
    .\] 
    Wahle wege $h_{ij}$ von $x_0$ nach $v_{ij}$, wobei dieser Wege in $U_l$ verlaufe, wenn  $v_{ij}$ in $U_l$ verläuft (der Weg ist also möglichst restriktiv). Falls  $v_{ij} = x_0$, so wähle die konstante Schleife.

    Setzen wir nun $\tilde{a_{ij}}\coloneqq h_{i-1,j} \cdot a_{ij} \cdot h_{ij}^{-1}$, so haben wir Schleifen gebaut, die per Definition in $U_1$ oder $U_2$ verlaufen (oder beides).

    Dann ist auch weiterhin
    \begin{IEEEeqnarray*}{rCl}
        \gamma & = & [a_{10} \cdot  \ldots \cdot  a_{p_0}]_{U_1} \cdot  \ldots\cdot [a_{r,0} \cdot \ldots.\cdot a_{n,0}]_{U_2} \\
               & = & [\tilde{a_{10}}_{U_1}  \star [\tilde{a_{20}}_{U_1}
    \end{IEEEeqnarray*}
\end{proof}

    % end lectures
\end{document}
