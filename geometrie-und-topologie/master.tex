\makeatletter
% patched command of loading a package
\def\find@load#1[#2]#3[#4]{%
    % load the package
    \find@fileswith@pti@ns{#1}[#2]{#3}[#4]%
    % check if commands exists now
    \ifcsname\find@command\endcsname
        \typeout{Package #3 introduces command \find@command.}%
        % command has been found, revert to original version without checks
        \let\@fileswith@pti@ns\find@fileswith@pti@ns
    \else
        % somehow, \@fileswith@pti@ns is restored after loading a package
        % thus, patch it again
        \let\@fileswith@pti@ns\find@load
    \fi
}%

\newcommand*{\findpackagebycommand}[1]{%
    % using this multiple times - esp. when the package has not been found yet -
    % will break things. Thus, check first that \find@command has never been defined before
    \ifx\find@command\undefined
        \def\find@command{#1}%
        % first, check if this command is already defined
        \ifcsname\find@command\endcsname
            % in this case, just issue a warning and do nothing
            \@latex@warning@no@line{Command \find@command \space is already defined}%
        \else
            % overwrite the internal \@fileswith@pti@ns command, which does the actual loading
            % \@fileswith@pti@ns is used internally by \usepackage and \RequirePackage
            \let\find@fileswith@pti@ns\@fileswith@pti@ns
            \let\@fileswith@pti@ns\find@load
        \fi
    \else
        % used multiple times - prevent and give a warning
        \@latex@warning@no@line{You can use \protect\findpackagebycommand \space only once}%
        \errmessage{Invalid use of command findpackagebycommand.}%
    \fi
}

\makeatother


\findpackagebycommand{aside}



\documentclass[a4paper, german, lecturenumbers = true, number small environments = theorem, hide version]{mkessler-script}

\course{Einführung in die Geometrie und Topologie}
\lecturer{Daniel Kasprowski}
\assistant[f]{Arunima Ray}
\author{Maximilian Keßler}

\RequirePackage{mkessler-math}
\RequirePackage{mkessler-fancythm}
\usepackage{epsfig}
%\usepackage{psfrag}
%\usepackage{sseq} (if you need to draw spectral sequences, please use this package, available at http://wwwmath.uni-muenster.de/u/tbauer/)
\usepackage{mathrsfs}
\usepackage{amscd}
\usepackage{amsbsy}
\usepackage{verbatim}
\usepackage{moreverb}

\newtheorem{prop}[theorem]{Proposition}
\newtheorem{cor}[theorem]{Corollary}
\newtheorem{conj}[theorem]{Conjecture}


\theoremstyle{definition}
\newtheorem{hw}{Homework}
\newtheorem{exercise*}[exercise]{$\star$ Exercise}

\theoremstyle{remark}
\newtheorem{aside}[theorem]{Aside}

\newcommand{\nn}{\nonumber}
\newcommand{\nid}{\noindent}
\newcommand{\ra}{\rightarrow}
\newcommand{\la}{\leftarrow}
\newcommand{\xra}{\xrightarrow}
\newcommand{\xla}{\xleftarrow}
\newcommand{\tto}{\longrightarrow}

\newcommand{\weq}{\xrightarrow{\sim}}
\newcommand{\cofib}{\rightarrowtail}
\newcommand{\fib}{\twoheadrightarrow}

\newcommand{\IRep}{\mathrm{IRep}}
\newcommand{\IHom}{\mathrm{IHom}}

\def\llarrow{   \hspace{.05cm}\mbox{\,\put(0,-2){$\leftarrow$}\put(0,2){$\leftarrow$}\hspace{.45cm}}}
\def\rrarrow{   \hspace{.05cm}\mbox{\,\put(0,-2){$\rightarrow$}\put(0,2){$\rightarrow$}\hspace{.45cm}}}
\def\lllarrow{  \hspace{.05cm}\mbox{\,\put(0,-3){$\leftarrow$}\put(0,1){$\leftarrow$}\put(0,5){$\leftarrow$}\hspace{.45cm}}}
\def\rrrarrow{  \hspace{.05cm}\mbox{\,\put(0,-3){$\rightarrow$}\put(0,1){$\rightarrow$}\put(0,5){$\rightarrow$}\hspace{.45cm}}}

\def\cA{\mathcal A}\def\cB{\mathcal B}\def\cC{\mathcal C}\def\cD{\mathcal D}
\def\cE{\mathcal E}\def\cF{\mathcal F}\def\cG{\mathcal G}\def\cH{\mathcal H}
\def\cI{\mathcal I}\def\cJ{\mathcal J}\def\cK{\mathcal K}\def\cL{\mathcal L}
\def\cM{\mathcal M}\def\cN{\mathcal N}\def\cO{\mathcal O}\def\cP{\mathcal P}
\def\cQ{\mathcal Q}\def\cR{\mathcal R}\def\cS{\mathcal S}\def\cT{\mathcal T}
\def\cU{\mathcal U}\def\cV{\mathcal V}\def\cW{\mathcal W}\def\cX{\mathcal X}
\def\cY{\mathcal Y}\def\cZ{\mathcal Z}

\def\sA{\mathscr A}\def\cB{\mathcal B}\def\cC{\mathcal C}\def\cD{\mathcal D}
\def\cE{\mathcal E}\def\cF{\mathcal F}\def\sG{\mathscr G}\def\cH{\mathcal H}
\def\cI{\mathcal I}\def\cJ{\mathcal J}\def\cK{\mathcal K}\def\cL{\mathcal L}
\def\cM{\mathcal M}\def\cN{\mathcal N}\def\cO{\mathcal O}\def\cP{\mathcal P}
\def\cQ{\mathcal Q}\def\cR{\mathcal R}\def\cS{\mathcal S}\def\cT{\mathcal T}
\def\cU{\mathcal U}\def\cV{\mathcal V}\def\cW{\mathcal W}\def\cX{\mathcal X}
\def\cY{\mathcal Y}\def\cZ{\mathcal Z}

\def\fG{\mathfrak G}\def\fH{\mathfrak H}
\def\fS{\mathfrak S}\def\fN{\mathfrak N}\def\fX{\mathfrak X}\def\fY{\mathfrak Y}

\def\op{\textrm{op}}\def\ob{\textrm{ob}}

%\def\Iso{\mathcal Iso}\def\cInn{\mathcal Inn}

\def\fg{\mathfrak g}\def\fh{\mathfrak h}\def\fri{\mathfrak i}\def\fp{\mathfrak p}
\def\fA{\mathfrak A}\def\fU{\mathfrak U}

\def\AA{\mathbb A}\def\BB{\mathbb B}\def\CC{\mathbb C}\def\DD{\mathbb D}
\def\EE{\mathbb E}\def\FF{\mathbb F}\def\GG{\mathbb G}\def\HH{\mathbb H}
\def\II{\mathbb I}\def\JJ{\mathbb J}\def\KK{\mathbb K}\def\LL{\mathbb L}
\def\MM{\mathbb M}\def\NN{\mathbb N}\def\OO{\mathbb O}\def\PP{\mathbb P}
\def\QQ{\mathbb Q}\def\RR{\mathbb R}\def\SS{\mathbb S}\def\TT{\mathbb T}
\def\UU{\mathbb U}\def\VV{\mathbb V}\def\WW{\mathbb W}\def\XX{\mathbb X}
\def\YY{\mathbb Y}\def\ZZ{\mathbb Z}

\def\TOP{\mathcal{TOP}}\def\GRP{\mathcal{GRP}}\def\GRPD{\mathcal{GRPD}} \def\CAT{\mathcal{CAT}} \def\SET{\mathcal{SET}}

\def\id{\mathrm{id}}\def\Id{\mathrm{Id}}
\def\inverse{^{-1}}



\begin{document}
    \maketitle
    \begin{abstract}
    {\color{red} Achtung:} Diese Version des Skripts benutze ich zur Bearbeitung! Einige Dinge fehlen, dafür gibt es TODO-Notes. Für Inhalte, benutzt die \href{https://kesslermaximilian.github.io/LectureNotesBonn/2021_Topologie.pdf}{normale Version}
    \end{abstract}
    \newpage
    \listoftodos
    \newpage
    \summaryoflectures
    \newpage
    % start lectures
    \setcounter{section}{20}
    \setcounter{dummy}{8}
    \setcounter{smalldummy}{0}
    \setcounter{figure}{29}
    \setcounter{claim}{1}
    \setcounter{lecture}{22}
    %! TEX root = ./master.tex
\lecture[]{Di 13 Jul 2021 12:12}{Beweis des Satzes von Seifert-van-Kampen}


\begin{proof}[Fortsetzung des Beweises zu Seifert van Kampen]
    Wir haben schon bewiesen, dass $\psi $ surjektiv ist.
    \begin{notation*}
        Seien $a,b$ Wege in  $X$ (an  $x_0$). Dann schreiben wir
        \begin{itemize}
            \item $a \sim _{U_i} b\coloneqq $ $a$ und  $b$ sind homotope Wege in  $U_i$, für $i=1,2,3$.
            \item  $a\sim_X b\coloneqq $ $a$ und  $b$ sind homotope Wege in  $X$
            \item $[a]_{U_i}$ für die Klasso von $a$ in  $\pi_1(U_i,x_0)$, wobei wir implizit fordern, dass $a$ bereits in  $U_i$ liegt.
            \item  $[a]_X\coloneqq $ die Klasse von $a$ in  $\pi_1(X,x_0)$
        \end{itemize}
    \end{notation*}
    z.B. ist nun $\varphi 1([a]_{U_3}) = [a]_{U_1}$ sowie $\varphi_2 ([a]_{U_3}) = [a]_{U_2}$.
    \begin{notation*}
        Es bezeichne $\cdot$ die Wegemultiplikation, und es bezeichne  $\star$ die Multiplikation im freine Produkt  $\pi_1(U_1,x_0) \star \pi_1(U_2,x_0)$.
    \end{notation*}
    Damit ergibt sich nun z.B.
    \begin{IEEEeqnarray*}{rCl}
        \psi ([a_1]_{U_1} \star [a_2]_{U_2} \star \ldots \star [a_m]_{U_2}) & = & \psi_1([a_1]_{U_1}) \cdot \psi_2 [a_2]_{U_2}  \ldots
    \end{IEEEeqnarray*}

    Sei $N$ der normale Abschluss
     \[
         N\coloneqq  \overline{F(\pi_1(U_3,x_0))}
    .\] 
    Es ist dann noch zu zeigen, dass $N = \ker \psi $.

    \underline{1. Schritt}: Wir zeigen, dass $N \leq  \ker \psi $. Es genügt zu zeigen, dass $F(\pi_1(U_3,x_0)) \subset  \ker \psi $, denn $\ker \psi $ ist normal.

    Sei also $[a]_{U_3}\in \pi_2(U_3,x_0)$, dann ist
    \begin{comment}
    \begin{IEEEeqnarray*}{rCl}
        \psi  \circ  F([a]_{U_3}) & = & \psi (\varphi_1 ([a]_{U_3}) \star \varphi _2[a]_{U_3}^{-1}) \\
                                  & = & \psi ([a]_{U_1} \star [a]_{U_2}^{-1}) \\
                                  & = & \psi _1 [a]_{U_1} \cdot \psi _2 [a]^{-1}_{U_2} \\
                                  & = & [a]_X \cdot [a]^{-1}_X \\
                                  [ a \cdot a^{-1}]_X = 1
    \end{IEEEeqnarray*}
    \end{comment}

    \underline{2. Schritt}: Es ist $\ker \psi  \leq  N$. 

    Sei $\gamma = [a_1]_{U_1}\star [a_2]_{U_2} \star \ldots \star [a_k]_{U_2}\in \pi_1(U_1,x_0) \star \pi_1(U_2,x_0)$ mit $\psi (\gamma) = 1$ ein generisches Element aus dem Kern. Wir können $\gamma$ stets in diese Form bringen, indem wir Buchstaben aus der gleichen Fundamentalgruppe miteinander verknüpfen, und am Anfang bzw. Ende evtl. mit trivialen Wegen auffüllen.

    Es ist also  $[a_1 \cdot \ldots \cdot a_k]_X = 1 \iff  a_1\cdot \ldots \cdot a_k \sim _X c_{X_0}$. Zu zeigen ist, dass $\gamma \in N$.

    Sei $H\colon [0,1]\times [0,1]\to X$ eine Homotopie (relativ Endpunkten) von $a_1 \cdot \ldots \cdot a_k$ nach $c_{x_0}$. Setze nun für $n$ groß
     \[
    S_{ij} \coloneqq  \left[ \frac{i-1}{n}, \frac{i}{n} \right] \times \left[ \frac{j-i}{n}, \frac{j}{n} \right] 
    .\] 
    Da $[0,1] \times [0,1]$ kompakt ist, $\exists n\in \N$, sodass jedes $S_{ij}$ durch $H$ in  $U_1$ oder $U_2$ abgebildet wird. Zudem wählen wir $n$ groß genug (bzw. vor allem korrekt als Vielfaches), sodass die Endpunkte von  $a_i$ von der Form $\frac{i'}{n}$ für ein geeignetes $i'$ sind.

    Setze  $a_{ij}\coloneqq H|_{\left[ \frac{i-}{n}, \frac{i}{n} \right]\times \left \{\frac{j}{n}\right\}  }$. Damit stellen wir fest:
    \begin{IEEEeqnarray*}{rCl}
        H|_{[0,1]\times 0} & = & a_1 \cdot a_2\cdot a_3\cdot \ldots\cdot a_k \\
                           & = & (\underbrace{a_{10}\cdot a_{20}\cdot \ldots \cdot a_{p_0}}_{ = a_1}) \cdot (a_{p+1,0}\cdot \ldots)\cdot \ldots\cdot \underbrace{(a_{q,0}\cdot \ldots\cdot a_{n,0})}_{=a_k}
    \end{IEEEeqnarray*}

    Setze zudem $v_{ij}\coloneqq H\left( \frac{i}{n}, \frac{j}{n} \right) $ und $b_{ij}\coloneqq H|_{\left \{\frac{i}{n}\right\} \times  \left[ \frac{j-1}{n}, \frac{j}{n} \right] }$.

    Also ergibt sich in $\pi_1(U_1,x_0) \star \pi_1(U_2,x_0)$
    \[
        \gamma = [a_{10} \cdot \ldots \cdot a_{p_0}]_{U_1} \star [a_{p+1,0}\cdot \ldots] \star \ldots \cdot [a_{r,0} \cdot  \ldots \cdot a_{n,0}]_{U_2}
    .\] 
    Wahle wege $h_{ij}$ von $x_0$ nach $v_{ij}$, wobei dieser Wege in $U_l$ verlaufe, wenn  $v_{ij}$ in $U_l$ verläuft (der Weg ist also möglichst restriktiv). Falls  $v_{ij} = x_0$, so wähle die konstante Schleife.

    Setzen wir nun $\tilde{a_{ij}}\coloneqq h_{i-1,j} \cdot a_{ij} \cdot h_{ij}^{-1}$, so haben wir Schleifen gebaut, die per Definition in $U_1$ oder $U_2$ verlaufen (oder beides).

    Dann ist auch weiterhin
    \begin{IEEEeqnarray*}{rCl}
        \gamma & = & [a_{10} \cdot  \ldots \cdot  a_{p_0}]_{U_1} \cdot  \ldots\cdot [a_{r,0} \cdot \ldots.\cdot a_{n,0}]_{U_2} \\
               & = & [\tilde{a_{10}}_{U_1}  \star [\tilde{a_{20}}_{U_1}
    \end{IEEEeqnarray*}
\end{proof}

    % end lectures
\end{document}
