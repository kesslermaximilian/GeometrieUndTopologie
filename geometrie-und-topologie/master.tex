\documentclass[a4paper, german, lecturenumbers = true, number small environments = theorem, hide version]{mkessler-script}

\course{Einführung in die Geometrie und Topologie}
\lecturer{Daniel Kasprowski}
\assistant[f]{Arunima Ray}
\author{Maximilian Keßler}

\RequirePackage{mkessler-math}
\RequirePackage{mkessler-fancythm}
\usepackage{epsfig}
%\usepackage{psfrag}
%\usepackage{sseq} (if you need to draw spectral sequences, please use this package, available at http://wwwmath.uni-muenster.de/u/tbauer/)
\usepackage{mathrsfs}
\usepackage{amscd}
\usepackage{amsbsy}
\usepackage{verbatim}
\usepackage{moreverb}

\newtheorem{prop}[theorem]{Proposition}
\newtheorem{cor}[theorem]{Corollary}
\newtheorem{conj}[theorem]{Conjecture}


\theoremstyle{definition}
\newtheorem{hw}{Homework}
\newtheorem{exercise*}[exercise]{$\star$ Exercise}

\theoremstyle{remark}
\newtheorem{aside}[theorem]{Aside}

\newcommand{\nn}{\nonumber}
\newcommand{\nid}{\noindent}
\newcommand{\ra}{\rightarrow}
\newcommand{\la}{\leftarrow}
\newcommand{\xra}{\xrightarrow}
\newcommand{\xla}{\xleftarrow}
\newcommand{\tto}{\longrightarrow}

\newcommand{\weq}{\xrightarrow{\sim}}
\newcommand{\cofib}{\rightarrowtail}
\newcommand{\fib}{\twoheadrightarrow}

\newcommand{\IRep}{\mathrm{IRep}}
\newcommand{\IHom}{\mathrm{IHom}}

\def\llarrow{   \hspace{.05cm}\mbox{\,\put(0,-2){$\leftarrow$}\put(0,2){$\leftarrow$}\hspace{.45cm}}}
\def\rrarrow{   \hspace{.05cm}\mbox{\,\put(0,-2){$\rightarrow$}\put(0,2){$\rightarrow$}\hspace{.45cm}}}
\def\lllarrow{  \hspace{.05cm}\mbox{\,\put(0,-3){$\leftarrow$}\put(0,1){$\leftarrow$}\put(0,5){$\leftarrow$}\hspace{.45cm}}}
\def\rrrarrow{  \hspace{.05cm}\mbox{\,\put(0,-3){$\rightarrow$}\put(0,1){$\rightarrow$}\put(0,5){$\rightarrow$}\hspace{.45cm}}}

\def\cA{\mathcal A}\def\cB{\mathcal B}\def\cC{\mathcal C}\def\cD{\mathcal D}
\def\cE{\mathcal E}\def\cF{\mathcal F}\def\cG{\mathcal G}\def\cH{\mathcal H}
\def\cI{\mathcal I}\def\cJ{\mathcal J}\def\cK{\mathcal K}\def\cL{\mathcal L}
\def\cM{\mathcal M}\def\cN{\mathcal N}\def\cO{\mathcal O}\def\cP{\mathcal P}
\def\cQ{\mathcal Q}\def\cR{\mathcal R}\def\cS{\mathcal S}\def\cT{\mathcal T}
\def\cU{\mathcal U}\def\cV{\mathcal V}\def\cW{\mathcal W}\def\cX{\mathcal X}
\def\cY{\mathcal Y}\def\cZ{\mathcal Z}

\def\sA{\mathscr A}\def\cB{\mathcal B}\def\cC{\mathcal C}\def\cD{\mathcal D}
\def\cE{\mathcal E}\def\cF{\mathcal F}\def\sG{\mathscr G}\def\cH{\mathcal H}
\def\cI{\mathcal I}\def\cJ{\mathcal J}\def\cK{\mathcal K}\def\cL{\mathcal L}
\def\cM{\mathcal M}\def\cN{\mathcal N}\def\cO{\mathcal O}\def\cP{\mathcal P}
\def\cQ{\mathcal Q}\def\cR{\mathcal R}\def\cS{\mathcal S}\def\cT{\mathcal T}
\def\cU{\mathcal U}\def\cV{\mathcal V}\def\cW{\mathcal W}\def\cX{\mathcal X}
\def\cY{\mathcal Y}\def\cZ{\mathcal Z}

\def\fG{\mathfrak G}\def\fH{\mathfrak H}
\def\fS{\mathfrak S}\def\fN{\mathfrak N}\def\fX{\mathfrak X}\def\fY{\mathfrak Y}

\def\op{\textrm{op}}\def\ob{\textrm{ob}}

%\def\Iso{\mathcal Iso}\def\cInn{\mathcal Inn}

\def\fg{\mathfrak g}\def\fh{\mathfrak h}\def\fri{\mathfrak i}\def\fp{\mathfrak p}
\def\fA{\mathfrak A}\def\fU{\mathfrak U}

\def\AA{\mathbb A}\def\BB{\mathbb B}\def\CC{\mathbb C}\def\DD{\mathbb D}
\def\EE{\mathbb E}\def\FF{\mathbb F}\def\GG{\mathbb G}\def\HH{\mathbb H}
\def\II{\mathbb I}\def\JJ{\mathbb J}\def\KK{\mathbb K}\def\LL{\mathbb L}
\def\MM{\mathbb M}\def\NN{\mathbb N}\def\OO{\mathbb O}\def\PP{\mathbb P}
\def\QQ{\mathbb Q}\def\RR{\mathbb R}\def\SS{\mathbb S}\def\TT{\mathbb T}
\def\UU{\mathbb U}\def\VV{\mathbb V}\def\WW{\mathbb W}\def\XX{\mathbb X}
\def\YY{\mathbb Y}\def\ZZ{\mathbb Z}

\def\TOP{\mathcal{TOP}}\def\GRP{\mathcal{GRP}}\def\GRPD{\mathcal{GRPD}} \def\CAT{\mathcal{CAT}} \def\SET{\mathcal{SET}}

\def\id{\mathrm{id}}\def\Id{\mathrm{Id}}
\def\inverse{^{-1}}



\begin{document}
    \maketitle
    \begin{abstract}
    {\color{red} Achtung:} Diese Version des Skripts benutze ich zur Bearbeitung! Einige Dinge fehlen, dafür gibt es TODO-Notes. Für Inhalte, benutzt die \href{https://kesslermaximilian.github.io/LectureNotesBonn/2021_Topologie.pdf}{normale Version}
    \end{abstract}
    \newpage
    \listoftodos
    \newpage
    \summaryoflectures
    \newpage
    % start lectures
    \setcounter{section}{14}
    \setcounter{dummy}{12}
    \setcounter{smalldummy}{1}
    \setcounter{figure}{17}
    \setcounter{claim}{2}
    \setcounter{lecture}{13}
    %! TEX root = ./master.tex
\lecture[]{Do 10 Jun 2021 10:15}{}



    %! TEX root = ./master.tex
\lecture[Weghebungssatz. Lebesguelemma. Homotopieliftungssatz.]{Di 15 Jun 2021 12:15}{Hebungssätze}

\begin{theorem}[Weghebungssatz]\label{thm:weghebungssatz}
    Sei $p\colon E \to  X$ eine Überlagerung, und $w\colon  [0,1] \to  X$ ein Weg mit Anfangspunkt $x_0\coloneqq w(0)$. Sei $y\in p^{-1} (x_0)$ (also ein Punkt in der Faser von $x_0$). Dann existiert genau ein Weg $\tilde{w}\colon  [0,1] \to  E$, sodass $\tilde{w}(0) = y$ und $p \circ  \tilde{w} = w$, d.h. es kommutiert
    \[
    \begin{tikzcd}
        & E \ar{d}{p} \\
        I \ar[dashed]{ur}{\tilde{w}} \ar[swap]{r}{w} & X
    \end{tikzcd}
    \]
    Der Weg $\tilde{w}$ heißt \vocab[Weg!Hebung]{Hebung} oder \vocab[Weg!Lift]{Lift} von $w$.  
\end{theorem}

\begin{oral}[auf Nachfrage]
    Wir können uns den Endpunkt des Weges $\tilde{w}$, also $\tilde{w}(1)$, nicht aussuchen. Dieser ist aber eindeutig bestimmt durch die Wahl des Anfangspunktes $\tilde{w}(0) = y$.

    Wir wissen also, dass dieser Endpunkt existiert und eindeutig bestimmt ist, können aber noch keine Aussage darüber treffen.

    Zudem werden wir sehen, dass der Endpunkt nicht notwendigerweise der Anfangspunkt sein wird, selbst wenn es sich bei $w$ um eine Schleife handelt.
\end{oral}

\begin{proof}
    \underline{1. Fall}: Wir nehmen an, dass $p$ trivial ist, d.h. wir finden $F$, sodass kommutiert:
    \[
    \begin{tikzcd}
        E \ar{rr}{\cong}[swap]{u} \ar[swap]{dr}{p} & & X\times F \ar{dl}{\pr_X} \\
    & X
    \end{tikzcd}
    \]
    Wir müssen also zeigen, dass genau eine Hebung $\tilde{w}\colon I \to  X\times F$ existiert mit $\tilde{w}(0) = u(y)$. Sei $u(y) = (x_0,f)$ mit $f\in F$. Wir definieren nun
    \[
        \tilde{w}(t) = (w(t),f)
    .\] 
    d.h. wir heben den Weg einfach nach $X\times \left \{f\right\} \cong X$. Dann ist $\tilde{w}(0) = (w(0),f) = (x_0,f) = u(y)$, und die Projektion ist genau
    \[
        \pr_X(\tilde{w}(t)) = w(t)
    .\] 
    wie gewünscht.

    \underline{Eindeutigkeit}:  Sei $\tilde{\tilde{w}}\colon I \to  X \times F$ ein weiterer Lift mit Anfangspunkt $(w(0),f)$. Weil  $I$ zusammenhängend ist, ist $\pr_F \circ  \tilde{\tilde{w}} \colon  I \to  F$ konstant, weil das Bild zusammenhängend, $F$ aber diskret ist. Also folgt bereits
     \[
         \tilde{\tilde{w}} (t) = (w(t),f) = \tilde{w}(t)
    .\] 
    , denn die zweite Komponente ergibt sich aus vorherigem Argument, und die erste dann sofort aus der Hebungseigenschaft.

    \underline{2. Fall}: $w$ hat Bild in einer trivialisierenden Umgebung  $U$, d.h. in  $U\subset X$ mit $p|_{p^{-1} (U)}$ trivial. 

    Wir fassen $w$ als Weg  $I \to  U$ auf. Nach Fall 1 existiert $\tilde{w}\colon  I \to  p^{-1} (U)$ mit $\tilde{w}(0) = y$. Dann ist $\tilde{w}\colon  I \to  p^{-1} (U) \hookrightarrow E$ eine Hebung von $w$ entlang  $p\colon  E \to  X$.
    \[
    \begin{tikzcd}
        I \ar{r} \ar[swap]{dr}{w} & p^{-1} (U) \ar{d}{p|_{p^{-1} (U)}} \ar[hook]{r} & E \ar{d}{p}\\
                                  & U \ar{r}& X
    \end{tikzcd}
\]

    Da jede Hebung Bild in $p^{-1} (U)$ hat (Verknüpfung mit $p$ liefert ja einen Weg in  $U$, nämlich  $w$), folgt die Eindeutigkeit auch aus Fall 1.

    \underline{3. Fall}: Allgemeiner Fall. Sei $\left \{U_i\right\} _{i \in I}$ eine offene Überdeckung von $X$, so dass  $p|_{p^{-1} (U_i)}$ trivial ist für alle $i\in I$.

    Dann ist $\left \{w^{-1}(U_i)\right\}_{i \in I} $ eine offene Überdeckung von $I$. Es gibt also eine Lebesgue-Zahl  $ε>0$, d.h. ein  $ε>0$, sodass jeder  $ε$-Ball  $U(t,ε)\subset I$ in einem $w^{-1}(U_j)$ liegt. Also finden wir ein $n\in \N$, so dass 
    \[
      \forall k=0,\ldots,n-1 \; \exists i\in I \colon  \quad  \left[ \frac{k}{n}, \frac{k+1}{n} \right] \subset w^{-1}(U_i)
    \]
    d.h. analog, dass
    \[
    w|_{\left[ \frac{k}{n}, \frac{k+1}{n} \right] } \colon  \left[ \frac{k}{n}, \frac{k+1}{n} \right] \to  X
    .\] 
    hat Bild in einer trivialisierenden Umgebung $U_i$.

\begin{figure}[ht]
    \centering
    \incfig{hebung-auf-einzelnen-offenen-mengen}
    \caption{Zerlegung des Weges in trivialisierende Umgebungen, auf denen wir heben können}
    \label{fig:hebung-auf-einzelnen-offenen-mengen}
\end{figure}

    Wir zeigen per Induktion, dass $w|_{\left[ 0,\frac{k}{n} \right] }\colon  \left[ 0, \frac{k}{n} \right]  \to  X$ einen eindeutigen Lift $\tilde{w}$ mit Anfangspunkt $y$ hat.

     \underline{IA}: $k=1$ ist genau die Aussage von Fall 2, wir sind also fertig. 

     \underline{IS} Es gelte die Aussage für $k$. Sei  $\tilde{w}\colon  \left[0,\frac{k}{n}\right]\to  E$ die eindeutige Hebung von $w|_{\left[0,\frac{k}{n}\right]}$ und setze $y_k \coloneqq  \tilde{w}\left( \frac{k}{n} \right) $. Nach Fall 2 hat also der Weg
     \[
     w|_{\left[ \frac{k}{n}, \frac{k+1}{n} \right] }\colon  \left[ \frac{k}{n}, \frac{k+1}{n} \right] \to  X
     .\] 
     einen eindeutigen Lift $\tilde{w}'$ mit $\tilde{w}'\left( \frac{k}{n} \right) = \tilde{w}\left(\frac{k}{n}\right)$. Nun passen $\tilde{w}$ und $\tilde{w}'$ zusammen zu einer Hebung 
     \[
     w|_{\left[ 0, \frac{k+1}{n} \right] }
     .\] 
     zusammen. Zudem ist diese Hebung eindeutige, denn das Anfangsstück auf $\left[0, \frac{k}{n}\right]$ ist nach Induktion schon eindeutig, also auch der Anfangspunkt  $y_k$ von  $\tilde{w}'$, und somit auch $\tilde{w}'$ nach Fall 2.
\end{proof}

\begin{remark*}
    Das Lebesgue-Lemma fand sich auf den Übungsblättern als \autoref{aufgabe-5.4}, wir geben dies im folgenden wieder:
\end{remark*}

\begin{definition*}[Lebesguezahl]\label{def:lebesguezahl}
    Sei $\mathcal{U}$ eine (offene) Überdeckung eines metrischen Raumes $X$. Dann ist eine  $ε>0$ eine  \vocab{Lebesguezahl}, falls $\forall x\in X$ eine (offene) Umgebung $U\in \mathcal{U}$ mit $U(x,ε) \subset U$.
\end{definition*}

\begin{lemma*}[Lebesgue-Lemma]\label{lm:lebesgue}
    Ist $X$ eine kompakter metrischer Raum,  $\mathcal{U}$ eine offene Überdeckung. Dann existiert eine Lebesguezahl $ε>0$.
\end{lemma*}

\begin{remark*}
    Auf dem Übungsblatt haben wir das Lemma für einen \textit{folgenkompakten} metrischen Raum gezeigt, und das ist auch die Eigenschaft, die wir im Beweis verwenden. Allerdings ist \autoref{aufgabe-5.4} auch genau dazu da, zu zeigen, dass Folgenkompaktheit und Kompaktheit für metrische Räume äquivalent sind, in der Formulierung des Lebesgue-Lemmas ist dies alos nicht (mehr) wichtig, sobald wir das wissen.
\end{remark*}

\begin{remark}
    Ist $w\colon  I \to  X$ eine Schleife, so ist $\tilde{w}$ im Allgemeinen trotzdem keine Schleife, hierzu betrachte wieder die Überlagerung $\R \stackrel{\exp }{\longrightarrow} S^1$ und die Schleife $w$, die einmal um den Kreis läuft, die Hebung ist in $\R$ jedoch einfach ein Weg von $2\pi k$ zu $2\pi(k+1)$.


    \begin{minipage}{\textwidth}
    \centering
    \incfig{hebung-von-schleife-zu-weg}
    \captionof{figure}{Hebung der Schleife in $S^1$ zu einem Weg in  $\R$}
    \label{fig:hebung-von-schleife-zu-weg}
    \end{minipage}
\end{remark}

\begin{theorem}[Homotopieliftungssatz]\label{thm:homotopieliftungssatz}
    Sei $p\colon  E \to  X$ eine Überlagerung, und seien $\tilde{w}_0, \tilde{w}_1\colon  I \to  E$ Wege mit $\tilde{w}_0(0) = \tilde{w}_1(0)$, also gleichem Anfangspunkt. Sei $w_i = p \circ  \tilde{w}_i$.

    Sei $H\colon  I \times I \to  X$ eine Homotopie von $w_0$ nach $w_1$ relativ Anfangspunkten.

    \begin{enumerate}[i)]
        \item Es gibt eine eindeutige Hebung $\tilde{H}\colon  I \times I \to  E$ von $H$ mit  $\tilde{H}(0,0) = \tilde{w}_0(0)$.
        \item $\tilde{H}$ ist eine Homotopie von $\tilde{w}_0$ nach $\tilde{w}_1$ relativ Anfangspunkten.
        \item Ist $H$ eine Homotopie relativ Endpunkt, so auch  $\tilde{H}$.
    \end{enumerate}
\end{theorem}


\begin{figure}[ht]
    \centering
    \incfig{hebung-von-homotopie}
    \caption{Hebung der Homotopie $H$ in den Überlagerungsraum  $E$}
    \label{fig:hebung-von-homotopie}
\end{figure}


\begin{proof}[Beweis von \autoref{thm:homotopieliftungssatz}]
    \begin{description}
        \item[Existenz von $\tilde{H}$]. Sei $\left \{U_i\right\} $ eine offene Überdeckung von $X$, so dass  $p$ über jedem  $U_i$ trivial ist. Dann ist  $\left \{H^{-1}(U_i)\right\} _{i \in I}$ eine offene Überdeckung von $I^2$. Nach dem Lebesguelemma existiert $m\in \N$, so dass jedes Quadrat
            \[
            Q_{i,j} \coloneqq  \left[ \frac{i-1}{m}, \frac{i}{m} \right] \times \left[ \frac{j-1}{m}, \frac{j}{m} \right] 
            .\] 
            für $i,j = 1,\ldots,m$.
            \todo{Ausführlichere anwendung von lebesgue-lemma}
            \todo{$i$ wird doppelt benutzt.}
            Wir definieren im Folgenden stetige Abbildungen $\tilde{H}_{ij}\colon  Q_{ij} \to  E$, so dass
            \begin{itemize}
                \item $ p \circ  \tilde{H}_{ij} = H|_{Q_{ij}}$ 
                \item $\tilde{H}_{11}(0,0) = \tilde{w}_0(0) = \tilde{w}_1(0)$
                \item $(\tilde{H}_{ij})|_{Q_{ij} \cap  Q_{i'j'}} = (\tilde{H}_{i'j'})|_{Q_{/j} \cap  Q_{i'j'}}$ (die Abbildungen passen an den Rändern zusammen)
            \end{itemize}
    Das machen wir induktiv in der Reihenfolge $11,12,\ldots$,$1m, 21,22$,\ldots,$2m, \ldots, 3m$, \ldots,$mm$.

    Es liegt $H_{Q_{ij}}$ in einem $U$. Sei  $s\colon  U \to  E$ eine lokale Umkehrfunktion mit
    \[
        s\left(H\left( \frac{i-1}{m}, \frac{j-1}{m} \right) \right) = \begin{cases}
            \tilde{w}_0(0) & i=j=1 \\
            \tilde{H}_{i-1,j}\left( \frac{i-1}{m}, \frac{j-1}{m} \right) & i>1 \\
            \tilde{H}_{i,j-1}\left( \frac{i-1}{m}, \frac{j-1}{m} \right) & j>1
        \end{cases}
    .\] 
    - wir wollen also, dass der Funktionswert an der linken unteren Ecke bereits stimmt.
    \begin{remark}
        Da $\tilde{H}_{i-1,j}$ und $ \tilde{H}_{i,j-1}$ auf $\left[ \frac{i-1}{m}, \frac{j-1}{m} \right) $ übereinstimmen, ist das wohldefiniert für $i,j > 1$.
    \end{remark}
    Setze $\tilde{H}_{ij}\coloneqq  s \circ  H|_{Q_{ij}}$. Per Definition ($s$ ist eine Umkehrfunktion) ist nun  $p \circ  \tilde{H}_{ij} = p \circ  s \circ  H\mid _{Q_{ij}} = H\mid _{Q_{ij}}$.
    \begin{claim}
        Ist $i>1$, so ist  $\tilde{H}_{ij}|_{Q_{ij} \cap  Q_{i-1,j}} = \tilde{H}_{i-1,j} |_{Q_{ij} \cap Q_{i-1,j}}$
    \end{claim}
    \begin{subproof}
        Es ist $Q_{ij} \cap  Q_{i-1,j} = \left \{\left( \frac{i-1}{m}, \frac{j-t}{m} \mid t\in I \right) \right\} $.
        
        Die Wege
        \begin{IEEEeqnarray*}{rCl}
            t &\mapsto &\tilde{H}_{ij}\left( \frac{i-1}{m}, \frac{j-t}{m} \right)  \\
            t & \mapsto & \tilde{H}_{i-1,j} \left( \frac{i-1}{m}, \frac{j-t}{m} \right) 
        \end{IEEEeqnarray*}
        heben $t \mapsto H\left( \frac{i-1}{m}, \frac{j-t}{m} \right) $ und stimmen für $t=0$ überein. Nach dem \nameref{thm:weghebungssatz} sind sie also gleich.

        Analog zeigen wir: Ist $j>1$, so ist  $\tilde{H}_{ij}\mid _{Q_{ij} \cap  Q_{i,j-1}} = \tilde{H}_{i,j-1}\mid_{ Q_{ij}\cap  Q_{i,j-1}}$
\end{subproof}
\item[Eindeutigkeit] Sei $\tilde{\tilde{H}}$ eine weitere Hebung mit $\tilde{\tilde{H}} (0,0) = \tilde{w}_0(0)$. Sei $(t,s) \in I^2$ und $w$ ein Weg von  $(0,0)$ nach  $(t,s)$ in  $I^2$ (z.B. der lineare Weg). Dann sind $\tilde{H} \circ w$ und $\tilde{\tilde{H}} \circ  w$ Hebungen von $H\circ  w$. mit demselben Anfangspunkt. 

    Nach der Eindeutigkeit im \nameref{thm:weghebungssatz} sind also auch schon $\tilde{H} \circ w$ und $\tilde{\tilde{H}} \circ w$ gleich, insbosender stimmen sie an $(t,s)$ überein, und somit  $\tilde{\tilde{H}}(t,s) = \tilde{H}(t,s) $. Da $(t,s)\in I^2$ beliebig war, folgt also wie gewünscht $\tilde{H} = \tilde{\tilde{H}} $.
\end{description}
\begin{enumerate}[i)]
    \setItemnumber{2}
\item Der Weg $\tilde{H}(-,0)$ hebt $H(-,0)= w_0$ mit Anfangspunkt $\tilde{w}_0(0)$. Auch $\tilde{w}_0$ ist ein solcher Lift. Aus der Eindeutigkeit im \nameref{thm:weghebungssatz} folgt also genau $\tilde{H}(-,0) = \tilde{w}_0$.
    \missingfigure{Illustration der verschiedenen Wege}
    Weiter hebt $\tilde{H}(0,-)$ den Weg $H(0,-) = c_{w_0(0)}$, weil $H$ eine Homotopie relativ Anfangspunkt ist. Auch  $c_{\tilde{w}_0(0)}$ ist eine solche Hebung (mit gleichem Anfangspunkt), also ist bereits $\tilde{H}(0,-) = c_{\tilde{w}_0(0)}$.

    Damit folgt bereits, dass $\tilde{H}$ eine Homotopie relativ Anfangspunkt ist, und dass $\tilde{H}(0,1) = c_{\tilde{w}_0(0)}$. Also hebt der Weg $\tilde{H}(-,1)$ den Weg $H(-,1) = w_1$ \textit{mit Anfangspunkt} $\tilde{w}_0(0) = \tilde{w}_1(0)$, und wegen der Eindeutigkeit der Wegeliftung erhalten wir wie gewünscht $\tilde{H}(-,1) = \tilde{w}_1$.
\item Der Weg $\tilde{H}(1,-)$ hebt nun $H(1,-)$. Ist  $H(1,-)$ konstant, dann auch  $\tilde{H}(1,-)$, weil wir auch die konstante Hebung haben, und die Hebung eindeutig ist.
\end{enumerate}
\end{proof}

\section{Beispiele für $\pi_1$}

\begin{definition}
    Ein topologischer Raum $X$ heißt  \vocab[Topologischer Raum!einfach zusammenhängend]{einfach zusammenhängend}, falls $X$ wegzusammenhängend ist und je zwei Wege mit gleichen Anfangs- und Endpunkten homotop sind (relativ Anfangs- und Endpunkt).
\end{definition}

\begin{lemma}
    Ein Raum $X$ ist einfach zusammenhängend genau dann, wenn er wegzusammenhängend ist und  $\pi_1(X,x)$ trivialist für ein (oder äquivalent alle) $x\in X$.
\end{lemma}

\begin{proof}
'$\implies$' Spezialfall der Definition.    

'$\impliedby$' Seien $w, w'\colon I \to X$ Wege mit $w(0) = w'(0) =x$ und  $w(1) = w'(1) = y$. Dann ist 
 \[
     w \simeq (w' \star  \overline{w'}) \star w \simeq w' \star (\overline{w'} \star w) \stackrel{\pi_1(X,y) = 0}{\simeq} w'
.\] 
\end{proof}


    % end lectures
    %\input{fragestunden.tex}
\end{document}
