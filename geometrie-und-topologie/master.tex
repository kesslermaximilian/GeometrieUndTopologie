\documentclass[a4paper, german, lecturenumbers = true, number small environments = theorem, hide version]{mkessler-script}

\course{Einführung in die Geometrie und Topologie}
\lecturer{Daniel Kasprowski}
\assistant[f]{Arunima Ray}
\author{Maximilian Keßler}

\RequirePackage{mkessler-math}
\RequirePackage{mkessler-fancythm}
\usepackage{epsfig}
%\usepackage{psfrag}
%\usepackage{sseq} (if you need to draw spectral sequences, please use this package, available at http://wwwmath.uni-muenster.de/u/tbauer/)
\usepackage{mathrsfs}
\usepackage{amscd}
\usepackage{amsbsy}
\usepackage{verbatim}
\usepackage{moreverb}

\newtheorem{prop}[theorem]{Proposition}
\newtheorem{cor}[theorem]{Corollary}
\newtheorem{conj}[theorem]{Conjecture}


\theoremstyle{definition}
\newtheorem{hw}{Homework}
\newtheorem{exercise*}[exercise]{$\star$ Exercise}

\theoremstyle{remark}
\newtheorem{aside}[theorem]{Aside}

\newcommand{\nn}{\nonumber}
\newcommand{\nid}{\noindent}
\newcommand{\ra}{\rightarrow}
\newcommand{\la}{\leftarrow}
\newcommand{\xra}{\xrightarrow}
\newcommand{\xla}{\xleftarrow}
\newcommand{\tto}{\longrightarrow}

\newcommand{\weq}{\xrightarrow{\sim}}
\newcommand{\cofib}{\rightarrowtail}
\newcommand{\fib}{\twoheadrightarrow}

\newcommand{\IRep}{\mathrm{IRep}}
\newcommand{\IHom}{\mathrm{IHom}}

\def\llarrow{   \hspace{.05cm}\mbox{\,\put(0,-2){$\leftarrow$}\put(0,2){$\leftarrow$}\hspace{.45cm}}}
\def\rrarrow{   \hspace{.05cm}\mbox{\,\put(0,-2){$\rightarrow$}\put(0,2){$\rightarrow$}\hspace{.45cm}}}
\def\lllarrow{  \hspace{.05cm}\mbox{\,\put(0,-3){$\leftarrow$}\put(0,1){$\leftarrow$}\put(0,5){$\leftarrow$}\hspace{.45cm}}}
\def\rrrarrow{  \hspace{.05cm}\mbox{\,\put(0,-3){$\rightarrow$}\put(0,1){$\rightarrow$}\put(0,5){$\rightarrow$}\hspace{.45cm}}}

\def\cA{\mathcal A}\def\cB{\mathcal B}\def\cC{\mathcal C}\def\cD{\mathcal D}
\def\cE{\mathcal E}\def\cF{\mathcal F}\def\cG{\mathcal G}\def\cH{\mathcal H}
\def\cI{\mathcal I}\def\cJ{\mathcal J}\def\cK{\mathcal K}\def\cL{\mathcal L}
\def\cM{\mathcal M}\def\cN{\mathcal N}\def\cO{\mathcal O}\def\cP{\mathcal P}
\def\cQ{\mathcal Q}\def\cR{\mathcal R}\def\cS{\mathcal S}\def\cT{\mathcal T}
\def\cU{\mathcal U}\def\cV{\mathcal V}\def\cW{\mathcal W}\def\cX{\mathcal X}
\def\cY{\mathcal Y}\def\cZ{\mathcal Z}

\def\sA{\mathscr A}\def\cB{\mathcal B}\def\cC{\mathcal C}\def\cD{\mathcal D}
\def\cE{\mathcal E}\def\cF{\mathcal F}\def\sG{\mathscr G}\def\cH{\mathcal H}
\def\cI{\mathcal I}\def\cJ{\mathcal J}\def\cK{\mathcal K}\def\cL{\mathcal L}
\def\cM{\mathcal M}\def\cN{\mathcal N}\def\cO{\mathcal O}\def\cP{\mathcal P}
\def\cQ{\mathcal Q}\def\cR{\mathcal R}\def\cS{\mathcal S}\def\cT{\mathcal T}
\def\cU{\mathcal U}\def\cV{\mathcal V}\def\cW{\mathcal W}\def\cX{\mathcal X}
\def\cY{\mathcal Y}\def\cZ{\mathcal Z}

\def\fG{\mathfrak G}\def\fH{\mathfrak H}
\def\fS{\mathfrak S}\def\fN{\mathfrak N}\def\fX{\mathfrak X}\def\fY{\mathfrak Y}

\def\op{\textrm{op}}\def\ob{\textrm{ob}}

%\def\Iso{\mathcal Iso}\def\cInn{\mathcal Inn}

\def\fg{\mathfrak g}\def\fh{\mathfrak h}\def\fri{\mathfrak i}\def\fp{\mathfrak p}
\def\fA{\mathfrak A}\def\fU{\mathfrak U}

\def\AA{\mathbb A}\def\BB{\mathbb B}\def\CC{\mathbb C}\def\DD{\mathbb D}
\def\EE{\mathbb E}\def\FF{\mathbb F}\def\GG{\mathbb G}\def\HH{\mathbb H}
\def\II{\mathbb I}\def\JJ{\mathbb J}\def\KK{\mathbb K}\def\LL{\mathbb L}
\def\MM{\mathbb M}\def\NN{\mathbb N}\def\OO{\mathbb O}\def\PP{\mathbb P}
\def\QQ{\mathbb Q}\def\RR{\mathbb R}\def\SS{\mathbb S}\def\TT{\mathbb T}
\def\UU{\mathbb U}\def\VV{\mathbb V}\def\WW{\mathbb W}\def\XX{\mathbb X}
\def\YY{\mathbb Y}\def\ZZ{\mathbb Z}

\def\TOP{\mathcal{TOP}}\def\GRP{\mathcal{GRP}}\def\GRPD{\mathcal{GRPD}} \def\CAT{\mathcal{CAT}} \def\SET{\mathcal{SET}}

\def\id{\mathrm{id}}\def\Id{\mathrm{Id}}
\def\inverse{^{-1}}



\begin{document}
    \maketitle
    \begin{abstract}
    {\color{red} Achtung:} Diese Version des Skripts benutze ich zur Bearbeitung! Einige Dinge fehlen, dafür gibt es TODO-Notes. Für Inhalte, benutzt die \href{https://kesslermaximilian.github.io/LectureNotesBonn/2021_Topologie.pdf}{normale Version}
    \end{abstract}
    \newpage
    \listoftodos
    \newpage
    \summaryoflectures
    \newpage
    % start lectures
    \setcounter{section}{0}
    \setcounter{dummy}{0}
    \setcounter{smalldummy}{0}
    \setcounter{figure}{0}
    \setcounter{claim}{0}
    \setcounter{lecture}{0}
    %! TEX root = ./master.tex
\lecture[Metrische Räume. Umgebungen, offene Mengen, Stetigkeit. Topologische Räume. Metrisierbarkeit.]{Di 13 Apr 2021 12:16}{Einführung}
\begin{orga}
\begin{itemize}
\item    Die Vorlesung wird aufgezeichnet.
\item Wir duzen uns.
\item Für die Übungen muss man sich auf eCampus anmelden, ob Do, 20:00 Uhr (Do 15 Apr 2021 20:00 Uhr)
\item Die Übungsblätter werden Donnerstag zur Verfügung gestellt und werden nach 10 Tagen am Montag, 10 Uhr abgegeben.
\item Es wird eine Fragestunde um Donnerstag, 16 Uhr geben.
\item Es wird kein Skript geben, allerdings werden die geschriebenen Notizen auf eCampus zur Verfügung gestellt.
\item Die Vorlesung orientiert sich an der vom letzten Jahr.
\item Für Literatur sind empfohlen: \cite{topology-waldhausen}, \cite{algebraic-topology-hatcher} sowie \cite{topology-and-geometry} (auch auf der Vorlesungshomepage zu finden).
\end{itemize}
\end{orga}

\setcounter{section}{-1}

\section{Motivation und Überblick}
In der Topologie studieren wir topologische Räume. Diese verallgemeinern metrische Räume. Wir wollen zwei metrische Räume $X,Y$ als 'gleich' ansehen, wenn es stetige, zueinander inverse Abbildungen  $X \to  Y, Y\to X$ gibt.
\begin{example}
    Betrachte ein Quadrat und einen Kreis, wir können sie durch Streckung aufeinander abbilden. Gleiches gilt für eine Tasse und einen Donut. \\
    \begin{minipage}{\textwidth}
    \centering
    \begin{minipage}{0.45\textwidth}
     \incfig{quadrat-und-kreis-sind-gleich}
    \end{minipage}
    \begin{minipage}{0.45\textwidth}
    \incfig{tasse-und-donut-sind-gleich}
    \end{minipage}
    \captionof{figure}{Beispiele 'gleicher' metrischer Räume (homöomorph)}
\end{minipage}
\end{example}




\begin{idea}
    Räume sind gewissermaßen aus 'Knete'.
\end{idea}
\begin{goal}
    Wann sind zwei Räume gleich?
\end{goal}
Dazu werden wir algebraische Invarianten verwenden.
\begin{example}
    $\R^n$ und $\R^m$ sind nicht 'gleich' für $n\neq m$.
\end{example}
Der Aufbau ist wie folgt:
\begin{description}
    \item[1. Teil] Grundlagen
    \item[2. Teil] erste Invarianten: Fundamentalgruppe (dazu Überlagerungen)
\end{description}


\newpage
\part{Mengentheoretische Topologie}

\section{Metrische Räume}
\begin{definition}[Metrik]\label{def:metrik}
    Eine \vocab[Metrik]{Metrik} auf einer Menge $X$ ist eine Funktion  $d: X\times X \to  \R_{\geq 0}$ mit folgenden Eigenschaften:
    \begin{enumerate}[(i)]
        \item $d(x,y) = 0 \iff  x = y$
        \item $d(x,y) = d(y,x) \quad \forall x,y\in X$
        \item (Dreiecksungleichung) $d(x,z) \leq  d(x,y) + d(y,z)$.
    \end{enumerate}
    Ein \vocab{Metrischer Raum} ist ein Paar $(X,d)$ aus einer Menge $X$ und einer Metrik $d$ auf $X$.
\end{definition}

\begin{definition}[Stetigkeit]\label{def:stetig-metrischer-raum}
    Seien $(X,d)$ und  $(X',d')$ zwei metrische Räume. Dann ist eine Funktion $f:X \to  Y$ \vocab[Stetig!in $x\in X$]{stetig in $x\in X$}, falls
    \[
        \forall ε > 0 \; \exists \delta > 0 \; \forall x' \colon d(x,x') < \delta \implies d'(f(x), f(x')) < ε
    .\] 
    Eine Funktion $f$ heißt \vocab[Stetig]{stetig}, wenn sie in jedem Punkt  $x\in X$ stetig ist.
    \begin{minipage}{\textwidth}
        \centering
    \incfig{definition-von-stetigkeit-in-metrischen-raeumen}
    \captionof{figure}{Definition von Stetigkeit in metrischen Räumen}
    \end{minipage}
\end{definition}


\begin{example}
    \begin{itemize}
        \item 
    Sei $V$ ein reeller Vektorraum mit Norm  $\lVert \cdot  \rVert$. Dann definiert
    \[
        d(v,w) := \lVert v-w \rVert 
    .\] 
    eine Metrik auf $V$. Insbesondere ist $\R^n$ mit euklidischer Norm
    \[
        \lVert (x_1,\ldots,x_n) \rVert _2 = \sqrt{x_1^2 + \ldots + x_{n}^2} 
    .\] 
    dadurch ein metrischer Raum.
\item Ist $(X,d)$ ein metrischer Raum und  $Y\subset X$ eine Teilmenge, dann ist $(Y, d| _{Y\times Y})$ ein metrischer Raum.
\item Sei $X$ eine Menge. Dann ist
    \[
        d(x,y) = \begin{cases}
            0 & \text{falls } x=y \\
            1 & \text{sonst}
        \end{cases}
    .\] 
    eine Metrik auf $X$, genannt die \vocab[Metrik!diskrete]{diskrete Metrik}.
    \end{itemize}
\end{example}
\begin{notation}
    Sei $X$ ein metrischer Raum. Für  $x\in X$ und $ε>0$ setzen wir
     \[
         U(x,ε) := \left \{y\in X \mid  d(x,y) < ε\right\} 
    .\] 
    und nennen dies den \vocab[Offener $ε$-Ball um  $x$]{offenen $ε$-Ball um $x$}
\end{notation}
\begin{observe}
    Sei $f: (X,d_X) \to  (Y,d_Y)$ eine Funktion, $x\in X$ sowie $ε,δ>0$. Dann sind äquivalent:
\begin{enumerate}[1)]
    \item $\forall x' \in X$ mit $d_X(x',x) < δ$ gilt  $d_Y(f(x'),f(x)) < ε$
    \item Es ist $f(U(x,\delta)) \subset U(f(x),ε)$
    \end{enumerate}
\end{observe}
\begin{definition}[Umgebung]\label{def:umgebung-metrischer-raum}
    Sei $X$ ein metrischer Raum,  $U\subset X$ und $x\in X$. Dann heißt $U$ \vocab[Umgebung]{Umgebung von $x$}, falls ein $ε>0$ existiert, sodass  $U(x,ε) \subset U$. 
\end{definition}
\begin{theorem}[Urbilder von Umgebungen]\label{thm:stetig-gdw-urbild-von-umgebung-ist-umgebung}
    Sei $f:X \to  Y$ eine Abbildung zwischen metrischen Räumen und sei $x\in X$. Dann ist $f$ stetig in  $x$ genau dann, wenn für alle Umgebungen  $V$ um  $f(x)$ in  $Y$ das Urbild  $f^{-1}(V)$ eine Umgebung von $x$ ist.
\end{theorem}

\begin{proof}
'$\implies$' Sei $V$ eine Umgebung von  $f(x)$. Dann  $\exists \; ε>0$ mit $U(f(x),ε) \subset V\}$. Da $f$ stetig ist,  $\exists \; δ>0$, sodass $f(X(x,δ)) \subset U(f(x),ε)\subset V$. Also ist $U(x,δ)\subset f^{-1}(V)$ und somit ist $f^{-1}(V)$ eine Umgebung von $x$. \\
'$\impliedby$'.  Sei $ε>0$. Dann ist  $U(f(x),ε)$ eine Umgebung von  $f(x)$. Also ist  $f^{-1}(U(f(x),ε))$ eine Umgebung von $x$, also  $\exists \; δ>0$ mit $U(x,δ) \subset f^{-1}(U(f(x),ε))$. Also wie gewünscht $f(U(x,δ)) \subset U(f(x),ε))$.
\end{proof}

\begin{definition}[Offene Mengen]\label{def:offene-menge-metrischer-raum} 
    Sei $X$ ein metrischer Raum. Eine Teilmenge  $U\subset X$ heißt \vocab[Metrischer Raum!offene Menge]{offen}, falls sie Umgebung all ihrer Punkte ist, d.h. $\forall x\in U \;\exists ε>0$ mit $U(x,ε)\subset U$.
\end{definition}
\begin{remark}
    $U(x,ε)$ ist offen.
\begin{proof}
    Für alle $y\in U(x,ε)$ ist
    \[
        U(y, \underbrace{ε - d(x,y)}_{>0}) \subset U(x,ε)
    .\] 
    nach der Dreiecksungleichung.
\end{proof}
\end{remark}

\begin{theorem}[Urbilder offener Mengen sind offen]\label{thm:urbild-offener-menge-ist-offen}
    Eine Abbildung $f:X\to Y$ zwischen metrischen Räumen ist stetig genau dann, wenn $\forall U \subset Y \text{  offen}$ auch das Urbild $f^{-1}(U)$ offen in $X$ ist.
\end{theorem}
\begin{proof}
    '$\implies$ '. Sei $U\subset Y$ eine offene Teilmenge und $x\in f^{-1}(U)$ beliebig. Dann ist $f(x) \in U$ und somit ist $U$ eine Umgebung von  $f(x)$. Da  $f$ stetig ist, ist  $f^{-1}(U)$ eine Umgebung von $x$ nach \autoref{thm:stetig-gdw-urbild-von-umgebung-ist-umgebung}. Also ist $f^{-1}(U)$ offen, da $x$ beliebig war.\\
    '$\impliedby$' Sei $x\in X$, $V$ eine Umgebung von  $f(x)$. Dann  $\exists ε>0$ mit $U(f(x),ε)\subset V$. Nach Annahme ist $f^{-1}(U(f(x),ε))$ offen. Also gibt es ein $δ>0$ mit  $U(x,δ) \subset f^{-1}(U(f(x),ε))\subset f^{-1}(V)$. Also ist $f^{-1}(V)$ eine Umgebung von $x$. \\
    Damit ist  $f$ stetig nach \autoref{thm:stetig-gdw-urbild-von-umgebung-ist-umgebung}
\end{proof}

\begin{theorem}[Offene Mengen in metrischen Räumen]\label{thm:offene-mengen-in-metrischem-raum}
    Sei $X$ ein metrischer Raum. Dann gilt:
    \begin{enumerate}[1)]
        \item Die leere Menge $\emptyset$ und $X$ sind offen
        \item  $\forall U_1,\ldots,U_n\subset X$ offen ist auch $\bigcap_{i=1}^n U_i$ offen.
        \item Für jede Familie $\left \{U_i\right\} _{i\in I}$ von offenen Mengen ist auch $\bigcup_{i\in I} U_i$ offen.
    \end{enumerate}
\end{theorem}
\begin{warning}
    Eigenschaft $2)$ gilt nicht für unendliche Schnitte. Es ist $\left( -\frac{1}{n},\frac{1}{n} \right) \subset \R$ offen für alle $n\in \N_{>0}$, allerdings ist dann
    \[
        \bigcap_{n\in \N_{>0}} \left( -\frac{1}{n},\frac{1}{n} \right)  = \left \{0\right\} 
    .\] 
    nicht offen.
\end{warning}

\begin{proof}[Beweis von \autoref{thm:offene-mengen-in-metrischem-raum}]
    \begin{enumerate}[1)]
        \item klar
        \item Sei $x\in \bigcap_{i=1}^n U_i$. $\forall i = 1,\ldots,n$ gibt es nun $ε_i$ mit  $U(x,ε_i)\subset U_i$. Setze $ε := \min \left \{ε_i \mid  i=1,\ldots,n\right\}$. Dann ist
            \[
                U(x,ε) \subset U(x,ε_i) \subset U_i
            .\] 
            für alle $i=1,\ldots,n$ und somit wie gewünscht $U(x,ε) \subset \bigcap_{i=1}^n U_i$
        \item Sei $x\in \bigcup_{\in I} U_i$ beliebig. Dann $\exists i\in I$ mit $x\in U_i$. Da $U_i$ offen ist,  $\exists ε>0$ mit $U(x,ε) \subset U_i$. Also ist $U(x,ε) \subset  \bigcup_{i\in I} U_i$ und somit ist die Vereinigung offen.
    \end{enumerate}
\end{proof}

\section{Topologische Räume} 
    
\begin{definition}[Topologie]\label{def:topologie}
    Eine \vocab{Topologie} auf einer Menge $X$ ist eine Menge  $\mathcal{O}$ von Teilmengen von  $X$, so dass gilt:
    \begin{enumerate}[1)]
        \item $\emptyset,X \in \mathcal{O}$
        \item Für $U_1,\ldots,U_n \in \mathcal{O}$ ist auch $\bigcap_{i=1}^n U_i \in  \mathcal{O}$
        \item Für jede Familie $\left \{U_i\right\} _{i\in I}$ mit $U_i \in \mathcal{O}$ ist auch $\bigcup_{i\in I} U_i \in  \mathcal{O}$
    \end{enumerate}
    Die Mengen in $\mathcal{O}$ heißen \vocab[Menge!offen]{offene Mengen}. \\
    Ein \vocab[Topologischer Raum]{topologischer Raum} ist ein Paar  $(X,\mathcal{O})$ aus einer Menge  $X$ und einer Topologie  $\mathcal{O}$ auf  $X$.
\end{definition}


\begin{definition}[Stetigkeit]\label{def:stetig}
    Seien $X,Y$ topologische Räume. Eine Abbildung  $f:X \to  Y$ heißt \vocab[Stetig]{stetig}, falls für jede offene Teilmenge $U\subset Y$ das Urbild $f^{-1}(U) \subset X$ offen ist.
\end{definition}

\begin{example}
    Sei $(X,d)$ ein metrischer Raum. Dann ist
     \[
         (X, \mathcal{O}) := \left \{U\subset X \mid  U \text{ ist offen bezüglich $d$}\right\} 
    .\] 
    ein topologischer Raum. $\mathcal{O}$ ist die von der Metrik  $d$ \vocab[Topologie!induzierte]{induzierte Topologie}.
\end{example}

\begin{definition}[Metrisierbarkeit]\label{def:metrisierbar}
    Ein topologischer Raum heißt \vocab[Topologischer Raum!metrisierbar]{metrisierbar}, falls die Topologie von einer Metrik induziert ist.
\end{definition}

\begin{example}
    Sei $X$ eine Menge. Die \vocab[Topologie!diskrete]{diskrete Topologie} auf $X$ ist die Menge aller Teilmengen, d.h.  $\mathcal{O} := \mathcal{P}(X)$. Diese ist von der diskreten Metrik
    \[
        d(x,y) = \begin{cases}
            0 & \text{falls }x=y \\
            1 & \text{sonst}
        \end{cases}
    .\] 
    induziert.
\end{example}
\begin{proof}
    Ist $x\in X$, dann ist
    \[
        \left \{x\right\} =U\left(x,\frac{1}{2}\right)
    .\] 
    offen. Ist $U\subset X$ eine Teilmenge, dann ist
    \[
    U = \bigcup_{x\in U} \left \{x\right\}
    .\] 
    offen als Vereinigung offener Mengen.
\end{proof}

\begin{theorem}\label{thm:endlicher-metrisierbarer-raum-ist-diskret}
    Sei $X$ ein endlicher (endlich als Menge), metrisierbarer topologischer Raum. Dann ist  $X$ diskret (d.h. $X$ trägt die diskrete Topologie).
\end{theorem}
\begin{proof}
    Es reicht zu zeigen, dass $\left \{x\right\} $ offen ist $\forall x\in X$. Sei $d$ eine Metrik, die die Topologie induziert, dann wähle
     \[
         ε := \min \left \{d(x,y) \mid  x,y\in X , x\neq y\right\} > 0
    .\] 
    Beachte, dass dies existiert, da $d(x,y) >0$ für  $x\neq y$ und die Menge nach Voraussetzung endlich ist. Nun ist:
    \[
    \left \{x\right\}  = U(x,ε)
    .\] 
    offen und wir sind fertig.
\end{proof}

\begin{example}
    \begin{enumerate}[1)]
        \item Wähle $X = \left \{a,b\right\} $ und setze
            \[
            \mathcal{O} = \left \{\emptyset,X, \left \{a\right\} \right\} 
            .\]
            Dies ist ein topologischer Raum (leicht prüfen), er ist jedoch nicht metrisierbar, da endlich und nicht diskret. Dieser Raum heißt \vocab{Sierpinski-Raum}. 
        \item Sei $X$ eine Menge. Die  \vocab[Topologie!indiskrete]{indiskrete Topologie} auf $X$ enthält nur  $\emptyset,X$. Man prüft leicht, dass dies eine Topologie ist. 
            \begin{itemize}
                \item 
            Hat $X$ mindestens 2 Elemente, so ist  $X$ nicht metrisierbar.
             \begin{proof}
                 Nimm $\abs{X}>2$ an und wähle $x,y\in X$ mit $x\neq y$. Sei $d$ eine Metrik, die die Topologie auf $X$ induziert und setze  $ε := d(x,y)$. Dann ist
                  \[
                      x\in U(x,ε) \quad y\not\in U(x,ε)
                 .\] 
                 also ist $U(x,ε) \neq  \emptyset,X$, Widerspruch.
            \end{proof}
        \item Sei $Y$ ein topologischer Raum. Dann ist  $f: Y \to  X$ stetig für beliebige Abbildungen $f$.
             \begin{proof}
                 Es sind $f^{-1}(\emptyset) = \emptyset$ sowie $f^{-1}(X) = Y$ beide offen.
            \end{proof}
            \end{itemize}
    \end{enumerate}    
\end{example}

\begin{remark}
    Ist $Y$ diskret und  $X$ beliebig, so ist jede Abbildung  $f:Y \to  X$ stetig.
\end{remark}










    %! TEX root = ./master.tex
\lecture[\"Aquivalente Metriken. Abgeschlossene Mengen. Teilraumtopologie. Hom\"oomorphismen. Quotientenräume und -topologie.]{Do 15 Apr 2021 10:14}{Grundbegriffe}


\begin{definition}[Äquivalente Metriken]\label{def:äquivalente-metrik}
    Zwei Metriken $d_1,d_2$ auf $X$ heißen \vocab[Metrik!äquivalente]{äquivalent}, falls Konstanten $c_1,c_2$ existieren, sodass
    \[
        \forall x,y\in X \colon \quad c_1\cdot d_1(x,y) \leq  d_2(x,y) \leq  c_2\cdot d_1(x,y)
    .\] 
\end{definition}
\begin{theorem}\label{thm:äquivalenz-von-metriken-ist-äquivalenzrelation}
    Äquivalenz (von Metriken) ist eine Äquivalenzrelation.
\end{theorem}
\begin{proof}
    \begin{description}
        \item[Reflexivität:] Klar mit $c_1 = c_2 = 1$
        \item[Symmetrie:] Seien $c_1,c_2$ wie in der Definition. Dann gilt mit entsprechender Division, dass
            \[
                \forall x,y \in X \colon : \quad \frac{1}{c_2}\cdot d_2(x,y) \leq  d_1(x,y) \leq  \frac{1}{c_1}d_2(x,y)
            .\] 
        \item[Transitivität:]. Seien $c_1,c_2,c_1',c_2'$ gewählt, sodass $\forall x \; \forall y\colon c_1d_1 (x,y) \leq  d_2 (x,y)\leq  c_2d_1(x,y)$ sowie $c_1'd_2 (x,y)\leq  d_3(x,y) \leq  c_2'd_2(x,y)$ (Also $d_1 \sim  d_2$ und $d_2 \sim d_3$). Dann ist auch
            \[
\forall x \; \forall y \colon \quad                c_1c_1'd_1(x,y) \leq  c_1'd_2(x,y)\leq d_3(x,y) \leq  c_2'd_2 (x,y) \leq  c_2d_1(x,y)
            .\] 
    \end{description}
\end{proof}

\begin{theorem}\label{thm:äquivalente-metriken-erzeugen-dieselbe-topologie}
    Äquivalente Metriken induzieren dieselbe Topologie.
\end{theorem}
\begin{proof}
    Wegen der Symmetrie genügt es zu zeigen, dass Mengen, die offen bezüglich $d_2$ sind, auch offen bezüglich $d_1$ sind. \\
    Sei nun $U\subset X$ offen bezüglich $d_2$ und $x\in U$. Dann existiert ein $ε>0$ mit  $U_{d_2}(x,ε) \subset U$. Ist nun $d_1(x,y) < \frac{ε}{c_2}$, dann ist
    \[
        d_2(x,y) \leq  c_2d_1(x,y) < ε
    .\] 
und damit ist
\[
    U_{d_1}\left(x,\frac{ε}{c_2}\right) \subset U_{d_2} \left( x,ε \right) \subset U
.\] 
und somit ist $U$ auch offen bezüglich  $d_1$.
\end{proof}

\begin{remark}
    Es gibt auch nicht-äquivalente Metriken, die die gleiche Topologie induzieren. Siehe hierzu \autoref{aufgabe-1.2}.
\end{remark}
\begin{remark}
    Je zwei Normen auf $\R^n$ sind äquivalent, induzieren also dieselbe Topologie, das beweisen wir jedoch hier nicht.
\end{remark}

\begin{definition}[Umgebung]\label{def:umgebung}
    Sei $X$ ein topologischer Raum und $U\subset X$ sowie $x\in X$. Dann heißt $U$ \vocab[Umgebung!von $x$]{Umgebung von $x$}, falls es eine offene Teilmenge  $O\subset X$ gibt, mit $x\in O\subset U$.
\end{definition}
\begin{remark}
    Für metrische Räume stimmt dies mit der vorherigen Definiton überein.
\end{remark}


\begin{theorem}\label{thm:offene-menge-ist-umgebung-all-ihrer-punkte}
    Sei $X$ ein topologischer Raum und  $U\subset X$. Dann sind äquivalent:
    \begin{enumerate}[1)]
        \item $U$ ist offen.
        \item $U$ ist Umgebung aller ihrer Punkte.
    \end{enumerate}
\end{theorem}
\begin{proof}
    '$1) \implies 2)$' ist klar, wähle einfach $O = U$. \\
    '$2)\implies_1)$'. Für jedes $x\in U$ existiert also $U_x$ mit  $x\in U_x \subset U$. Dann ist aber
    \[
    U = \bigcup_{x\in U} U_x
    .\] 
    offen als Vereinigung offener Mengen.
\end{proof}

\begin{definition*}[Abgeschlossene Mengen]\label{def:abgeschlossene-menge}
    Sei $X$ ein topologischer Raum. Eine Teilmenge  $A\subset X$ heißt \vocab[Menge!abgeschlossen]{abgeschlossen}, falls ihr Komplement $X \setminus A = \left \{x\in X \mid  x\not\in A\right\} $ offen ist.
\end{definition*}
\begin{remark}
    Für metrische Räume stimmt das mit dem Begriff aus der Analysis überein.
\end{remark}

\begin{theorem}[Dualität]\label{offen-abgeschlossen-ist-dual}
    Ein topologischer Raum lässt sich auch über seine abgeschlossenen Mengen charakterisieren. Diese müssen erfüllen:
    \begin{enumerate}[i)]
        \item $\emptyset,X$ sind abgeschlossen
        \item Für $A_1,\ldots,A_n$ abgeschlossen ist auch $A_1\cup \ldots \cup A_n$ abgeschlossen.
        \item Für eine Familie $\left \{A_i\right\} _{\in I}$ abgeschlossener Mengen ist auch
            \[
            \bigcap_{i\in I} A_i
            .\] 
            abgeschlossen.
    \end{enumerate}
\end{theorem}

\begin{recap}
    Wenn wir von einer Familie von Mengen $\left \{A_i\right\} _{\in I}$ sprechen, meinen wir, dass $I$ eine Menge ist, und für jedes $\in I$ ist $A_i$ eine Teilmenge von  $X$. Formal können wir dies als eine Funktion  $I \to  \mathcal{P}(X)$ darstellen.
\end{recap}
\begin{proof}[Beweis von \autoref{offen-abgeschlossen-ist-dual}]
    \begin{enumerate}[i)]
        \item $X \setminus \emptyset = X$, $X\setminus X = \emptyset$ sind abgeschlossen.
        \item  \[
                \underbrace{\bigcap_{i=1}^n (X\setminus A_i)}_{\text{offen}} = X \setminus  \bigcup_{i=1}^n A_i \quad \implies \bigcup_{i=1}^n A_i \text{ abgeschlossen}
        .\] 
        \item \[
                \underbrace{\bigcup_{i\in I} (\underbrace{X\setminus A_i}_{\text{offen}})}_{\text{offen}} = X \setminus \bigcap_{i \in I} A_i \quad \implies \bigcap_{i \in I} A_i \text{ abgeschlossen}
        .\] 
    \end{enumerate}
\end{proof}

\begin{theorem}[Stetigkeit mit abgeschlossenen Mengen]\label{thm:stetig-gdw-urbilder-abgeschlossener-mengen-sind-abgeschlossen}
    Sei $f:X \to  Y$ eine Funktion zwischen topologischen Räumen. Dann sind äuqivalent:
    \begin{enumerate}[1)]
        \item $f$ ist stetig
        \item $\forall U\subset Y$ offen ist $f^{-1}(U) \subset X$ offen
        \item  $\forall A\subset Y$ abgeschlossen ist $f^{-1}(A)$ abgeschlossen
    \end{enumerate}
\end{theorem}
\begin{proof}
    \begin{equation*}
        \begin{split}
            f \text{ stetig} &\iff \forall U \subset  Y \text{ offen ist } f^{-1}(U) \text{ offen}  \\
                             &\iff  \forall A \subset Y \text{ abgeschlossen ist } f^{-1}(Y \setminus A) \text{ offen} \\
                             &\iff \forall A\subset Y \text{ abgeschlossen ist } X \setminus f^{-1}(A) \text{ offen} \\
                             &\iff  \forall A\subset Y \text{ abgeschlossen ist } f^{-1}(A) \text{ abgeschlossen}
        \end{split}
    \end{equation*}
\end{proof}


Wir erinnern uns: Ist $(X,d)$ ein metrischer Raum, so auch  $\left(Y, d_{Y\times Y}\right) \quad \forall Y\subset X$. Wie ist dies für topologische Räume?
\begin{warning}
    $(Y, \mathcal{O}_X \cap \mathcal{P}(Y))$ ist im allgemeinen \textbf{kein} topologischer Raum. (wenn $Y$ nicht offen ist, denn dann ist $Y\not\in \mathcal{S}_X \cap \mathcal{P}(X)$)
\end{warning}
\begin{theoremdef}[Teilraumtopologie]\label{def:teilraumtopologie}
    Sei $X$ ein topologischer Raum,  $Y\subset X$. Dann ist
    \[
    \mathcal{O}_Y := \left \{U \cap Y \mid  U\subset X \text{ offen}\right\} 
    .\] 
    eine Topologie auf $Y$, die  \vocab[Topologie!Teilraum-]{Teilraumtopologie} oder auch \vocab[Topologie!Unterraum-]{Unterraumtopologie} genannt wird.  
\end{theoremdef}

\begin{example}
    Betrachte $\R^1 \subset \R^2$ als Unterraum. Schneiden wir eine offene Menge in $\R^2$ mit $\R^1$, so erhalten wir ein offenes Intervall: \\
\begin{minipage}{\textwidth}
\centering    
    \incfig{r1-als-unterraum-von-r2}
    \captionof{figure}{$\R^1$ als Unterraum von $\R^2$}
\end{minipage}
\end{example}
\begin{proof}
    \begin{itemize}
        \item Es sind $\emptyset = \emptyset \cap Y$ und $Y = X \cap Y$ offen.
        \item Es ist 
            \[
                \bigcap_{i=1}^n \left( U_i \cap Y \right)  = \left( \bigcap_{i=1}^n U_i \right)  \cap Y
            .\] 
        \item Es ist
            \[
                \bigcup_{i\in I} \left( U_i \cap Y \right) = \left( \bigcup_{i \in  I} U_i \right) \cap Y
            .\] 
    \end{itemize}
\end{proof}
\begin{warning}
    Für $Z\subset Y\subset X$ muss man zwischen 'offen in $Y$' und  'offen in  $X$' unterscheiden, falls  $Y$ nicht offen ist.
\end{warning}
\begin{remark*}
    Ist $Y\subset X$ offen, so stimmen die beiden vorherigen Konzepte tatsächlich überein, d.h. eine Menge $Z\subset Y$ ist offen in $Y$, genau dann, wenn sie offen in  $X$ ist.
\end{remark*}

\begin{remark}
    Sei $(X,d)$ ein metrischer Raum und  $Y\subset X$ eine Teilmenge. Die Unterraumtopologie auf $Y$ bzgl. der Topologie auf  $X$ ist gleich der Topologie indzuziert von der eingeschränkten Metrik.
\end{remark}
\begin{proof}
    Für $y\in Y$ ist
    \[
        U_{d\mid _{Y\times Y}} (y,ε) = U_d(y,ε)\cap Y
    .\] 
    , deswegen werden von beiden Metriken die gleichen offenen Mengen induziert.
\end{proof}



\begin{example}
    Der \vocab{Einheitskreis} als Unterraum von $\R^2$:
    \[
    \left \{x\in \R^2 \mid  \lVert x \rVert _2 = 1\right\}  =: S^1 \subset \R^2
    .\] 
    Genauso gibt es die  \vocab{$n$-Sphäre} definiert durch
    \[
    \left \{x\in \R^{n+1}\mid  \lVert x \rVert _2 = 1\right\} =: S^n \subset \R^{n+1}
    .\] 
\end{example}

\begin{definition}[Homöomorphie]\label{def:homöomorph}
    Eine Abbildung $f: X \to  Y$ zwischen topologischen Räumen heißt \vocab{Homöomorphismus}, falls $f$ stetig und bijektiv ist und  auch $f^{-1}: Y \to  X$ stetig ist.  \\
    Existiert solch ein $f$, so heißen  $X,Y$  \vocab[Homöomorphismus!homöomorph]{homöomorph} 
\end{definition}
\begin{example}
    Die Räume $(\R^2, \lVert \cdot  \rVert _2)$ und $(\C, \abs{\cdot })$ sind homöomorph mittels der Abbildung
        \begin{equation*}
        \begin{array}{c c l} 
            \R^2 & \longrightarrow & \C \\
            (x,y) & \longmapsto &  x+iy
        \end{array}
    \end{equation*}
\end{example}
\begin{warning}
    Nicht jede stetige Bijketion ist ein Homöomorphismus.
\end{warning}
\begin{example}
    Betrachte für eine Menge $X$ die Identitätsabbildung  $(X, \mathcal{P}(X)) \stackrel{\id_X}{\to} (X, \left \{\emptyset,X\right\})$ von der diskreten in die indiskrete Topologie. Diese ist stetig, aber die Umkehrabbildung nicht (falls $\abs{X} \geq 2$).
\end{example}

\section{Quotientenräume} 
\begin{ddefinition}[Äquivalenzklasse]
Sei $\sim $ eine Äquivalenzrelation auf $X$. Für  $x\in X$ definieren wir die \vocab{Äquivalenzklasse} $[x]$ von $x$ durch:
 \[
     [x] := \left \{x' \in X \mid  x\sim x'\right\} 
.\] 
Wir setzen
\[
X / \sim  := \left \{[x] \mid  x\in X\right\} 
.\] 
als die \vocab[Äquivalenzklasse!Menge der]{Menge der Äquivalenzklassen} von $X$ bezüglich  $\sim $. Definiere nun
    \begin{equation*}
    q: \left| \begin{array}{c c l} 
    X & \longrightarrow & X / \sim  \\
    x & \longmapsto &  [x]
    \end{array} \right.
\end{equation*}
als die \vocab[Projektion!kanonische]{kanonische Projektion} von $X$ auf seine Äquivalenzklassen.
\end{ddefinition}
\begin{fact}
   Wir stellen fest, dass $q$ surjektiv ist.
\end{fact}

 \begin{recap}
     Für eine Surjektion $f: X \to  Y$ ist $x\sim y :\iff  f(x) = f(y)$ eine Äquivalenzrelation auf $X$ und
             \begin{equation*}
             \begin{array}{c c l} 
             X / \sim  & \longrightarrow & Y \\
             \left[x\right] & \longmapsto &  f(x)
             \end{array}
         \end{equation*}
ist eine Bijektion, wir erhalten also eine Korrespondenz zwischen Äquivalenzrelationen und surjektiven Abbildungen aus $X$.
\end{recap}

\begin{theoremdef}[Quotiententopologie]\label{def:quotiententopologie}
    Sei $X$ ein topologischer Raum und  $\sim $ eine Äquivalenzrelation auf $X$. Sei  $q: X \to  X / \sim $ die kanonische Projektion. Dann definiert
    \[
        \mathcal{O}_{X / \sim } := \left \{U\subset X / \sim \mid  q^{-1}(U) \subset X \text{ offen}\right\} 
    .\] 
    eine Topologie auf $X / \sim $, genannt die \vocab[Topologie!Quotienten-]{Quotiententopologie}. 
\end{theoremdef}
\begin{proof}
    Wir prüfen die Axiome einer Topologie:
    \begin{itemize}
        \item Es ist $q^{-1}(\emptyset) = \emptyset$ und $q^{-1}(X / \sim ) = X$, also sind beide Mengen offen.
        \item Sind $U_1, \ldots,U_n\subset X / \sim $ offen, so ist
            \[
                q^{-1}(U_1\cap \ldots \cap A_n) = q^{-1}(U_i) \cap \ldots \cap q^{-1}(U_n)
            .\] 
            offen in $X$, also ist  $U_1\cap \ldots \cap U_n$ offen in $X / \sim $ nach Definition.
        \item Ist $\left \{U_i\right\} _{i\in I}$ eine Familie offener Teilmengen von $X / \sim $, dann ist
            \[
                q^{-1}\left( \bigcup_{i \in  I} U_i \right) = \bigcup_{i \in I} q^{-1}(U_i)
            .\] 
            offen in $X$, also ist  $\bigcup_{i \in I} U_i$ offen in $X / \sim $ nach Definition.
    \end{itemize}
\end{proof}






\begin{remark}
    Die Projektion $q: X \to  X / \sim $ ist stetig und die Quotiententopologie ist maximal (bezüglich Inklusion, lies: 'am feinsten') unter allen Topologien auf $X / \sim $, für die $q$ stetig ist.
\end{remark}
\begin{theorem}[Universelle Eigenschaft der Quotiententopologie]\label{thm:universelle-eigenschaft-der-quotiententopologie}
    Sei $f : X \to  Y$ stetig und $q : X \to  X / \sim $ die kanonische Projektion. Angenommen, es existiert $g : X / \sim \to  Y$ mit $f = g \circ  q$. Dann ist $g$ stetig und in diesem Fall ist $g$ eindeutig. \\
    \begin{minipage}{\textwidth}
    \centering    
     \begin{tikzcd}
         X \ar{r}{f} \ar[swap]{d}{q} & Y  \\
                                      X / \sim \ar[dotted,swap]{ur}{g}
    \end{tikzcd}
    \end{minipage}
\end{theorem}
\begin{remark}
    $g$ existiert genau dann, wenn  $x \sim x' \implies f(x) = f(x')$
\end{remark}
\begin{trivial*}
    Das ist eine universelle Eigenschaft im Sinne der Kategorientheorie, d.h. für einen Raum $X$ und eine Äquivalenzrelation existiert bis auf eindeutigen Isomorphismus stets genau ein topologischer Raum $(X / \sim  , \mathcal{S})$ zusammen mit einer stetigen Abbildung $q : X \to  X / \sim $, sodass $x\sim x' \implies q(x) = q(x')$, sodass das Tripel $(X, X / \sim , q)$ obige Eigenschaft hat. Wir können also obige Eigenschaft auch als Definition der Quotiententopologie verwenden, und aus dieser folgt auch die Eindeutigkeit. Existenz haben wir mit unserer vorherigen Definition gezeigt.
\end{trivial*}

\begin{proof}[Beweis von \autoref{aufgabe-3.2}]
    Sei $U\subset Y$ offen. Dann ist
    \[
        q^{-1}(g^{-1}(U)) \stackrel{f = g \circ  q}{=} f^{-1}(U)
    .\] 
    offen, weil $f$ stetig ist. Also ist  $g^{-1}(U)$ offen per Definition ($g^{-1}(U)$ ist nach Definition genau dann offen in $X / \sim $, wenn $q^{-1}(g^{-1}(U))$ offen in $X$ ist) und somit ist $g$ stetig. 
\end{proof}
\begin{dexample}
Sei $X = [0,1]\subset \R$ das \vocab{Einheitsintervall} (mit der Unterraumtopologie bezüglich $\R$) mit der Äquivalenzrelation erzeugt von $0\sim 1$. Wir 'identifizieren' also die Punkte $\left \{0\right\} ,\left \{1\right\} $ miteinander.
\end{dexample}
\begin{theorem}[Kreishomöomorphie]\label{thm:kreis-ist-quotientenraum-von-einheitsintervall}
    Der Quotientenraum $[0,1] / (0\sim 1)$ ist homöomorph zu $S_1$.
\end{theorem}



    % end lectures
    %\input{fragestunden.tex}
\end{document}
