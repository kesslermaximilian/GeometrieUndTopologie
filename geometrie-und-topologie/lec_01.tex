\lecture[Metrische Räume. Umgebungen, offene Mengen, Stetigkeit. Topologische Räume. Metrisierbarkeit.]{Di 13 Apr 2021 12:16}{Einführung}
\section*{Organisatorisches}
\begin{itemize}
\item    Die Vorlesung wird aufgezeichnet.
\item Wir duzen uns.
\item Für die Übungen muss man sich auf eCampus anmelden, ob Do, 20:00 Uhr (Do 15 Apr 2021 20:00 Uhr)
\item Die Übungsblätter werden Donnerstag zur Verfügung gestellt und werden nach 10 Tagen am Montag, 10 Uhr abgegeben.
\item Es wird eine Fragestunde um Donnerstag, 16 Uhr geben.
\item Es wird kein Skript geben, allerdings werden die geschriebenen Notizen auf eCampus zur Verfügung gestellt.
\item Die Vorlesung orientiert sich an der vom letzten Jahr.
\item Für Literatur siehe die 3 Vorschläge auf der Homepage
\end{itemize}

\setcounter{section}{-1}

\section{Motivation und Überblick}
In der Topologie studieren wir topologische Räume. Diese verallgemeinern metrische Räume. Wir wollen zwei metrische Räume $X,Y$ als 'gleich' ansehen, wenn es stetige, zueinander inverse Abbildungen  $X \to  Y, Y\to X$ gibt.
\begin{example}
    Betrachte ein Quadrat und einen Kreis, wir können sie durch Streckung aufeinander abbilden. Gleiches gilt für eine Tasse und einen Donut. \\
    \begin{minipage}{\textwidth}
    \centering
    \begin{minipage}{0.45\textwidth}
     \incfig{quadrat-und-kreis-sind-gleich}
    \end{minipage}
    \begin{minipage}{0.45\textwidth}
    \incfig{tasse-und-donut-sind-gleich}
    \end{minipage}
    \captionof{figure}{Beispiele 'gleicher' metrischer Räume (homöomorph)}
\end{minipage}
\end{example}




\begin{idea}
    Räume sind gewissermaßen aus 'Knete'.
\end{idea}
\begin{goal}
    Wann sind zwei Räume gleich?
\end{goal}
Dazu werden wir algebraische Invarianten verwenden.
\begin{example}
    $\R^n$ und $\R^m$ sind nicht 'gleich' für $n\neq m$.
\end{example}
Der Aufbau ist wie folgt:
\begin{description}
    \item[1. Teil] Grundlagen
    \item[2. Teil] erste Invarianten: Fundamentalgruppe (dazu Überlagerungen)
\end{description}

\section{Metrische Räume}
\begin{definition}[Metrik]\label{def:metrik}
    Eine \vocab[Metrik]{Metrik} auf einer Menge $X$ ist eine Funktion  $d: X\times X \to  \R_{\geq 0}$ mit folgenden Eigenschaften:
    \begin{enumerate}[(i)]
        \item $d(x,y) = 0 \iff  x = y$
        \item $d(x,y) = d(y,x) \quad \forall x,y\in X$
        \item (Dreiecksungleichung) $d(x,z) \leq  d(x,y) + d(y,z)$.
    \end{enumerate}
    Ein \vocab{Metrischer Raum} ist ein Paar $(X,d)$ aus einer Menge $X$ und einer Metrik $d$ auf $X$.
\end{definition}

\begin{definition}[Stetigkeit]\label{def:stetig-metrischer-raum}
    Seien $(X,d)$ und  $(X',d')$ zwei metrische Räume. Dann ist eine Funktion $f:X \to  Y$ \vocab[Stetig!in $x\in X$]{stetig in $x\in X$}, falls
    \[
        \forall ε > 0 \; \exists \delta > 0 \; \forall x' \colon d(x,x') < \delta \implies d'(f(x), f(x')) < ε
    .\] 
    Eine Funktion $f$ heißt \vocab[Stetig]{stetig}, wenn sie in jedem Punkt  $x\in X$ stetig ist.
    \begin{minipage}{\textwidth}
        \centering
    \incfig{definition-von-stetigkeit-in-metrischen-raeumen}
    \captionof{figure}{Definition von Stetigkeit in metrischen Räumen}
    \end{minipage}
\end{definition}


\begin{example}
    \begin{itemize}
        \item 
    Sei $V$ ein reeller Vektorraum mit Norm  $\lVert \cdot  \rVert$. Dann definiert
    \[
        d(v,w) := \lVert v-w \rVert 
    .\] 
    eine Metrik auf $V$. Insbesondere ist $\R^n$ mit euklidischer Norm
    \[
        \lVert (x_1,\ldots,x_n) \rVert _2 = \sqrt{x_1^2 + \ldots + x_{n}^2} 
    .\] 
    dadurch ein metrischer Raum.
\item Ist $(X,d)$ ein metrischer Raum und  $Y\subset X$ eine Teilmenge, dann ist $(Y, d| _{Y\times Y})$ ein metrischer Raum.
\item Sei $X$ eine Menge. Dann ist
    \[
        d(x,y) = \begin{cases}
            0 & \text{falls } x=y \\
            1 & \text{sonst}
        \end{cases}
    .\] 
    eine Metrik auf $X$, genannt die \vocab[Diskrete Metrik]{diskrete Metrik}.
    \end{itemize}
\end{example}
\begin{notation}
    Sei $X$ ein metrischer Raum. Für  $x\in X$ und $ε>0$ setzen wir
     \[
         U(x,ε) := \left \{y\in X \mid  d(x,y) < ε\right\} 
    .\] 
    und nennen dies den \vocab[Offener $ε$-Ball um  $x$]{offenen $ε$-Ball um $x$}
\end{notation}
\begin{observe}
    Sei $f: (X,d_X) \to  (Y,d_Y)$ eine Funktion, $x\in X$ sowie $ε,δ>0$. Dann sind äquivalent:
\begin{enumerate}[1)]
    \item $\forall x' \in X$ mit $d_X(x',x) < δ$ gilt  $d_Y(f(x'),f(x)) < ε$
    \item Es ist $f(U(x,\delta)) \subset U(f(x),ε)$
    \end{enumerate}
\end{observe}
\begin{definition}[Umgebung]\label{def:umgebung-metrischer-raum}
    Sei $X$ ein metrischer Raum,  $U\subset X$ und $x\in X$. Dann heißt $U$ \vocab[Umgebung]{Umgebung von $x$}, falls ein $ε>0$ existiert, sodass  $U(x,ε) \subset U$. 
\end{definition}
\begin{theorem}[Urbilder von Umgebungen]\label{thm:stetig-gdw-urbild-von-umgebung-ist-umgebung}
    Sei $f:X \to  Y$ eine Abbildung zwischen metrischen Räumen und sei $x\in X$. Dann ist $f$ stetig in  $x$ genau dann, wenn für alle Umgebungen  $V$ um  $f(x)$ in  $Y$ das Urbild  $f^{-1}(V)$ eine Umgebung von $x$ ist.
\end{theorem}

\begin{proof}
'$\implies$' Sei $V$ eine Umgebung von  $f(x)$. Dann  $\exists \; ε>0$ mit $U(f(x),ε) \subset V\}$. Da $f$ stetig ist,  $\exists \; δ>0$, sodass $f(X(x,δ)) \subset U(f(x),ε)\subset V$. Also ist $U(x,δ)\subset f^{-1}(V)$ und somit ist $f^{-1}(V)$ eine Umgebung von $x$. \\
'$\impliedby$'.  Sei $ε>0$. Dann ist  $U(f(x),ε)$ eine Umgebung von  $f(x)$. Also ist  $f^{-1}(U(f(x),ε))$ eine Umgebung von $x$, also  $\exists \; δ>0$ mit $U(x,δ) \subset f^{-1}(U(f(x),ε))$. Also wie gewünscht $f(U(x,δ)) \subset U(f(x),ε))$.
\end{proof}

\begin{definition}[Offene Mengen]\label{def:offene-menge-metrischer-raum} 
    Sei $X$ ein metrischer Raum. Eine Teilmenge  $U\subset X$ heißt \vocab[Menge!offen]{offen}, falls sie Umgebung all ihrer Punkte ist, d.h. $\forall x\in U \;\exists ε>0$ mit $U(x,ε)\subset U$.
\end{definition}
\begin{remark}
    $U(x,ε)$ ist offen.
\end{remark}
\begin{proof}
    Für alle $y\in U(x,ε)$ ist
    \[
        U(y, \underbrace{ε - d(x,y)}_{>0}) \subset U(x,ε)
    .\] 
    nach der Dreiecksungleichung.
\end{proof}

\begin{theorem}[Urbilder offener Mengen sind offen]\label{thm:urbild-offener-menge-ist-offen}
    Eine Abbildung $f:X\to Y$ zwischen metrischen Räumen ist stetig genau dann, wenn $\forall U \subset Y \text{  offen}$ auch das Urbild $f^{-1}(U)$ offen in $X$ ist.
\end{theorem}
\begin{proof}
    '$\implies$ '. Sei $U\subset Y$ eine offene Teilmenge und $x\in f^{-1}(U)$ beliebig. Dann ist $f(x) \in U$ und somit ist $U$ eine Umgebung von  $f(x)$. Da  $f$ stetig ist, ist  $f^{-1}(U)$ eine Umgebung von $x$ nach \autoref{thm:stetig-gdw-urbild-von-umgebung-ist-umgebung}. Also ist $f^{-1}(U)$ offen, da $x$ beliebig war.\\
    '$\impliedby$' Sei $x\in X$, $V$ eine Umgebung von  $f(x)$. Dann  $\exists ε>0$ mit $U(f(x),ε)\subset V$. Nach Annahme ist $f^{-1}(U(f(x),ε))$ offen. Also gibt es ein $δ>0$ mit  $U(x,δ) \subset f^{-1}(U(f(x),ε))\subset f^{-1}(V)$. Also ist $f^{-1}(V)$ eine Umgebung von $x$. \\
    Damit ist  $f$ stetig nach \autoref{thm:stetig-gdw-urbild-von-umgebung-ist-umgebung}
\end{proof}

\begin{theorem}[Offene Mengen in metrischen Räumen]\label{thm:offene-mengen-in-metrischem-raum}
    Sei $X$ ein metrischer Raum. Dann gilt:
    \begin{enumerate}[1)]
        \item Die leere Menge $\emptyset$ und $X$ sind offen
        \item  $\forall U_1,\ldots,U_n\subset X$ offen ist auch $\bigcap_{i=1}^n U_i$ offen.
        \item Für jede Familie $\left \{U_i\right\} _{i\in I}$ von offenen Mengen ist auch $\bigcup_{i\in I} U_i$ offen.
    \end{enumerate}
\end{theorem}
\begin{warning}
    Eigenschaft $2)$ gilt nicht für unendliche Schnitte. Es ist $\left( -\frac{1}{n},\frac{1}{n} \right) \subset \R$ offen für alle $n\in \N_{>0}$, allerdings ist dann
    \[
        \bigcap_{n\in \N_{>0}} \left( -\frac{1}{n},\frac{1}{n} \right)  = \left \{0\right\} 
    .\] 
    nicht offen.
\end{warning}

\begin{proof}[Beweis von \autoref{thm:offene-mengen-in-metrischem-raum}]
    \begin{enumerate}[1)]
        \item klar
        \item Sei $x\in \bigcap_{i=1}^n U_i$. $\forall i = 1,\ldots,n$ gibt es nun $ε_i$ mit  $U(x,ε_i)\subset U_i$. Setze $ε := \min \left \{ε_i \mid  i=1,\ldots,n\right\}$. Dann ist
            \[
                U(x,ε) \subset U(x,ε_i) \subset U_i
            .\] 
            für alle $i=1,\ldots,n$ und somit wie gewünscht $U(x,ε) \subset \bigcap_{i=1}^n U_i$
        \item Sei $x\in \bigcup_{\in I} U_i$ beliebig. Dann $\exists i\in I$ mit $x\in U_i$. Da $U_i$ offen ist,  $\exists ε>0$ mit $U(x,ε) \subset U_i$. Also ist $U(x,ε) \subset  \bigcup_{i\in I} U_i$ und somit ist die Vereinigung offen.
    \end{enumerate}
\end{proof}

\section{Topologische Räume} 
    
\begin{definition}[Topologie]\label{def:topologie}
    Eine \vocab{Topologie} auf einer Menge $X$ ist eine Menge  $\mathcal{O}$ von Teilmengen von  $X$, so dass gilt:
    \begin{enumerate}[1)]
        \item $\emptyset,X \in \mathcal{O}$
        \item Für $U_1,\ldots,U_n \in \mathcal{O}$ ist auch $\bigcap_{i=1}^n U_i \in  \mathcal{O}$
        \item Für jede Familie $\left \{U_i\right\} _{i\in I}$ mit $U_i \in \mathcal{O}$ ist auch $\bigcup_{i\in I} U_i \in  \mathcal{O}$
    \end{enumerate}
    Die Mengen in $\mathcal{O}$ heißen \vocab[Offene Menge]{offene Mengen}. \\
    Ein \vocab[Topologischer Raum]{topologischer Raum} ist ein Paar  $(X,\mathcal{O})$ aus einer Menge  $X$ und einer Topologie  $\mathcal{O}$ auf  $X$.
\end{definition}


\begin{definition}[Stetigkeit]\label{def:stetig}
    Seien $X,Y$ topologische Räume. Eine Abbildung  $f:X \to  Y$ heißt \vocab[Stetig]{stetig}, falls für jede offene Teilmenge $U\subset Y$ das Urbild $f^{-1}(U) \subset X$ offen ist.
\end{definition}

\begin{example}
    Sei $(X,d)$ ein metrischer Raum. Dann ist
     \[
         (X, \mathcal{O} := \left \{U\subset X \mid  U \text{ ist offen bezüglich $d$}\right\} )
    .\] 
    ein topologischer Raum. $\mathcal{O}$ ist die von der Metrik  $d$ \vocab[Topologie!induzierte]{induzierte Topologie}.
\end{example}

\begin{definition}[Metrisierbarkeit]\label{def:metrisierbar}
    Ein topologischer Raum heißt \vocab[Metrisierbar]{metrisierbar}, falls die Topologie von einer Metrik induziert ist.
\end{definition}

\begin{example}
    Sei $X$ eine Menge. Die \vocab[Diskrete Topologie]{diskrete Topologie} auf $X$ ist die Menge aller Teilmengen, d.h.  $\mathcal{O} := \mathcal{P}(X)$. Diese ist von der diskreten Metrik
    \[
        d(x,y) = \begin{cases}
            0 & \text{falls }x=y \\
            1 & \text{sonst}
        \end{cases}
    .\] 
    induziert.
\end{example}
\begin{proof}
    Ist $x\in X$, dann ist
    \[
        \left \{x\right\} =U\left(x,\frac{1}{2}\right)
    .\] 
    offen. Ist $U\subset X$ eine Teilmenge, dann ist
    \[
    U = \bigcup_{x\in U} \left \{x\right\}
    .\] 
    offen als Vereinigung offener Mengen.
\end{proof}

\begin{theorem}\label{thm:endlicher-metrisierbarer-raum-ist-diskret}
    Sei $X$ ein endlicher (endlich als Menge), metrisierbarer topologischer Raum. Dann ist  $X$ diskret (d.h. $X$ trägt die diskrete Topologie).
\end{theorem}
\begin{proof}
    Es reicht zu zeigen, dass $\left \{x\right\} $ offen ist $\forall x\in X$. Sei $d$ eine Metrik, die die Topologie induziert, dann wähle
     \[
         ε := \min \left \{d(x,y) \mid  x,y\in X , x\neq y\right\} > 0
    .\] 
    Beachte, dass dies existiert, da $d(x,y) >0$ für  $x\neq y$ und die Menge nach Voraussetzung endlich ist. Nun ist:
    \[
    \left \{x\right\}  = U(x,ε)
    .\] 
    offen und wir sind fertig.
\end{proof}

\begin{example}
    \begin{enumerate}[1)]
        \item Wähle $X = \left \{a,b\right\} $ und setze
            \[
            \mathcal{O} = \left \{\emptyset,X, \left \{a\right\} \right\} 
            .\]
            Dies ist ein topologischer Raum (leicht prüfen), er ist jedoch nicht metrisierbar, da endlich und nicht diskret. Dieser Raum heißt \vocab{Sierpinski-Raum}. 
        \item Sei $X$ eine Menge. Die  \vocab[Topologie!indiskrete]{indiskrete Topologie} auf $X$ enthält nur  $\emptyset,X$. Man prüft leicht, dass dies eine Topologie ist. 
            \begin{itemize}
                \item 
            Hat $X$ mindestens 2 Elemente, so ist  $X$ nicht metrisierbar.
             \begin{proof}
                 Nimm $\abs{X}>2$ an und wähle $x,y\in X$ mit $x\neq y$. Sei $d$ eine Metrik, die die Topologie auf $X$ induziert und setze  $ε := d(x,y)$. Dann ist
                  \[
                      x\in U(x,ε) \quad y\not\in U(x,ε)
                 .\] 
                 also ist $U(x,ε) \neq  \emptyset,X$, Widerspruch.
            \end{proof}
        \item Sei $Y$ ein topologischer Raum. Dann ist  $f: Y \to  X$ stetig für beliebige Abbildungen $f$.
             \begin{proof}
                 Es sind $f^{-1}(\emptyset) = \emptyset$ sowie $f^{-1}(X) = Y$ beide offen.
            \end{proof}
            \end{itemize}
    \end{enumerate}    
\end{example}

\begin{remark}
    Ist $Y$ diskret und  $X$ beliebig, so ist jede Abbildung  $f:X \to  Y$ stetig.
\end{remark}









